\documentclass{memoire}
\begin{document}
\renewcommand{\bibsection}{\section*{Bibliographie}}

% ORGANISATION GENERALE
% Le mémoire comprend une partie principale de synthèse, nécessairement courte, complétée par un dossier d’analyses détaillées.
% Rédiger la synthèse vous conduit à mettre en cohérence vos études et travaux, et à rendre accessible au jury, sous un format clair et rapidement lisible, l’essentiel de votre travail.
% Les analyses détaillées sont un récapitulatif de vos travaux : études, comptes-rendus, expérimentations et lectures faites en cours d’année.

% PARTIE SYNTHESE
% Son introduction résume de manière succincte l'origine du questionnement initial et explique le titre du mémoire. Elle introduit et justifie le plan du mémoire, en indiquant bien dans quelle partie la problématique est définie et justifiée. Dans le cas d’un mémoire préparé à deux ou à trois, elle explicite par ailleurs comment vous avez collaboré pour rédiger le mémoire à plusieurs (synthèse et analyses).
% La synthèse doit comprendre une partie de définition et une justification de la problématique, qui s'appuie sur une synthèse argumentée de ce qui vous y a conduit: vos lectures d'une part, et d'autre part certaines de vos observations ou expérimentations.
% Elle décrit et justifie également le corpus permettant de répondre à la problématique ou de la faire évoluer: entretiens, analyses de documents (manuels, copies, …), observations, construction de séances / séquences, expérimentations en classe, notes de lecture.
% Elle aboutit à un résumé de vos principales conclusions et propose des perspectives pour développer le travail engagé, en termes de nouvelles lectures, de poursuite d'étude (formations, diplômes certifications envisagées..) ou de valorisation de votre travail.
% Elle inclut une bibli- et sitographie aux normes et commentée pour les références les plus importantes (2-3 lignes de commentaire pour les lectures qui vous ont le plus servi).
% Longueur de l’ordre de 15 à 20 pages pour un mémoire individuel, 25 à 30 pages en binôme ou trinôme.
% Police et espacement, pour toute partie dactylographiée: Times 12 pt, interligne 1,5.

% la page de garde comporte les éléments et mentions suivants:
% le logo de l'ESPE de Créteil (obligatoire) + logo de votre université d'inscription (facultatif)
% "MÉMOIRE de MASTER MEEF 2nd degré, parcours mathématiques. Année 2018-19";
% Vos noms et prénoms;
% Le titre du mémoire;
% Le nom du ou des responsable(s) de suivi de mémoire;
\begin{titlepage}
    \begin{center}
        \includegraphics[scale=0.5]{logo_UPEC_ESPE.png}

        MEEF 2\up{nd} degré, parcours Mathématiques

        Année 2018-19 UE d'accompagnement de stage

        \vspace*{\fill}

        \Huge{Différenciation : quelles méthodes pour quels résultats ?}

        \vspace*{\fill}

        \begin{tabular}{ccc}
            \Large{Henriette, Julia}&\Large{Jidouard, Xavière}&\Large{Milis, Victoire}
        \end{tabular}

        \vspace*{\fill}

        \Large{Responsable de suivi de mémoire : Grapin, Nadine}

        \vspace*{\fill}
    \end{center}
\end{titlepage}


% Tous les mémoires doivent comporter un résumé de 200 à 250 mots et une liste de 5 mots-clés.
% la deuxième page comporte les éléments et mentions suivants:
% Un résumé de 300 à 500 mots. Il indique le contexte du mémoire, résume sa problématique et ses principales conclusions. (Contradiction avec le cadrage Mahara)
% Les autorisations de diffusion, sous la forme suivante:
% J'autorise l'ESPE:
% à exploiter le texte de mon mémoire dans la formation des étudiants MEEF  [OUI / NON]
% à communiquer mon nom et mes coordonnées à de futurs étudiants MEEF qui souhaiteraient me contacter au sujet de mon mémoire [OUI / NON]
\begin{abstract}
Ce mémoire rassemble les réflexions et expérimentations de 3 professeurs stagiaires
à propos de différentes méthodes de différenciation. Le but de ces méthodes
étant d'avoir des techniques appliquables généralement, en non spécialisées sur
certaines séquences.
\end{abstract}

Mots-clés : \textbf{différenciation}, \textbf{méthodes}, \textbf{autonomie}

\vfill

Nous autorisons l'ESPÉ :
\begin{itemize}
\item à exploiter le texte de notre mémoire dans la future formation des étudiants MEEF : \textbf{\textsc{Oui}} ;
\item à communiquer notre nom et coordonnées à de futurs étudiants MEEF qui souhaiteraient nous contacter au sujet de notre mémoire :\textbf{\textsc{Oui}}.
\end{itemize}


% La troisième page comporte le sommaire de la synthèse et inclut la liste des analyses détaillées.
\include{sections/toc}

% Le corps de la synthèse ne dépasse pas 30 pages et est organisé en parties et sous-parties lisibles.

% De l’ordre de 1 à 2 pages
% Elle résume de manière succincte l'origine du questionnement initial et explique le titre du mémoire.
% Elle introduit et justifie le plan du mémoire, en indiquant bien dans quelle partie la problématique est définie et justifiée.
% Elle explicite par ailleurs comment vous avez collaboré pour rédiger le mémoire à plusieurs (synthèse et analyses).
% La rédaction permet d'identifier les parties rédigées collectivement ou individuellement; les choix que vous avez faits pour organiser entre vous la rédaction, sont indiqués en introduction.
\section{Introduction}
Oui bonjour.\cite{articletest}


% La synthèse doit comprendre une partie de définition et une justification de la problématique, qui s'appuie sur une synthèse argumentée de ce qui vous y a conduit: vos lectures d'une part, et d'autre part certaines de vos observations ou expérimentations.
% Elle décrit et justifie également le corpus permettant de répondre à la problématique ou de la faire évoluer: entretiens, analyses de documents (manuels, copies, …), observations, construction de séances / séquences, expérimentations en classe, notes de lecture.
\section{synthèse}
La synthèse doit comprendre une partie de définition et une justification de la problématique, qui s'appuie sur une synthèse argumentée de ce qui vous y a conduit: vos lectures d'une part, et d'autre part certaines de vos observations ou expérimentations.
Elle décrit et justifie également le corpus permettant de répondre à la problématique ou de la faire évoluer: entretiens, analyses de documents (manuels, copies, …), observations, construction de séances / séquences, expérimentations en classe, notes de lecture.\\

\textbf{25 à 30 pages}

\section{Bilan des lectures}


\textit{<- Début du bilan d'étape ->}

tat des lieux des travaux menés depuis Novembre 
\subsection{Recherches documentaires et premières observations}

Nous avons tout d’abord ciblé nos lectures sur certains documents généraux (voir bibliographie en Annexe) faisant l’état des lieux de la différenciation pédagogique. Nous avons en effet découvert une quantité importante de documents abordant le sujet de manière théorique, soit de manière générale, soit sur un axe spécifique («Différencier avec les TIC», «Différencier sur une évaluation en mathématiques», etc.). La quantité de ressources disponibles étant trop importante, nous avons choisi de nous concentrer sur la documentation suggérée par Nadine, la page Eduscol sur la différenciation pédagogique, les documents du Conseil national d’évaluation du système scolaire (Cnesco) et le dossier de veille de l’Institut Français de l’Éducation (IFÉ).

Ces lectures ont amélioré notre compréhension de la différenciation pédagogique : amener tous les élèves à un niveau de connaissance et de compétence commun en tenant compte de leurs différences par des pratiques d’enseignement adaptées à chacun, pour que chaque élève avance à son rythme. Elles nous ont également permis de comprendre que, malgré qu’il existe un consensus sur ce que représente la différenciation pédagogique, elle peut être mise en place par des pratiques très variées (supports différenciés, mise en place de tutorat entre élèves, classe inversée, parcours différenciés, groupes de niveau, différenciation des productions attendues, etc.) pour des objectifs finaux différents (lutter contre le décrochage scolaire, assurer une égalité des chances scolaires, amener chaque élève au maximum de son potentiel, verrouiller l’acquisition d’un niveau de compétence et de connaissance minimal, faire gagner en autonomie les élèves, responsabiliser les élèves face à leur apprentissage, etc.) 

Les entretiens que nous avons effectués avec nos tuteurs terrains, notre directrice de mémoire et nos échanges avec nos collègues (dont les autres élèves stagiaires) nous ont permis de mûrir nos lectures, d’une part, et de spécifier notre axe de recherche, d’autre part.

Les visites des tuteurs ESPÉ et les travaux effectués en UE2 et UE4 nous ont également permis de faire un état des lieux de nos pratiques en classe. Nous avons constaté que nous ne faisions quasiment pas de différenciation en classe, hors différenciation successive avec des exercices supplémentaires pour les élèves les plus à l’aise, ou aide individuelle pouvant prendre la forme d’une piste donnée suite à une question posée1. 
En nous entretenant avec nos collègues, nous avons constaté que peu d’entre eux avaient une pratique de différenciation continue : certains ne cherchent pas spécialement à pratiquer une différenciation pédagogique en classe alors que d’autres adaptent leurs pratiques sur des séquences spécifiques où ils estiment nécessaire de différencier les variables didactiques ou le mode de travail des élèves.

Enfin, la visite de nos tuteurs ESPÉ a été pour nous l’occasion de questionner notre travail autour du mémoire dans une démarche à long terme (certifications, modification de nos pratiques, ambition de carrière). Par exemple, Julia a montré un intérêt pour la question des élèves à besoins particuliers en début d’année. L’entretien a été l’occasion de revenir sur la possibilité de croiser le travail de mémoire avec cet intérêt. 

\subsection{Impact des travaux sur notre réflexion}

Nos lectures et entretiens nous ont amenées à orienter notre réflexion sur la différenciation pédagogique dans le but de rendre plus autonomes nos élèves. Ce choix est motivé par des besoins différents pour chacune d’entre nous. Pour Xavière, qui enseigne en collège REP (5e, 4e et 6e AP),  le premier objectif est de rendre les élèves autonomes en classe pour avoir plus de temps à consacrer à l’accompagnement personnalisé des élèves et passer d’une gestion de classe globale systématique à une gestion de classe plus ponctuelle notamment avec les élèves en difficulté. Le second objectif est de donner à tous les élèves les outils d’autoévaluation nécessaires à un travail hors classe autonome. Sa tutrice terrain (Marie-Hélène Le Yaouanq) a validé ce choix, en cohérence avec les observations effectuées lors de visites et avec les contenus de cours déjà proposés. Victoire a validé ce choix suite à ses propres recherches sur le travail de groupe, qu’elle souhaitait établir dans ses classes de 6e et 4e. L’intérêt est notamment pour elle d’explorer la pratique du tutorat par les élèves et les îlots bonifiés. Julia souhaite de son côté réduire les écarts de niveau dans ses classes de 6e et 5e, en limitant son intervention lors des phases d’exercices pour accompagner les élèves nécessitant une assistance personnalisée.

Dès que nous avons choisi d’orienter nos recherches autour de l’autonomisation des élèves par la différenciation pédagogique, nous avons sélectionné dans nos lectures des pratiques que nous souhaiterions expérimenter en classe. 

Ainsi Victoire a choisi d’effectuer des recherches autour des permis de tutorat pour les élèves et des îlots bonifiés. De son côté, Xavière a préféré s’orienter vers la différenciation par la mise en place de parcours différenciés (en 4e) et d’images mentales variées (dans les trois classes), pour faciliter l’auto-évaluation des compétences. Julia a, quant à elle, d’abord opté pour la mise en pratique d’un parcours d’exercices adapté à la réussite des élèves, dans le but de les faire travailler en autonomie. Elle a finalement décidé de travailler à une autonomisation des élèves pour un travail différencié plus efficace à travers ce même parcours d’exercice, pour que chaque élève avance à son rythme. 
Nous avons toutes les trois choisi de ne pas explorer en profondeur les pratiques comme la classe inversée, la différenciation dans l’enseignement par les TICE, la différenciation dans l’évaluation, etc. Ces pratiques pourront cependant faire l’objet d’une mise en pratique occasionnelle en classe, elles ne sont cependant pas l’objet principal de notre mémoire. Elles pourront faire l’objet d’une analyse annexe.

Lorsque nous avons voulu mettre en place nos différentes méthodes de différenciation, nous avons eu à faire face à des difficultés de mise en place, dues à notre manque d’expérience en temps que professeurs, et surtout à notre manque d’évaluation de l’efficacité de nos méthodes. Pour palier le premier point, nous allons faire appel à différentes personnes avec qui nous avons pris contact : Victoire souhaite visiter une classe dans le premier degré, Julia va visiter un de ses collègues qui pratique une inversion de classe, et Xavière et Victoire peuvent facilement prévoir des visites croisées. Pour le second point, Nadine Grapin nous a recommandé de créer une métrique, par exemple en mettant en place un questionnaire élève afin de voir ce qu’ils en pensent. Bien entendu, il ne faudra pas que ce questionnaire reste notre seule métrique, on pourra aussi prendre en compte les notes, la qualité des exercices faits, la participation en classe…

\subsection{Évolution de notre réflexion}

Grâce aux échanges et aux recherches, nous avons pu découvrir de nouvelles méthodes et pistes de réflexion quant à la différenciation. Ces nouveaux éléments nous permettent de mettre en place des expérimentations, qui nous mèneront, à leur tour, à de nouvelles questions et autres constats.
Notre questionnement a évolué avec chaque recherche que nous effectuons. La première question, qui a évolué est « pourquoi différencier ?». Il y a plusieurs réponses possibles à cette question, chacune dépendante des pratiques et de l’environnement du professeur. Dans notre cas, la réponse a été que nous souhaitons différencier afin de pouvoir rendre les élèves plus autonomes et de faciliter la gestion de classe.

La seconde question qui vient est «comment différencier ?». Cette question a évolué en plusieurs questions qui seront la trame principale de nos réflexions :
\begin{itemize}
    \item quelles pratiques peut-on raisonnablement mettre en place pour effectuer une différenciation ? Nous manquons de temps et d’expérience, nous ne pouvons pas différencier chaque exercice ou point de cours. Nous avons besoin de méthodes pédagogiques ;
    \item sur quelles modalités pouvons nous différencier ? Le plus classique reste les variables didactiques dans un même exercice, un autre que des collègues utilisent déjà est un parcours différencié d’exercices, réactif aux erreurs commises… Il est aussi possible de faire de la différenciation au niveau des supports mis à disposition des élèves, mais là aussi, cela demande beaucoup de temps à mettre en œuvre.
\end{itemize}

Enfin, une question qui a beaucoup d’importance dans le cadre de notre réflexion est «comment évaluer l’impact des méthodes de différenciation mises en place?» Ce type de retour est essentiel à la mise en place de stratégies de différenciation afin de savoir si elles sont efficaces, inefficaces voire nocives. Dans notre cas, nous étudierons l’évolution de l’autonomie des élèves et de l’acquisition des compétences, et nous avons besoin de ces retours pour voir sur quels points améliorer les méthodes de différenciation que nous avons mises en place jusque là, qui ont eu des résultats mitigés. 

Pour l’instant, les exemples trouvés d’évaluations de l’autonomie des élèves sont dédiés au primaire ou à la maternelle, et nous n’avons aucun retour d’expérience suite à l’utilisation de telles évaluations. Cela nous pose problème dans l’exploitation de l’évaluation de l’autonomie des élèves : « Comment ajuster nos pratiques si les élèves ne sont pas évalués plus autonomes ou si les compétences ne sont pas acquises ? »

La suite de notre travail va donc à la fois se tourner vers la construction de pratiques différenciées à expérimenter, la mise en place d’un mode d’évaluation de l’autonomie et de l’acquisition des compétences par les élèves et l’analyse des évaluations pour ajuster nos pratiques.

\subsection{Recherches documentaires et collaborations extérieures supplémentaires}

Il nous reste des lectures à terminer parmi les documents que nous avions initialement identifiés pour la construction de notre réflexion, en particulier sur les pratiques de différenciation pédagogique que nous souhaitons développer (voir bibliographie). 

De plus, nous souhaitons effectuer des recherches sur l’évaluation de l’impact de nos pratiques sur l’autonomie et l’acquisition des compétences, mais manquons de ressources à ce sujet. Enrichir notre bibliographie sur ce point est donc notre principal objectif à court terme dans ce domaine.
Afin d’enrichir notre analyse de pratique, Victoire effectuera une visite dans une école primaire afin de pouvoir y analyser les pratiques mises en place pour l’autonomie des élèves, les élèves d’école primaire ayant tendance à être autonomes et à perdre cette autonomie au collège.  La visite reste à programmer, Victoire est en attente d’un retour de ses collègues du primaire. De même, Julia assistera au cours d’une collègue de français qui pratique la classe inversée. L’objectif sera d’observer le mode de fonctionnement de la classe en autonomie, le format d’évaluation des compétences, ainsi que les modes de différenciation mis en place. Xavière a assisté et assistera à nouveau à des cours d’EPS, pour étudier le format des consignes, la remédiation par la différenciation et leurs impacts sur l’autonomie des élèves.

\textit{<- Fin du bilan d'étape ->}
\section{Expérimentations}

\textit{<- Début du bilan d'étape ->}

Nous travaillons chacune de notre côté avec des points hebdomadaires de partage de connaissances et expériences. Nous avons également beaucoup d’échanges informels en dehors de l’ESPÉ sur un salon de discussion virtuel dédié. Nous discutons aussi chacune de notre côté avec nos encadrants et collègues, et rapportons les remarques au trinôme.
Afin de plus facilement retrouver ce que nous avons acté lors d’un point d’échange, nous mettons nos comptes-rendus sur Mahara, ainsi que nos fiches de lecture qui sont mises en commun avec un autre groupe de mémoire qui travaille aussi sur la différenciation (Fanny, Léa et Morgane).

Concernant la rédaction du mémoire en lui-même, nous allons utiliser LaTeX, outil avec lequel nous avons l’habitude de travailler, avec une gestion des versions et des commentaires sur GitHub. Ceci nous permettra de très facilement gérer et suivre les modifications de chacune et de passer moins de temps sur la mise en page.
Xavière souhaitait rédiger un article (ayant déjà rédigé un article scientifique par le passé pour valider un autre master). C’est à nouveau en réflexion, étant donné que la charge de travail a grandement augmenté depuis le dernier point.

Nous travaillerons chacune de notre côté sur la mise en place des pratiques que nous souhaitons étudier. Nous ferons cependant toutes les semaines une mise en commun de ce que nous avons effectué et des observations que nous avons faites.

\textit{<- Fin du bilan d'étape ->}
\subsection{Parcours différencié (Julia)}

\textit{<- Début du bilan d'étape ->}
\paragraph{}
Le principe du parcours différencié est de proposer aux élèves de traiter une série d'exercices avec une progression adaptée aux difficultés que ceux-ci peuvent rencontrer. Ce parcours peut être complété par des aides progressives données par l'enseignant ou sous différents supports (capsules vidéos, document avec rappel de cours, questions intermédiaires\ldots ).
\paragraph{}
Avant de mettre en place les parcours différenciés, j'ai eu besoin de voir comment les élèves réagissaient face à une successions de travaux à faire en autonomie. J'ai donc mis en place une séance d'exercices en autonomie avec ma classe de sixième. \\
Les élèves ont commencé avec un exercice sur les fractions à faire seul, la correction et les exercices suivants se trouvant sur un îlot. J'ai ajouté une version "bis" de certains exercices disponibles pour les élèves ayant rencontré des difficultés ou voulant se rassurer. \\
Les élèves ont globalement joué le jeu, ils ont apprécié de pouvoir avancer à leur  rythme et de circuler librement dans la salle même si cela a causé de l'agitation en début de cours. Certains élèves ont cependant choisi de ne faire qu'un ou deux exercices simples, alors que d'autres se sont rapidement jetés sur les corrections.\\
\newline
Lors de cette expérience, je me suis rendue compte des élèves qui devaient a priori me poser des problèmes d'attitude étaient plus épanouis dans ce type de travail (lorsqu'ils faisaient l'effort d'essayer), alors que des élèves très à l'aise en condition \emph{classique} de travail individuel ont été perdus au démarrage de la première séance. De plus, une fois le dispositif en place, des élèves habituellement très discrets m'ont demandé de l'aide. J'ai cependant eu de grandes difficultés à \emph{jongler} entre les élèves. Ceux-ci étaient encore peu autonomes et avaient surtout besoin d'être rassurés. J'ai également eu du mal à trouver le temps à laisser à chacun et à identifier les moments où mettre le travail en commun.\\
Cela m'a confortée dans l'idée que la construction de parcours différenciés, avec des étapes clés et à réaliser en autonomie par les élèves pouvait avoir un apport pédagogique intéressant pour mes classes.

\paragraph{à modifier}
J'ai décidé de lancer ma première expérimentation de parcours différencié avec ma classe de cinquième pour la séquence \emph{Symétrie centrale}. Nous avons commencé la séquence par un travail de rappels sur la symétrie axiale, suivie par une activité de l'IREM de Lille sur la construction par symétrie centrale.\\
J'ai construit un parcours d'exercices applicatifs sur la construction de symétriques. Le parcours  

S’il est à l’aise et a “réussi” (modalités à définir) son exercice, l’élève continue sur la branche “OK”. Des exercices clefs sont corrigés en plénière en début de séance. Les élèves peuvent également demander de l’aide à l’enseignant ou à des camarades (si les modalités de travail et l’enseignant le permettent).

Des exercices supplémentaires avec des variables didactiques plus complexes peuvent également être prévus en cours de parcours pour les élèves les plus avancés (énoncés donnant peu d’informations, question ouverte, demi-droite graduée à tracer par l’élève, etc.).

Afin de suivre leur propre progression, les élèves colorient leur parcours et notent la date sous le dernier exercice effectué en séance. Je le garde avec moi afin de suivre à distance l’évolution des élèves et d’identifier les exercices ayant posé le plus de problèmes.

Les objectifs sont multiples : 
\begin{itemize}
    \item Permettre à tous les élèves d’acquérir les compétences à son rythme.
    \item Faire gagner les élèves en autonomie, ils deviennent ainsi acteurs de leur formation. 
    \item Pour les élèves en difficulté, proposer des exercices complémentaires avec des variables didactiques différentes et adaptées aux difficultés rencontrées.
    \item Donner aux élèves les plus à l’aise des exercices leur permettant de développer des compétences supplémentaires ou de renforcer les compétences attendues.
\end{itemize}

\textit{<- Fin du bilan d'étape ->}
\subsection{Tutorat entre élèves (Victoire)}
\subsubsection{Motivations}
Le tutorat entre élèves peut être une autre source d'apprentissage : pendant que
l'un explique, l'autre apprend.

L'enseignement en lui-même est une forme d'apprentissage : la personne qui
enseigne doit assez bien maitriser la notion, les méthodes et le vocabulaire
utilisé afin de pouvoir l'enseigner à quelqu'un d'autre. Savoir enseigner
demande plus que simplement maitriser une technique ; cela demande de savoir
pourquoi on peut utiliser une technique, et quel raisonnement se trouve derrière
la technique.

Permettre et encourager les élèves à enseigner serait donc une autre méthode de
faire apprendre une notion à un élève. C'est aussi une opportunité pour un élève
qui n'aurait pas compris d'avoir un autre regard sur une notion, et un cours
individualisé. C'est le raisonnement derrière la « méthode Feynman », une méthode
qui applique la méthode d'enseignement de Richard Feynman\footnote{\url{https://www.youtube.com/watch?v=_f-qkGJBPts} vidéo présentant la méthode Feynman.}
.
\subsubsection{Risques et contraintes}

Tout comme on n'imaginerait pas un enseignant ne pas maitriser son domaine, il
faudra s'assurer que l'élève maitrise le sien avant de l'enseigner à un autre
élève.

Il y a également un risque de créer ou d'aggraver une scission dans la classe :
les tuteurs pourraient être vus comme des élèves étant privilégiés, et ainsi créer
une hiérarchie dans la classe.\cite{pedagogie_cooperative_hierarchie}

Il faut également bien maitriser le minutage de sa séance afin de pouvoir controler
à quels moments les élèves peuvent se déplacer dans la salle de classe : il faut
pouvoir reprendre la séance rapidement, sans à avoir à attendre plusieurs minutes
que tout le monde se rassissent.

\subsubsection{Expérimentations}

\paragraph{Mise en autonomie sans préparation}

Une des méthodes de mise en autonomie qui parait le plus simple d'accès est de
simplement les laisser faire et d'autoriser les élèves à s'entraider. J'ai
considéré cette possibilité pour plusieurs raisons :
\begin{itemize}
    \item elle ne demande pas de préparation particulière préalable ;
    \item les règles sont simples pour les élèves ;
    \item elle permet l'apprentissage entre élèves (voir les bénéfices au paragraphe précédent) ;
    \item les mathématiques sont une discipline propice à la transmission de savoir et de savoir-faire entre élèves : une définition, une propriété ou une méthode peuvent facilement
    se partager et être discutés entre élèves ;
    \item personne ne m'a dit de ne pas faire ça\footnote{On ne sait jamais, ça aurait pu fonctionner}.
\end{itemize}

Les objectifs de cette expérimentation étaient :
\begin{enumerate}
    \item permettre aux élèves avancés de ne pas « rien faire » ;
    \item permettre aux élèves ayant des difficultés d'avoir une aide plus
    rapidement ;
    \item augmenter l'expertise des élèves avancés, les forçant à adopter d'autres
    façons de voir la notion ;
    \item réduire le nombres d'appels au professeur, restant ainsi disponible
    pour les questions avancées ;
    \item réguler le niveau de bruit\footnote{Ne pas confondre bruit et niveau sonore} de la classe.
\end{enumerate}

Lors d'une séance d'exercices, j'ai donc autorisé les élèves qui voulaient
aider leurs camarades à se lever. Les premiers élèves demandaient à se lever, et
avaient bel et bien fini correctement leurs exercices. Au fur et à mesure, des
élèves avaient pris la liberté de se lever, sans avoir forcément fini leurs exercices,
et sans forcément avoir correctement répondu. Et je ne suis pas persuadée que
tous les élèves s'entraidaient effectivement.

Je considère que cette première expérimentation s'est donc soldée par un échec, car
l'objectif de réguler le niveau de la classe n'était pas atteint : certes certains
élèves sérieux aidaient d'autres élèves qui en avaient besoin, mais d'autres tiraient
profit de la situation pour bavarder, et ainsi perturber d'autres élèves qui
auraient pu réussir les exercices dans des conditions normales.

Pire encore : certains élèves de bonne foi, avaient fini leurs exercices et aidaient
leurs camarades, mais leurs réponses et méthodes étaient fausses, et propagaient
ainsi de mauvaises méthodes au reste de la classe\footnote{Et il est extrêmement difficile de
rattraper le coup après ceci} !

En conclusion, cette méthode de tutorat « improvisé » ne fonctionne pas (pas de réduction, voire augmentation
du bruit en classe), et est même contre-productive car elle nuit au développement
de certains élèves.

\paragraph{Tétra'aide}

\textit{Voir annexe \ref{tetraaide}}

Le concept du tétra'aide est simple : un tétraèdre permet à l'élève d'indiquer
son besoin d'aide en fonction du sommet orienté vers le haut :
\begin{enumerate}
    \item Tout va bien
    \item J'ai une question non-urgente
    \item À l'aide !
    \item J'aide ou je suis aidé par quelqu'un
\end{enumerate}

Utiliser le tétra'aide permet d'avoir une meilleure visualisation de ceux qui ont
besoin d'aide, et de faire un meilleur triage parmi ceux qui ont besoin d'aide :
ceux qui bloquent vraiment sur l'exercice et ont une question technique, et ceux
qui ont fini ou ont une question qui n'a pas directement rapport au cours.

Objectifs de l'utilisation du tétra'aide :
\begin{enumerate}
    \item pouvoir effectuer un triage des questions urgentes et non-urgentes ;
    \item réduire le bruit dû aux appels au professeur ;
    \item tenter d'intégrer le tutorat entre élèves ;
    \item permettre aux élèves n'osant pas lever la main de pouvoir poser des questions.
\end{enumerate}

J'ai mis le système du tétra'aide en place à la mi-février, en leur fournissant
le patron (qu'ils devaient colorier et assembler à la maison) et en leur expliquant
le fonctionnement. Nous avons mis en œuvre le tétra'aide à la séance suivante,
qui était une séance d'exercices. Les élèves ont assez vite compris le principe,
levant de moins en moins la main au fil de la séance.

Après plusieurs semaines, malgré des explications répétées, certains élèves ont du mal à avoir le réflexe
d'utiliser le tétra'aide et vont plutôt lever la main pour demander mon attention,
parfois pour des questions non-urgentes. Je pensais que ce système allait permettre
aux élèves timides, qui ont du mal à lever la main, de pouvoir profiter de mon aide.
Au final, ceux qui utilisent le tétra'aide sont les même qui osaient lever la main
en début d'année.

Petit à petit, certains élèves ont adopté une attitude de consommateur : si
leur tétra'aide était en position « j'ai besoin d'aide », ils exigeaient une
aide immédiate et sans attendre. Certains élèves jouaient avec leur tétra'aide ou
détruisaient celui d'un autre élève : pour pallier ceci, il faudrait mettre
le patron à disposition sur un espace en ligne auquel les élèves ont l'habitude
d'accéder\footnote{Ce qui est loin d'être le cas : les élèves (en tout cas de mes
classes) ne vont quasiment jamais sur le cartable en ligne, et savent à peine
l'utiliser}.

Une remarque d'ordre pratique : en fonction de la disposition des tables, il
peut être pertinent d'imprimer les patrons dans une taille plus grande : à
l'origine ce système a été fait pour les écoles primaires, beaucoup plus variées
et inventives au niveau des dispositions de tables. Dans une disposition classique
en « rangs d'ognon\footnote{Voir rectifications orthographiques de 1990 en cas de doute
sur l'orthographe} » les tétra'aides sont perdus au milieu des trousses, crayons
et cahiers des élèves. Imprimer en A3 peut être une bonne idée.

\subsubsection{Piste de réflexion : aménagement de l'espace classe}

L'aménagement de l'espace classe peut permettre une meilleure collaboration entre
élèves. La disposition classique (en « autobus » ou « rangs d'ognon ») ne permet
qu'une interaction possible : entre le professeur et les élèves.

D'autres dispositions permettent une meilleure collaboration entre élèves : par
exemple, en ilots bonifiés\cite{ilots_bonifies} (voir explications en annexe \ref{ilots_bonifies_annexe}), ou en U\cite{amenagement_classe}.

Lors d'un remplacement, j'ai pu tester la méthode en ilots bonifiés qui était mise
en place par la professeur absente. Pour résumer : les tables sont organisés
en ilots permettant d'asseoir 4 élèves (2 tables face à face). Chaque ilot a une
fiche de bonification : à chaque bonne action (les exercices ont été faits, le travail
en séance est fait, participation orale de chaque membre de l'ilot), l'ilot gagne un point.
En cas de mauvaise action (bavardage…) il est possible de retirer un point.

Dès qu'un ilot a accumulé 20 points, les points de tous les ilots sont transformés
en note sur 20 ajoutée à la moyenne\footnote{Avec un coefficient faible en général},
et à la prochaine séance, les configurations des ilots changent.

Je n'ai pas pu tester cette disposition cette année, car je partage ma salle de classe
avec ma tutrice (Mme Sandrine Adam), qui n'était pas partante pour tester ce genre de disposition : elle a déjà testé d'autres dispositions (notamment en U), et elles
ne convenaient pas à ses techniques d'enseignement, et, apparemment, causerait plus
de mal que de bien\footnote{Et qui suis-je pour remettre en cause la parole d'une
excellente professeur ?}.

\subsection{Évaluations différenciées (Victoire)}
\subsubsection{Motivations}

L'idée de différencier les évaluations m'est venue au contact de Julia et Xavière,
qui évaluent toutes les deux par compétences. Mon collège est encore resté aux
notes, et je voulais faire en sorte de pouvoir évaluer mes élèves sur des
compétences, et donc leur proposer plusieurs manières de prouver leur maitrise
d'une compétence. Différencier les évaluations est donc un choix naturel.

\subsubsection{Comment je l'ai découvert}
Avec Sandrine, nous étions toutes les deux
intéressées par les évaluations différenciées. Au départ, nous voulions tester
ceci sous sa forme la plus simple : pour un exercice, les élèves pourraient
choisir une version simple, qui rapporterait $n$ points, ou\footnote{Ici, un ou
exclusif} une version plus experte, qui rapporterait $m$ points\footnote{$n < m$}.

Alors que nous allions passer à la rédaction de nos sujets, Julia me fournit
un article traitant des évaluations différenciées\cite{differenciation_devoir_surveille}.
En lisant cet article, je me rends compte que la manière dont Sandrine et moi
allions justement procéder d'une manière qui n'est pas recommandée par les
auteurs\footnote{Cette fois-ci, il y a vraiment quelqu'un pour nous dire de ne
\textbf{pas} faire ce que nous avions prévu.} !

\subsubsection{Expérimentation}

Les auteurs recommandent, plutôt que de proposer aux élèves le choix entre 2
versions d'un exercice, de proposer un « buffet d'exercices » : par exemple, proposer
12 exercices à 2 points chacun\footnote{Les valeurs en points de chaque exercice
peut varier bien sûr}, et laisser aux élèves choisir quels exercices ils veulent
effectuer. Dans cet exemple, il est possible d'obtenir jusqu'à 24 points, et
donc la note de 24/20 !

J'ai donc rédigé un sujet\footnote{Voir annexe \ref{sujet_differencie}}, avec les objectifs suivants
en tête :

\begin{itemize}
    \item permettre aux élèves de prouver leur maitrise d'une notion par différentes manières ;
    \item faire en sorte qu'un élève \textbf{doive} au moins couvrir chaque notion une fois ;
    \item donner du challenge aux élèves en avance ;
    \item redonner confiance aux élèves en difficulté mais qui travaillent.
\end{itemize}


Résultats :
Peu d'effets sur les élèves non-travailleurs et forts, amélioration de quelques
points pour les moyens (exam trop dur)
injustices chez les 6e

\subsubsection{Pistes de réflexion}

DS moins durs
Appliquable à des petites interros sur des chapitres passés ?
Faire ça de manière plus régulière ?
Quid des ceintures de compétences ?

\subsection{Différenciation par les supports (Xavière)}

% BILAN D'ETAPE

%Au départ, je suivais plusieurs pistes de différenciation. En particulier, je souhaitais instaurer un parcours différencié pour mes deux classes principales, selon le modèle présenté par Julia. Ma tutrice ESPÉ m’a fortement conseillé de me limiter à ma classe de 4e, plus difficile à mettre au travail.
%
%J’ai également créé une évaluation à jokers avec ma classe de 4e. Les jokers sont des petits papiers contenant une indication sur l’exercice que je distribue aux élèves en échange d’une pénalité minime sur la note. Cette évaluation a obtenu un franc succès.
%J’ai également commencé le travail d’auto-évaluation avec mes 5e. Ce qu’il manque pour approfondir ce point, c’est créer une métrique adaptée. Pour le moment je n’ai pas de retour quantifiable.
%
%Sur les conseils de Nadine Grapin, je me suis concentrée sur un axe de recherche en particulier, et j’ai choisi l’auto-évaluation. Pour cela, je souhaite m’intéresser aux supports.
%
%Mes classes sont caractérisées par la présence de mauvais lecteurs (des élèves en très grande difficulté en Français, sans toutefois présenter des troubles dys). Il est donc primordial de m’assurer que :
%\begin{enumerate}
%    \item ils se construisent des images mentales correctes et cela ne peut pas toujours passer par une compréhension de la trace écrite. J’ai remarqué, en particulier pour la classe de 5e, que les images mentales dynamiques (nécessitant des manipulations) étaient plus facilement assimilées et restituées par les élèves.
%    \item les consignes données à l’oral sont parfaitement comprises de tous
%\end{enumerate}
%
%De plus, comme je vise l’autonomie des élèves à la fois en classe et hors classe, je dois compléter cette approche par d’autres supports, en particulier des fiches de correction. Pour le moment, je les guide sur la construction de fiches de révision agréables à l’oeil pour que les élèves s’y réfèrent plus volontiers. Je souhaite me tourner progressivement vers des fiches de correction, puis d’auto-correction, avec un guide de construction de fiches créé par les élèves. L’objectif final est de produire des grilles d’auto-évaluation.
%Tout ceci est très expérimental et je ne suis pas encore en capacité de mesurer quantitativement les conséquences de la multiplication des supports

%{\color{red}Tous les passages écrits en rouge dans cette partie sont des remarques destinées à améliorer le contenu, souvent sur la forme, ou sont des réflexions ou des questionnements.}
%
%\subsubsection{Motivations}
%
%{\color{red}Rappels en vrac de ma ligne directrice pour mes expérimentations. À reformuler et à remettre en contexte. Ce sont les conseils donnés par ma tutrice en début d'année que j'ai essayé d'appliquer de mon mieux, à l'exception des vidéos.}
%
%\begin{itemize}
%\item Se renseigner sur la différenciation en axant sur les images mentales ;
%\item diversifier les images mentales (toutes ne fonctionnent pas sur tous les élèves) ;
%\item diversifier les images mentales grâce aux supports :
%\begin{itemize}
%\item cartes mentales (en particulier celles centralisant toutes les façons de répondre à un problème donné, par exemple comment montrer que deux droites sont parallèles) ;
%\item manipulations manuelles/gestuelles ;
%\item utilisation de geogebra par les élèves (et non par moi) ;
%\item vidéos explicatives ;
%\item fiches de méthodologie cartonnées, idéalement faites spontanément par les élèves ;
%\end{itemize}
%\item le but est de fixer par écrit les images mentales.
%\end{itemize}
%
%\paragraph{Quelques détails}
%
%Le but est que \textbf{les élèves créent eux-mêmes leurs outils}. La priorité est aux fiches méthodes, fiches erreur et cartes mentales. Sur les fiches méthodes, lorsque nous voyons une méthode technique en cours, les élèves en font une fiche cartonnée qui leur servira de référence lors des exercices et évaluations. La fiche erreur est construite par l'enseignant sur la base de plusieurs publications des élèves. Je leur donne un bilan récapitulatif des erreurs à éviter. Eux en font une fiche qui complète la fiche méthode. Pour la carte mentale, il est convenu avec ma tutrice que je leur fournisse une base qu'ils complètent.
%
%Je peux aussi leur fournir une carte mentale de résumé de cours, mais comme cela consiste essentiellement en de la recopie du cours, \textbf{ne pas la considérer comme un exercice mathématique} à part entière et ne pas leur donner à compléter.
%
%{\color{red}Faire de tout ce paragraphe un tableau synthétique.}
%
%Le but est de leur permettre de s'emparer, de s'approprier les notions vues en classe et de leur constituer {\color{red}(ou apprendre à constituer)} une banque de ressources qu'ils utiliseront en classe sur exercices ou à la maison pour s'auto-corriger.
%
%Ce travail sera essentiellement \textbf{étudié sur ma classe de 5\up{e}}, mais ceci est proposé sur les deux classes.
%
%Note : ma tutrice ESPÉ m'a conseillé de me limiter aux parcours différenciés aux 4\up{e}. Ceci étant couvert par Julia, je n'en parlerai pas. {\color{red}Retour très intéressant de Claire sur le problème du saucissonnage sur la séquence Pythagore, cela vaut sans doute la peine d'en parler en discussion.}\\
%
%{\color{red} Pour ce qui suit, j'ai le sentiment que cela fait beaucoup trop. J'ai entamé un très gros travail sur la symétrie et j'ai également des scans de copies d'élèves appuyant la nécessité de corriger une mauvaise conception pour les angles alternes-internes. La partie sur le calcul algébrique est sans doute dispensable si je n'en ai pas le temps et je manque de productions d'élèves pour l'étayer. Enfin, la dernière partie sera uniquement axée sur le traitement des erreurs à éviter (ou dans le cas présent, des éléments manquants). J'ai à la fois les productions d'élèves scannées, le fichier regroupant les productions les plus intéressantes et la fiche méthodologique construite par les élèves. Les cartes mentales ont surtout été vues avec les 4e, sur le principe elles reprennent ce qu'on a fait avec les fiches méthodologiques. J'ai encore une séquence à finir et une autre à faire avec mes 5e avant de pouvoir faire des cartes mentales intéressantes portant sur plusieurs séquences.}

%\subsubsection{Symétries : manipulations manuelles et sur geogebra - fiches de méthodologie}
%
%Le but est de montrer comment j'ai ajouté des couches successives pour renforcer les deux images mentales principales en symétrie axiale et symétrie centrale. J'ai toujours un problème avec mon élève italien qui a tendance à effectuer des translations.
%
%\paragraph{Les cocottes en symétrie}
%
%\paragraph{Remplacement progressif par les mains}
%
%\paragraph{Utilisation de Geogebra en auto-correction}
%
%Lien vers ma séance TICE. Probablement pas aussi développé que le reste.
%
%\paragraph{Fiches de méthodologie}
%
%\paragraph{Retours, productions d'élèves}
%
%Lien vers mon analyse d'évaluation + la toute dernière évaluation faite. Parler de mon élève italien ?
%
%\subsubsection{Angles alternes-internes : comment corriger une potentielle image fausse}
%
%\paragraph{Situation qui a amené les élèves à créer cette image mentale fausse}
%
%Lien vers mini-évaluation + stats rapides sur présence de l'erreur dans les copies.
%
%\paragraph{Utilisation de Geogebra pour corriger une potentielle image mentale fausse}
%
%\paragraph{Une autre approche : la libellule et la coccinelle}
%
%Très rapide, l'évaluation récente montre que cette image mentale n'est pas très populaire au sein de ma classe, alors qu'elle est très utilisée dans les deux autres classes de 5\up{e} (le timing où elle a été présentée est différent).
%
%\paragraph{Initiation à la démonstration : erreurs et correction par les pairs}
%
%{\color{red} Je ne sais pas si celui-ci a sa place dans ce mémoire, dans la mesure où cela se passe intégralement à l'oral}
%
%\paragraph{Cartes mentales - droites parallèles / calcul d'angles}
%
%Les deux sont envisageables et liées à cette séquence. {\color{red}Je ne sais pas si j'aurai le temps de les amorcer avant la date de rendu du mémoire.}
%
%\subsubsection{Calcul algébrique : deux approches différentes pour toucher un maximum d'élèves}
%
%\paragraph{Nombres relatifs - approche vectorielle}
%
%\paragraph{Nécessité de recourir à une deuxième approche}
%
%\paragraph{Nombres relatifs - approche [nom à définir plus tard]}
%
%\subsubsection{Gestion de données : trouver les erreurs à éviter}
%
%\paragraph{Problème initial et productions d'élèves}
%
%\paragraph{Fiche méthodologique résultante}
%
%\subsubsection{Retours élèves}

\subsubsection{Motivations}

Le but de mon expérimentation était d'amener les élèves de ma classe de 5\up{e} à un niveau suffisant d'autonomie pour créer eux-mêmes leurs propres outils de travail (voir \textsc{Figure \ref{org:xav}} ci-dessous). Pour atteindre ce but, j'ai exploré plusieurs pistes et je souhaite en aborder deux en particulier : le travail effectué autour de la symétrie dans un premier temps et le travail effectué sur les angles alternes-internes dans un deuxième temps.

\begin{figure}[h!]
    \centering
    \tikzstyle{cat1} = [rectangle, rounded corners, minimum width=3cm, minimum height=1cm, text centered, text width=3cm, draw=black, fill=black!0]
\tikzstyle{cat2} = [rectangle, minimum width=3cm, minimum height=1cm, text centered, text width=3cm, draw=black,  fill=black!10]
\tikzstyle{cat3} = [rectangle, minimum width=5cm, text centered, text width=5cm, minimum height=1cm, draw=black,  fill=black!10]

\tikzstyle{arrow} = [thick,->,>=stealth]

\begin{tikzpicture}[node distance=3cm]
\node (auto) [cat2] {Développer l'autonomie};
\node (img) [cat2, right of=auto, xshift=7cm] {Images mentales correctes};
\node (pb) [cat3, below of=auto, xshift=5cm] {Élèves assez autonomes pour construire par eux-mêmes les outils pour s'approprier une notion};
\node (outils) [cat2, below of=pb] {Outils};
\draw [arrow] (auto) |- (pb);
\draw [arrow] (img) |- (pb);
\draw [arrow] (pb) -- (outils);
\node (fiche1) [cat1, left of=outils, xshift=-2cm] {Fiches erreur};
\node (fiche2) [cat1, below of=outils, xshift=-5cm, yshift=1cm] {Fiches méthodologiques};
\node (carte1) [cat1, below of=outils, yshift=1cm] {Cartes mentales bilan};
\node (geogebra) [cat1, right of=outils, xshift=2cm] {GeoGebra et tableur};
\node (carte2) [cat1, below of=outils, xshift=5cm, yshift=1cm] {Cartes mentales transversales};
\draw [arrow] (outils) -- (fiche1);
\draw [arrow] (outils) -- (fiche2);
\draw [arrow] (outils) -- (geogebra);
\draw [arrow] (outils) -- (carte1);
\draw [arrow] (outils) -- (carte2);
\end{tikzpicture}

    \caption{Organigramme des ressources nécessaires pour amener des élèves à créer leurs propres outils en toute autonomie}
    \label{org:xav}
\end{figure}

\paragraph{Présentation succincte des outils que les élèves sont amenés à utiliser sur l'année}

\remark{Ajouter une phrase d'introduction ici}
\begin{itemize}
	\item [Les fiches méthodologiques :] accompagnement de cours, elles se présentent sous la forme d'une méthode écrite sur une fiche cartonnée et illustrée selon les goûts des élèves. Les élèves les utilisent lorsqu'ils travaillent sur des exercices ou lors des devoirs avec ma permission.
	\item [Les fiches erreur :] ces fiches viennent en complément des fiches méthodologiques. Elles sont utilisées pour aider les élèves à valider ou invalider un résultat. C'est un axe de recherche que je n'ai pas assez développé cette année. J'ai limité mes expérimentations à la présentation d'extraits de copies à mes élèves pour générer des débats. Il en a résulté des fiches méthodologiques au lieu de fiches erreur.
	\item [Les cartes mentales :] initialement, si l'on se réfère à la  \textsc{Figure \ref{org:xav}}, il était prévu que je travaille sur deux types de cartes mentales avec les élèves. Le premier type est une carte rappelant les notions importantes du cours, telles qu'on les trouve dans les leçons de cycle 3 ou les manuels de cycle 4. Elle sert de bilan et est surtout utilisée lors des révisions. Les élèves en ont réalisé quelques-unes tout au long de l'année, toujours d'après un modèle établi en commun en classe. \\
	Le second type est une carte mentale transversale, répondant à une question\footnote{Par exemple : "Comment prouver que deux droites sont parallèles ?"} qui peut être traitée à l'aide de procédures issues de plusieurs séquences. J'ai manqué de temps et de recul sur le programme pour mettre ce type de carte mentale en place.
	\item [L'utilisation des TICE] (Technologies de l'Information et de la Communication pour l'Enseignement) : les élèves manipulent régulièrement en cours d'année les logiciels utilisés en mathématiques (Geogebra et les tableurs) pour apprendre à valider un résultat ou une conjecture de manière autonome.
\end{itemize}

\subsubsection{Pourquoi la différenciation est importante pour l'autonomie des élèves}

Lorsqu'on cherche à rendre ses élèves autonomes, on se rend vite compte que la différenciation intervient à plusieurs niveaux. Si l'on se réfère à l'organigramme présenté en \textsc{Figure \ref{org:xav}}, la première entrée est la poursuite du développement de l'autonomie, entrepris au cycle 3. J'ai suivi deux axes :
\begin{itemize}
    \item faire acquérir aux élèves le réflexe d'utiliser leurs outils (relire une fiche méthodologique quand on bloque devant un exercice, se référer à la fiche-erreur pour valider un résultat, utiliser les cartes mentales pour réviser...) ;
    \item développer le travail en autonomie en classe, l'objectif au troisième trimestre étant de m'assurer que mes élèves sont en activité un tiers du temps.
\end{itemize}
Concernant le premier point, la différenciation passe essentiellement par une attention particulière aux consignes données et le choix des procédures de résolution (différenciation simultanée \cite{Eduscol}). Ce point ne sera pas détaillé ici.

Concernant le second point, Il faut distinguer deux moments forts du cours lorsqu'on travaille sur les images mentales avec les élèves. Pour rappel, une image mentale est "une représentation d'une information sensitive sans perception d'un stimulus externe" \cite{mimagery}. Cette représentation peut être mémorisée ou imaginée. En mathématiques, les images mentales servent à donner du sens à un concept non perceptible par nos sens. 

Dans un premier temps, j'établis avec mes élèves une image mentale (ou plusieurs) à laquelle se référer lorsqu'ils rencontrent un nouveau concept. La différenciation porte sur la variété des supports présentés aux élèves (différenciation successive \cite{Eduscol}).

Dans un deuxième temps, après avoir observé les élèves travailler sur le nouveau concept avec leur(s) image(s) mentales, j'entre parfois dans une phase de remédiation, l'image mentale présentée initialement étant mal assimilée ou incomplète. La remédiation passe soit par la présentation d'une autre image mentale, soit par un changement de support pour une même image mentale.

Le travail présenté dans ce mémoire porte sur ces deux temps de fabrication d'une image mentale correcte :
\begin{itemize}
    \item une première sous-partie est consacrée à variété des supports utilisés pour passer d'une perception de la symétrie (axiale et centrale) à une image mentale en variant les sens utilisés ;
    \item une seconde sous-partie est consacrée à la remédiation d'une mauvaise conception des angles alternes-internes.
\end{itemize}

La différenciation intervient également au niveau des outils eux-mêmes : les logiciels nécessitent une phase de prise en main et un de mes axes cette année a été de proposer des exercices différenciés sur support informatique, adaptés à l'aisance des utilisateurs avec les logiciels. J'en parle très brièvement au moment où j'aborde la section consacrée aux symétries.

\subsubsection{Différencier pour acquérir une image mentale : la symétrie}

%Le but est de montrer comment j'ai ajouté des couches successives pour renforcer les deux images mentales principales en symétrie axiale et symétrie centrale. J'ai toujours un problème avec mon élève italien qui a tendance à effectuer des translations.

J'ai commencé à construire la séquence dédiée à la symétrie axiale à l'aide du livre \textit{Des maths ensemble et pour chacun 5\up{e}} \cite{mepcc} (séquence 2, pages 88-103). La première activité présentée utilise des silhouettes de cocottes et les indications à destination de l'enseignant emploient également les cocottes (voir \textsc{Figure \ref{fig:cocottes}} et annexe \ref{annexe:symetrie-act}). J'ai repris cette idée pour mon expérimentation.

\begin{figure}[h!]
    \centering
    \includegraphics[width=0.6\linewidth]{img/cocottesmepcc.png}
    \caption{Figure accompagnant les explications destinées à l'enseignant et pouvant être photocopiées et distribuées aux élèves}
    \label{fig:cocottes}
\end{figure}

\paragraph{Création d'un outil à manipuler : les cocottes}

En premier lieu, avant de leur donner l'activité du livre qui consistait à classer des couples de cocottes grises et blanches par type de transformation (symétrie axiale, symétrie centrale et translation), j'ai appris aux élèves à créer deux cocottes identiques, excepté un signe distinctif, avec une feuille de papier. Lorsqu'ils ont démarré l'activité, ils superposaient la cocotte blanche sur la cocotte grise et essayaient de trouver quelle manipulation ils faisaient avec les mains pour arriver au résultat présenté sur la feuille d'activité.

Très vite, les élèves ont retrouvé l'image du miroir pour la symétrie axiale et le mouvement rectiligne pour la translation. Ils ont également découvert que leurs cocottes effectuaient un arc de cercle pour la symétrie centrale.

Les cocottes sont pratiques pour deux raisons : premièrement, elles représentent concrètement les figures de l'énoncé. Les élèves n'ont pas à faire un effort d'abstraction supplémentaire pour passer de la manipulation à la situation de l'énoncé. Deuxièmement, elles se glissent toutes dans une pochette et sont facilement transportables. Même si elles sont égarées, elles sont très rapides à fabriquer.

\paragraph{Remplacement progressif par les mains}

La phase de consolidation de l'image mentale s'est faite en deux temps. Tout d'abord, j'ai présenté aux élèves des exercices où les figures n'étaient plus des cocottes, pour qu'elles prennent le statut d'outil. Puis certains élèves ont compris qu'ils effectuaient les mouvements avec les mains qui tenaient les cocottes et ont commencé à faire les manipulations avec les mains seulement.

À ce moment, j'ai moi-même commencé à évoquer les images mentales du miroir et de la rotation avec mes mains sans tenir les cocottes. Le comportement s'est généralisé à l'ensemble de la classe.

\paragraph{Fiches méthodologiques}

Lors du travail sur cette séquence, les élèves ont construits deux fiches méthodologiques pour la symétrie axiale et deux fiches méthodologiques pour la symétrie axiale. 

La première fiche détaille la construction de l'axe de symétrie ou du centre de symétrie. Elle a été construite en commun avec la classe après leur avoir fait construire les axes de symétrie et les centres de symétrie à nouveau sur la première activité de la séquence. La fiche résulte du débat généré à la fin de l'exercice et synthétisé par un élève. À ce moment de l'expérimentation, les élèves notaient sur leur feuille des flèches de rotations comme celles que l'on voit sur la \textsc{Figure \ref{fig:cocottes}} pour évoquer l'image mentale qu'ils s'étaient construite.

La seconde fiche détaille la construction du symétrique d'une figure par symétrie axiale ou centrale.

Des exemples sont disponibles dans l'annexe \ref{annexe:symetrie-fiches}.

Ces fiches ont été reproduites sur papier cartonné et étaient utilisées lors des exercices d'entrainement sur la durée ou lors de certains devoirs surveillés.

\paragraph{Utilisation de Geogebra}

J'ai beaucoup travaillé cette séquence sur la durée et je me suis rendue compte en cours d'année que certains élèves oubliaient l'image mentale s'ils ne travaillaient pas assez régulièrement dessus. Pour les aider, sur un conseil de ma tutrice terrain, Madame Hizembert, j'ai créé une séance en salle informatique pour leur apprendre à utiliser de façon autonome Geogebra pour s'entrainer chez eux. 

Un résumé de la séance est présent dans l'annexe \ref{annexe:symetrie-tice}

Avec le recul que j'ai à présent et suite à des discussions avec Madame Hizembert, j'aurais pu utiliser davantage Geogebra pour renforcer les images mentales des élèves. La séance informatique est intervenue trop tard dans l'année. J'ai également découvert que le manuel \textit{Des maths ensemble et pour chacun 5\up{e}} s'accompagnait d'un site\footnote{Site compagnon : http://edition.crdp-nantes.fr/?id=maths-ensemble-et-pour-chacun} permettant de télécharger des ressources qui permettent de visualiser l'image mentale de la rotation avec Geogebra.

\subsubsection{Remédier à une mauvaise conception : les angles alternes-internes}

\subsubsection{Retours des élèves}

Cette partie de l'expérimentation n'est pas encore menée. Il s'agit d'un questionnaire à faire remplir par les élèves, en particulier sur l'utilisation des fiches méthodologiques chez eux, sur l'utilisation de GeoGebra/Excel.

% Parle peut-être plutôt d'un tableur, plutôt qu'Excel. Si on peut éviter de mentionner
% des logiciels privateurs quand des alternatives libres sont disponibles, tant mieux !
% Il y a déjà assez de lobbying de la part des GAFAM dans l'EN…

\section{Discussions}
Dans cette partie, nous faisons le point sur nos pratiques avec un regard critique enrichi par les retours de nos collègues et de nos élèves, et confrontés aux expérimentations d'autres collègues de l'ESPE. Nous aborderons également la question de l'impact des pratiques expérimentées sur l'acquisition des connaissances et des compétences de nos élèves, et sur leur capacité à travailler en autonomie. Nous traiterons enfin des autres dispositifs mis en place lors de notre année de stage et dont l'analyse (brève) peut, selon nous, enrichir ce mémoire.
\subsection{Analyse croisée de nos pratiques}
Comme indiqué dans la partie \ref{Expérimentations}-Expérimentations, nous avons effectué des choix d'expérimentation différents au regard de nos besoins et de nos recherches littéraires.\\
\remark{Je trouve pas vraiment de titre, je pense que dire ce que nous avons en commun dans nos pratiques ou non et pourquoi nous avons fait ces choix est le plus judicieux}
\subsubsection{Comparaison de nos expérimentations}
\paragraph*{}Nous avons toutes choisies de mettre en place des dispositifs incluant tous élèves de nos classes. Ces dispositifs ont des impacts plus ou moins importants sur les élèves selon leur niveau, mais nous avons choisi de tous les inclure dans nos dispositifs.\\
Dans les faits, le tutorat des élèves, les exercices de remédiation ou les défis des parcours différenciés vont avoir une incidence plus importante sur les élèves en grande difficulté ou au contraire très à l'aise sur les compétences travaillées. Les évaluations différenciées et les supports adaptés à chaque élève sont en revanche supposées avoir un impact sur l'autonomie et l'acquisition des compétences de tous les élèves, sans distinction de niveau.\\
\remark{Je ne sais pas si ce que j'écris vous paraît pertinent. Je pense que c'est une première différence intéressante entre les dispositifs. Pour la suite, @Xavière, je te laisse modifier à ta guise!}
\paragraph*{}Dans la construction des parcours différenciés, Xavière et Julia ont toutes les deux mis en place des parcours de formes différentes. Alors que Julia propose un guide de parcours ainsi qu'une série d'exercices selon la réussite ou non de l'élève, Xavière propose une feuille d'exercice avec une version plus ou moins difficile de l'exercice (variation dans le choix des variables didactiques) et l'élève choisit celui qu'il traite \remark{il me semble, de mémoire}. Dans les deux cas, un élève en difficulté aura la possibilité de traiter un exercice plus facile pour appréhender une notion, et les élèves les plus à l'aise auront accès à des problèmes plus difficiles à aborder.\\
La différence de présentation présente cependant une différence d'approche très intéressante car dans l'expérimentation de Julia, les élèves n'ont pas le choix de traiter la version facile d'un exercice ou le défi de leur parcours. Ils doivent également identifier par eux-même une situation d'échec ou de fragilité dans le traitement d'un exercice, qui demande d'effectuer un exercice supplémentaire. Au contraire, l'approche de Xavière donne le choix aux élèves dans le niveau de l'exercice à traiter. D'un côté on peut s'attendre à ce que les élèves à l'aise essaient toutes les versions ou que les élèves ne choisissent que la facilité, d'un autre côté, un élève en difficulté ne passera pas par une étape "d'échec" avant de traiter un exercice plus abordable.
\paragraph*{}
\remark{Vous avez toutes les deux testé des évaluations différenciées mais pas du tout sous la même forme (je crois que Xavière c'était plus dans le sens de s'auto évaluer), si ça prend du sens, vous pourriez peut-être en parler ici?}\\
\remark{Pour la suite j'ai déplacé la partie "notre retour" après les retours des autres. Comme ça on peut s'appuyer sur cette partie pour alimenter notre propre retour}
\subsubsection{Retours de nos collègues}
\paragraph*{}
\remark{Retours de nos tutrices sur nos expérimentations et analyse de travaux de collègues de l’ESPE. J'ai listé les interlocuteurs mais on peut également les citer chacun leur tour et dire ce qu'ils nous ont apporté à chaque fois par exemple.}
Dans le cadre du mémoire, nous avons pu échanger sur nos expérimentations avec Nadine Grapin, notre responsable de suivi. Elle nous a en particulier aidées à identifier les modalités de différenciation ou de mise en autonomie dans nos pratiques. Elle nous a également indiqué quels collègues de la formation avaient potentiellement des approches communes ou très différentes des notres. Nous avons ainsi eu l'occasion d'échanger avec le groupe de Fanny Mauhé, Morgane Petigat et Léa Serrano\cite{memoire_fanny} en particulier lors des phases de documentation, ainsi qu'avec le groupe formé par Cédric Hamon et Juliette Kirouane\cite{memoire_eval_differenciee}, en particulier lors de la journée de valorisation des mémoires.\\
Nous avons enfin présenté nos pratiques à nos tutrices terrain qui ont pu assister à leurs mises en application en classe et nous faire part de leurs observations.
\paragraph*{}
Les différents échanges avec nos interlocuteurs a mis en avant des paramètres à prendre en compte et que nous n'avions pas toujours anticipé dans la préparation de nos expérimentations. La forme donnée aux supports est, par exemple, ressortie de la part de tous nos interlocuteurs (avec plus ou moins d'importance). Ainsi, la tutrice de Julia a indiqué que certains parcours présentés aux élèves de 6\up{e} avaient une forme inadaptée, voir décourageante pour les élèves, alors qu'elle serait adaptée à une classe de 4\up{e} (après quelques ajustements). Dans leur travail sur la différenciation, le groupe de Fanny, Morgane et Léa a également observé que la forme des supports de travail ou de différenciation avait une grande importance sur la mise au travail des élèves \remark{si j'ai bien compris?}\\
De même l'identification des différents temps de classe est apparue encore plus importante que lors de la préparation d'une séance "normale" pour la plupart des pratiques analysées. En effet, la mise en place d'îlots bonifiés, de tétra-aide ou de parcours de différenciation demande de connaître par avance les temps de travail individuel ou collectif, les période de déplacement des élèves (et leurs buts), les durées approximatives de chaque période, le possibles sources d'agitation ou de sollicitations importantes\ldots Il nous a notamment été demandé lors des analyses de séance d'identifier les différents moments de nos séances, leur objectif et le déroulement anticipé puis observé.\\
Ces temps de classes vont être variables selon le travail effectué mais aussi en fonction des élèves. Par exemple, une classe de 6\up{e} va généralement demander plus d'explications sur ce qui doit être fait qu'une classe de 4\up{e}. \remark{@victoire : Les élèves de 4\up{e} sont ils plus timides dans l'entraide ? Mes collègues m'ont dit que le tutorat après la 4e ça marchait qu'entre copains et encore! Question de puberté}. De plus, la maturité des élèves aura un impact sur sa capacité à effectuer le travail demandé en autonomie. Ainsi, des élèves de 6\up{e} auront plus de difficulté à identifier quelle série d'exercices ils seront capables de faire dans le temps imparti pour optimiser leur note. Pour les mêmes raisons, ces élèves ne seront pas forcément aptes à s'auto-évaluer dans des exercices de raisonnement de parcours différenciés (pour des exercices de géométrie par exemple). Nous devons donc également adapter nos pratiques au niveau d'autonomie de nos élèves sur les tâches demandées et leur apprendre à les effectuer pour la suite de leur scolarité (voir notamment la réflexion de Philippe Meirieu\cite{Meirieu_autonomie} sur le sujet).\\
\remark{Différences/similitudes avec travaux de groupe sur éval différenciée}
Les visites de nos tutrices en séance nous ont permis d'identifier les différentes variables didactiques sur lesquelles nous pouvons différencier dans nos dispositifs. Ainsi, l'analyse d'exercices effectués en classe a mis en avant l'importance de la question des objectifs à atteindre par les élèves dans la résolution de ceux-ci, mais aussi des objectifs leur correction. Cette dernière question est d'autant plus importante dans la construction des parcours différenciés et des évaluations différenciées où les élèves ne traitent pas les mêmes exercices et où la correction n'est pas toujours effectuée en classe entière.\\
Malgré les nombreux échanges entre nous ou avec nos collègues, il a été rapidement clair qu'un dispositif qui fonctionne lors d'une séance ne fonctionnera pas forcément avec une autre classe ou même à un autre horaire. Nous avons tenté d'identifier les paramètres de nos expérimentations qui permettent de rendre nos dispositifs ré-exploitables dans le futur (forme des supports, organisation de la salle, découpage de la séance, choix des exercices\ldots).

\subsubsection{Nos retours}
Dans cette partie, chacune d'entre nous effectue un retour de sa propre expérimentation et commentera les dispositifs mis en place par les deux autres. \remark{si ça vous va, sinon on organise ça par dispositif ou autre :)}\\
\paragraph*{Retour de Julia :}
Le travail de recherche et d'expérimentation sur les parcours différenciés a été très enrichissant pour moi car il m'a demandé de travailler sur la plupart des points de construction d'une séance. \\
\remark{à compléter Variables didactiques ,ce que je garde comme paramètres :eval diagnostique, forme, temps, aides, gestion de la salle\ldots}\\
Je le referai mais pas dès la rentrée de la 1ère année ou de manière light. Dès le début de la 2e année. Plutôt sur séquences de calcul ou constructions géométriques. Forme modifiée, intégrer construction "à la Xavère"\\
Je pense utiliser le tétra-aide avec mes élèves mais pas avec un parcours différencié (plutôt tutorat là) + îlots (pas forcément bonifiés) : objectif responsabiliser les élèves et modifier les formes d'apprentissage.\\
Évaluations différenciées intéressant dans une classe découragée + préparation aux examens je trouve (motiver les élèves, éviter le décrochage).
Math's up réutilisé pour les formes d'apprentissage.

\paragraph*{Retour de Xavière :}

\remark{1.	Le (re)ferais-je ?
	2.	De quelle manière (ce que je garderais) ? Avec qui ou sur quelle séquence ?
	3.	Pour quel objectif ?
4. autres expérimentations non abordées dans le mémoire que je referai ou non}
\paragraph*{Retour de Victoire :}
\remark{1.	Le (re)ferais-je ?
	2.	De quelle manière (ce que je garderais) ? Avec qui ou sur quelle séquence ?
	3.	Pour quel objectif ?
	4. autres expérimentations non abordées dans le mémoire que je referai ou non}

\paragraph*{}
Nous avons effectué nos premières expérimentations assez tôt dans l'année (fin octobre-début novembre) et il nous est arrivé de ne pas pouvoir effectuer notre séance normalement à cause de problèmes de gestion de classe. Au fil de l'année nous avons fait évolué nos pratiques, mais la période de "domptage" \remark{je ne trouve pas de bon mot} de la classe a été une période difficile pour construire notre pratique de différenciation. C'est pourquoi nous \remark{ou juste Julia ?} attendrions d'avoir plus d'assurance \remark{je ne trouve pas de bon mot} avec nos futures classes avant de ré-expérimenter certaines de nos pratiques (parcours différenciés, \remark{si vous pensez avoir d'autres pratiques qui demandent de maîtriser sa classe avant d'être mises en place})
\paragraph*{Retour de nos élèves}
\remark{Transition avec la suite. A ECRIRE}
\subsection{Mesurer l’impact de notre travail}
a.	Notre regard sur l’impact
b.	Difficulté à mesurer
c.	Conseils reçus et mis en place ou non
\subsection{Autres expérimentations liées à la différenciation ou l’autonomie}
Nos principales expérimentations n'ont pas été les seuls dispositifs de différenciation ou de travail sur l'autonomie que nous avons menés en classe.
a.	Présentation rapide
Quoi ? Pourquoi ? Modalités ? En complément/indépendante du reste ?
b.	On garde/on jette
\subsection{Retour à la problématique / aux problématiques}
a.	Avons-nous trouvé un dispositif adapté ?
b.	« ré exploitation possible ? » si oui. Modification/autre piste sinon


% La synthèse  aboutit à un résumé de vos principales conclusions et propose des perspectives pour développer le travail engagé, en termes de nouvelles lectures, de poursuite d'étude (formations, diplômes certifications envisagées..) ou de valorisation de votre travail.
\section{Conclusion}
Nous avons chacune trouvé que nos dispositifs permettaient de répondre en partie à nos problématiques autour du gain en autonomie et en compétence par un travail différencié. Nous avons cependant trouvé que le temps de préparation de ces dispositifs était long, surtout lors d'une année de stage où nous avons besoin de plus de temps pour préparer nos séquences et séances. De plus, les problèmes de gestion de classe ne nous ont pas permis d'expérimenter dans des conditions optimales. Nos dispositifs pédagogiques nous ont tout de même permis de mieux connaitre nos élèves et de mieux cerner nos capacités à gérer leurs attentes et leurs difficultés.\\
Nous avons le sentiment que varier les dispositifs de différenciation pédagogique permet d'atteindre une plus grande autonomie de nos élèves. Cette autonomie gagnée permet elle-même à nos élèves de travailler plus efficacement sur des dispositifs différenciés à pratiquer en autonomie (et à apprendre à apprendre). Ils maitrisent donc plus facilement les compétences attendues, en passant moins de temps à se demander quoi faire, comment, avec quelles informations\ldots \\
Ce travail de mémoire a clairement enrichi notre réflexion d'enseignantes en nous donnant des pistes de recherche sur les pratiques pédagogiques, en nous habituant à échanger sur nos pratiques avec nos collègues et surtout en nous apprenant à remettre continuellement nos pratiques et notre vision de l'enseignement en question. Nous avons apprécié travailler ensemble sur des pratiques différentes car cela nous a permis de découvrir plus de pratiques sur un an et nous a donné des pistes de réflexion pour les années à venir. Nous sommes convaincues que nous devons encore ajuster les pratiques expérimentées cette année pour les ré-exploiter et nous sommes convaincues que d'autres pratiques pédagogiques de différenciation vues chez nos collègues ou aperçues dans nos lectures nous permettront d'atteindre cet objectif. Nous n'avions pas imaginé en débutant qu'autant de dispositifs existaient en matière de différenciation et d'autonomie !

% La bibliographie est placée à la fin de la synthèse.
% Les références principales (celles qui ont le plus servi) doivent être commentées : à quoi elles ont été utiles, et pourquoi. S’il y a une note de lecture détaillée, y renvoyer.
% La bibliographie est à éditer aux normes APA.
% Évitez les URL sans indication aucune de l'auteur et du contenu. Pour plus de détail sur la manière de «bien référencer», voir le site «fralica» d’un collègue Belge, Philippe van Goethem : http://www.ecoles.cfwb.be/ismchatelet/fralica/importskynet/refer/theorie/annex/refbibl.htm
% Renvois à la bibliographie dans votre texte : dans le texte principal, renvoyez à toute référence sous la forme (Parzysz 2007, p.30) si vous utilisez un élément de la page 30.
\nocite{*}
\addcontentsline{toc}{section}{Bibliographie}
\bibliographystyle{plain}
\bibliography{bibliographie}



% PARTIE ANALYSES DETAILLEES
% Cette partie peut prendre la forme d’un ensemble de titre, avec des liens renvoyant à des documents en ligne.
% Chaque analyse correspondant à chacune des études qui viennent éclairer la problématique décrite et justifiée dans la synthèse. Exemples :
% Notes de lecture, notes de synthèse bibliographique ; analyse de programmes, de documents d'accompagnement, de manuels, ou d'autres documents institutionnels
% Analyse d’expérimentations, analyses didactiques; analyses épistémologiques ou historiques ;
% Analyses de copies d'élèves, d'entretiens ciblés, de questionnaires.
% Elles sont placées de préférence après la synthèse, avec des titres clairs et lisibles qui sont repris dans le sommaire. Si le document est en ligne copiez le dans le mémoire ou bien indiquez le lien.
% Si la taille et le format sont libres, l’important est que le document soit accessible et clairement référencé. Le titre doit indiquer la nature de l’étude, et le ou les auteurs être indiqués.
% Mettre des noms d’auteurs pour chaque analyse détaillée, et la date de rédaction.
\begin{appendices}
    \section{Tétra-aide}\label{tetraaide}
J'inclus ici l'intégralité du document du Tétra'aide, accessible à cette adresse :
\url{http://bdemauge.free.fr/tetraaide.pdf}. Un grand merci à Bruce Demaugé-Bost,
professeur des écoles en classe multi-âges de CE2-CM1-CM2 à Vaulx-en-Velin.
Son site est accessible à l'adresse \url{http://bdemauge.free.fr/}

\begin{center}
\includegraphics[scale=0.2]{annexes/01_tetraaide_0.png}
\includegraphics[scale=0.2]{annexes/01_tetraaide_1.png}
\includegraphics[scale=0.2]{annexes/01_tetraaide_2.png}
\end{center}

    \section{Sujet d'évaluation différenciée}\label{sujet_differencie}
\begin{center}
    \includegraphics[scale=0.2]{annexes/sujet_differencie0.png}
    \includegraphics[scale=0.2]{annexes/sujet_differencie1.png}
\end{center}

    \begin{frame}
    \frametitle{Julia 1}
\end{frame}

\begin{frame}
    \frametitle{Julia 2}
\end{frame}

    \subsection{Construction du troisième parcours différencié en sixième}
\subsubsection*{Evaluation diagnostique}\label{Eval_diag_ju}
\begin{figure}[!h]
	\center{\includegraphics{img/eval_diag_julia.jpg}}
	\caption{Evaluation diagnostique}
\end{figure}
\subsubsection*{Productions d'élèves}
\begin{figure}[!h]
	\centering
	\subfloat[copie n°1]{{\includegraphics[scale=0.4]{img/copie1_eval_diag_ju.jpg}}}
	\qquad
	\subfloat[copie n°2]{{\includegraphics[scale=0.5]{img/copie2_eval_diag_ju.jpg}}}
	\qquad
	\subfloat[copie n°3]{{\includegraphics[scale=0.5]{img/copie3_eval_diag_ju.jpg}}}
	\caption{Evaluations d'élèves}
	\label{fig:Eval_diag_copies}
\end{figure}
\subsubsection*{Analyse des productions d'élèves}
\begin{figure}[!h]
	\centering
	\includegraphics[scale=0.75]{img/Analyse_eval_diag_ju.jpg}
	\caption{Analyse des évaluations}
\end{figure}
\subsubsection*{Parcours différenciés}\label{parcours_diff3}
Le 3\up{e} parcours différencié sera plus détaillé à l'oral. Ci-dessous le parcours "triangles".
\begin{figure}[!h]
	\centering
	%\includegraphics[scale=0.4]{img/parcours_triangles.jpg}
	\caption{Parcours triangle}
\end{figure}
\paragraph{}Tous les élèves effectuerons le parcours « \textit{3- Je construis} » car la classe de 6\up{e} est, selon moi, la classe où l'on peut le plus effectuer des constructions géométriques. Cette phase est effectuée lors d'une séance en demi-groupe.\\
Je distribuerai les parcours aux élèves en les dispensant de faire certains exercices (ou au contraire en leur demandant de commencer par l'exercice de remédiation).\\
Ainsi, parmi les élèves précédents, l'élève 1 commencera par l'exercice 24 p 204 puis il pourra effectuer l'exercice 74p204 (son parcours « \textit{Je reconnais} » sera une succession de flèches horizontales). L'élève 2 aura le parcours tel que présenté mais je lui demanderai de ne pas faire l'exercice "défi" tant qu'il n'aura pas terminé le parcours « \textit{Quadrilatères} ». L'élève 3 aura l'intégralité du parcours à effectuer, s'il est retard car il a passé beaucoup de temps sur le parcours « \textit{Quadrilatères} » par exemple, il n'aura pas à faire l'exercice 8 de l'étape « \textit{3- Je construis} ».
\paragraph{} Les exercices de l'étape « \textit{3- Je reconnais} » sont des exercices où les élèves doivent reconnaître les différentes représentations de triangles particuliers à l'aide de codages. Le parcours se termine par la reconnaissance des triangles au sein d'une figure complexe.\\
L'étape « \textit{3- Je construis} » comporte des programmes de constructions simple de triangles. En cas de non réussite de l'élève, celui-ci doit construire le même type de triangle mais à partir d'un croquis. Il aura ainsi un lien entre le programme de construction écrit, le croquis et la figure finale. Le parcours se termine sur une réflexion autour de la manière de construire une figure complexe composée de triangles et à partir d'un croquis. Cet exercice fait le lien avec le parcours suivant.\\
Dans « \textit{4- Je raisonne} », l'élève construit à partir d'un programme de construction, reconnaît les triangles particuliers à partir de leur définition et conjecture sur la nature des triangles au sein d'une figure complexe à l'aide de mesures effectuées. L'exercice défi demande une plus grande maîtrise de la langue française et des pratiques de raisonnement (« \textit{Sachant que...} », exercice avec aucune indication).
    \section{Différenciation pour la symétrie}

\subsection{Activité des cocottes}\label{annexe:symetrie-act}

Énoncé initial : Chacune des cocottes blanches a été obtenue à partir de la grise par une transformation différente.  Classe les couples de cocottes en fonction des transformations utilisées pour les construire.

\begin{figure}[h!]
    \centering
    \includegraphics[width=0.9\linewidth]{img/activitemepcc.jpg}
    \caption{Source : \textit{Des maths ensemble et pour chacun 5\up{e}}}
    \label{fig:angles-fiche1}
\end{figure}

\subsection{Fiches méthodologiques}\label{annexe:symetrie-fiches}

Les fiches méthodologiques sont scannées et présentes à cette adresse : \url{https://vvvictoire.github.io/memoire-meef/fiches-methodologiques.html}

\subsection{Activité en salle informatique}\label{annexe:symetrie-tice}

Pour cette séance, suite à plusieurs évaluations diagnostiques ponctuelles, j'avais décidé de retravailler la symétrie axiale et la symétrie centrale. L'objectif de cette séance est donc de leur permettre de consolider l'image mentale qu'ils se font des symétries en travaillant chez eux sur GeoGebra pendant les vacances. Pour cela, je leur ai préparé un exercice à réaliser seul sur GeoGebra pour se familiariser avec le logiciel et pour apprendre à s'auto-évaluer avant de refaire les activités seul à la maison.

Le fichier GeoGebra contient trois lettres dont les élèves devaient tracer le symétrique (par rapport à un point et par rapport à une droite). Dans un premier temps, ils devaient tracer le symétrique sur fond blanc et dans un deuxième temps en s'aidant du quadrillage. Enfin, ils devaient placer les axes de symétrie et les centres de symétrie de chaque lettre.

GeoGebra avait déjà été montré aux élèves comme un \textit{outil de conjecture}. Dans cette séance, c'est la première fois qu'ils utilisaient Gogebra en tant qu'\textit{outil de contrôle et de remédiation}. J'avais particulièrement insisté sur l'utilisation de l'outil \textit{Symétrie axiale} et de l'outil \textit{Symétrie centrale} comme \textbf{moyens de vérification} et non comme aides visuelles.

L'analyse complète de la séance, la fiche élève et le fichier GeoGebra associé sont disponibles à cette adresse : \url{https://vvvictoire.github.io/memoire-meef/extraits-de-portfolio.html}

\subsection{Analyse d'évaluation portant sur la symétrie}\label{annexe:symetrie-eval}

Dans un exercice de cette évaluation, les élèves devaient préciser la méthode employée pour construire le symétrique d'une ligne brisée connaissant l'emplacement d'un des segments du symétrique.

\begin{figure}[h!]
    \centering
    \includegraphics[width=\linewidth]{img/page1-exo2.png}
    \caption{Les propriétés de conservation de la symétrie centrale venaient d'être vues, peu d'élèves avaient eu le temps de se les approprier.}
    \label{fig:xav-eval}
\end{figure}

En dehors des évaluations à joker et des évaluations adaptées à un élève Italien en difficulté avec la langue française, ceci est la seule question différenciée proposée dans toutes mes évaluations. Cela n'est pas clairement précisé dans l'analyse de cette évaluation, mais il est important de préciser que cette question n'apportait pas de point pour la note chiffrée, bien qu'elle apparaisse dans la grille d'évaluation par compétences.

L'analyse complète de l'évaluation est disponible à cette adresse : \url{https://vvvictoire.github.io/memoire-meef/extraits-de-portfolio.html}

\clearpage

\section{Différenciation pour les angles alternes-internes}

\subsection{Évaluations diagnostiques}\label{annexe:angles-prod1}

Voici quelques citations extraites de copies d'élèves. La question initiale était :

Les captures d'écran associées à ces citations sont consultables à cette adresse : \url{https://vvvictoire.github.io/memoire-meef/angles-productions-d-eleves.html}

\subsection{Fiches d'exercice}\label{annexe:angles-fiches}

\begin{figure}[h!]
    \centering
    \includegraphics[width=0.6\linewidth]{img/anglesfiche1.jpg}
    \caption{Figures accompagnant les explications destinées à l'enseignant et pouvant être photocopiées et distribuées aux élèves}
    \label{fig:angles-fiche1}
\end{figure}

\begin{figure}[h!]
    \centering
    \includegraphics[width=0.8\linewidth]{img/anglesfiche2.jpg}
    \caption{Figures accompagnant les explications destinées à l'enseignant et pouvant être photocopiées et distribuées aux élèves}
    \label{fig:angles-fiche1}
\end{figure}

\begin{figure}[h!]
    \centering
    \includegraphics[width=0.6\linewidth]{img/anglesfiche3.jpg}
    \caption{Figures accompagnant les explications destinées à l'enseignant et pouvant être photocopiées et distribuées aux élèves}
    \label{fig:angles-fiche1}
\end{figure}

\clearpage

\subsection{Histoire de la coccinelle et de la libellule}\label{annexe:angles-prod2}

\begin{figure}[h!]
    \centering
    \includegraphics[width=\linewidth]{img/insectes-point.jpg}
    \caption{Une version où la coccinelle et la libellule sont dessinées.}
    \label{fig:angles-fiche1}
\end{figure}

\begin{figure}[h!]
    \centering
    \includegraphics[width=\linewidth]{img/insectes.jpg}
    \caption{Une version évoluée où la coccinelle et la libellule sont schématisées par des points.}
    \label{fig:angles-fiche1}
\end{figure}

\end{appendices}
\end{document}
