\subsection{Différenciation par les supports (Xavière)}

\textit{<- Début du bilan d'étape ->}

Au départ, je suivais plusieurs pistes de différenciation. En particulier, je souhaitais instaurer un parcours différencié pour mes deux classes principales, selon le modèle présenté par Julia. Ma tutrice ESPÉ m’a fortement conseillé de me limiter à ma classe de 4e, plus difficile à mettre au travail.

J’ai également créé une évaluation à jokers avec ma classe de 4e. Les jokers sont des petits papiers contenant une indication sur l’exercice que je distribue aux élèves en échange d’une pénalité minime sur la note. Cette évaluation a obtenu un franc succès.
J’ai également commencé le travail d’auto-évaluation avec mes 5e. Ce qu’il manque pour approfondir ce point, c’est créer une métrique adaptée. Pour le moment je n’ai pas de retour quantifiable.

Sur les conseils de Nadine Grapin, je me suis concentrée sur un axe de recherche en particulier, et j’ai choisi l’auto-évaluation. Pour cela, je souhaite m’intéresser aux supports.

Mes classes sont caractérisées par la présence de mauvais lecteurs (des élèves en très grande difficulté en Français, sans toutefois présenter des troubles dys). Il est donc primordial de m’assurer que :
\begin{enumerate}
    \item ils se construisent des images mentales correctes et cela ne peut pas toujours passer par une compréhension de la trace écrite. J’ai remarqué, en particulier pour la classe de 5e, que les images mentales dynamiques (nécessitant des manipulations) étaient plus facilement assimilées et restituées par les élèves.
    \item les consignes données à l’oral sont parfaitement comprises de tous
\end{enumerate}

De plus, comme je vise l’autonomie des élèves à la fois en classe et hors classe, je dois compléter cette approche par d’autres supports, en particulier des fiches de correction. Pour le moment, je les guide sur la construction de fiches de révision agréables à l’oeil pour que les élèves s’y réfèrent plus volontiers. Je souhaite me tourner progressivement vers des fiches de correction, puis d’auto-correction, avec un guide de construction de fiches créé par les élèves. L’objectif final est de produire des grilles d’auto-évaluation.
Tout ceci est très expérimental et je ne suis pas encore en capacité de mesurer quantitativement les conséquences de la multiplication des supports

\textit{<- Fin du bilan d'étape ->}