\subsection{Différenciation par les supports (Xavière)}

% BILAN D'ETAPE

%Au départ, je suivais plusieurs pistes de différenciation. En particulier, je souhaitais instaurer un parcours différencié pour mes deux classes principales, selon le modèle présenté par Julia. Ma tutrice ESPÉ m’a fortement conseillé de me limiter à ma classe de 4e, plus difficile à mettre au travail.
%
%J’ai également créé une évaluation à jokers avec ma classe de 4e. Les jokers sont des petits papiers contenant une indication sur l’exercice que je distribue aux élèves en échange d’une pénalité minime sur la note. Cette évaluation a obtenu un franc succès.
%J’ai également commencé le travail d’auto-évaluation avec mes 5e. Ce qu’il manque pour approfondir ce point, c’est créer une métrique adaptée. Pour le moment je n’ai pas de retour quantifiable.
%
%Sur les conseils de Nadine Grapin, je me suis concentrée sur un axe de recherche en particulier, et j’ai choisi l’auto-évaluation. Pour cela, je souhaite m’intéresser aux supports.
%
%Mes classes sont caractérisées par la présence de mauvais lecteurs (des élèves en très grande difficulté en Français, sans toutefois présenter des troubles dys). Il est donc primordial de m’assurer que :
%\begin{enumerate}
%    \item ils se construisent des images mentales correctes et cela ne peut pas toujours passer par une compréhension de la trace écrite. J’ai remarqué, en particulier pour la classe de 5e, que les images mentales dynamiques (nécessitant des manipulations) étaient plus facilement assimilées et restituées par les élèves.
%    \item les consignes données à l’oral sont parfaitement comprises de tous
%\end{enumerate}
%
%De plus, comme je vise l’autonomie des élèves à la fois en classe et hors classe, je dois compléter cette approche par d’autres supports, en particulier des fiches de correction. Pour le moment, je les guide sur la construction de fiches de révision agréables à l’oeil pour que les élèves s’y réfèrent plus volontiers. Je souhaite me tourner progressivement vers des fiches de correction, puis d’auto-correction, avec un guide de construction de fiches créé par les élèves. L’objectif final est de produire des grilles d’auto-évaluation.
%Tout ceci est très expérimental et je ne suis pas encore en capacité de mesurer quantitativement les conséquences de la multiplication des supports

{\color{red}Tous les passages écrits en rouge dans cette partie sont des remarques destinées à améliorer le contenu, souvent sur la forme, ou sont des réflexions ou des questionnements.}

\subsubsection{Motivations}

{\color{red}Rappels en vrac de ma ligne directrice pour mes expérimentations. À reformuler et à remettre en contexte. Ce sont les conseils donnés par ma tutrice en début d'année que j'ai essayé d'appliquer de mon mieux, à l'exception des vidéos.}

\begin{itemize}
\item Se renseigner sur la différenciation en axant sur les images mentales ;
\item diversifier les images mentales (toutes ne fonctionnent pas sur tous les élèves) ;
\item diversifier les images mentales grâce aux supports :
\begin{itemize}
\item cartes mentales (en particulier celles centralisant toutes les façons de répondre à un problème donné, par exemple comment montrer que deux droites sont parallèles) ;
\item manipulations manuelles/gestuelles ;
\item utilisation de geogebra par les élèves (et non par moi) ;
\item vidéos explicatives ;
\item fiches de méthodologie cartonnées, idéalement faites spontanément par les élèves ;
\end{itemize}
\item le but est de fixer par écrit les images mentales.
\end{itemize}

\paragraph{Quelques détails}

Le but est que \textbf{les élèves créent eux-mêmes leurs outils}. La priorité est aux fiches méthodes, fiches erreur et cartes mentales. Sur les fiches méthodes, lorsque nous voyons une méthode technique en cours, les élèves en font une fiche cartonnée qui leur servira de référence lors des exercices et évaluations. La fiche erreur est construite par l'enseignant sur la base de plusieurs publications des élèves. Je leur donne un bilan récapitulatif des erreurs à éviter. Eux en font une fiche qui complète la fiche méthode. Pour la carte mentale, il est convenu avec ma tutrice que je leur fournisse une base qu'ils complètent.

Je peux aussi leur fournir une carte mentale de résumé de cours, mais comme cela consiste essentiellement en de la recopie du cours, \textbf{ne pas la considérer comme un exercice mathématique} à part entière et ne pas leur donner à compléter.

{\color{red}Faire de tout ce paragraphe un tableau synthétique.}

Le but est de leur permettre de s'emparer, de s'approprier les notions vues en classe et de leur constituer {\color{red}(ou apprendre à constituer)} une banque de ressources qu'ils utiliseront en classe sur exercices ou à la maison pour s'auto-corriger.

Ce travail sera essentiellement \textbf{étudié sur ma classe de 5\up{e}}, mais ceci est proposé sur les deux classes.

Note : ma tutrice ESPÉ m'a conseillé de me limiter aux parcours différenciés aux 4\up{e}. Ceci étant couvert par Julia, je n'en parlerai pas. {\color{red}Retour très intéressant de Claire sur le problème du saucissonnage sur la séquence Pythagore, cela vaut sans doute la peine d'en parler en discussion.}\\

{\color{red} Pour ce qui suit, j'ai le sentiment que cela fait beaucoup trop. J'ai entamé un très gros travail sur la symétrie et j'ai également des scans de copies d'élèves appuyant la nécessité de corriger une mauvaise conception pour les angles alternes-internes. La partie sur le calcul algébrique est sans doute dispensable si je n'en ai pas le temps et je manque de productions d'élèves pour l'étayer. Enfin, la dernière partie sera uniquement axée sur le traitement des erreurs à éviter (ou dans le cas présent, des éléments manquants). J'ai à la fois les productions d'élèves scannées, le fichier regroupant les productions les plus intéressantes et la fiche méthodologique construite par les élèves. Les cartes mentales ont surtout été vues avec les 4e, sur le principe elles reprennent ce qu'on a fait avec les fiches méthodologiques. J'ai encore une séquence à finir et une autre à faire avec mes 5e avant de pouvoir faire des cartes mentales intéressantes portant sur plusieurs séquences.}

\subsubsection{Symétries : manipulations manuelles et sur geogebra - fiches de méthodologie}

Le but est de montrer comment j'ai ajouté des couches successives pour renforcer les deux images mentales principales en symétrie axiale et symétrie centrale. J'ai toujours un problème avec mon élève italien qui a tendance à effectuer des translations.

\paragraph{Les cocottes en symétrie}

\paragraph{Remplacement progressif par les mains}

\paragraph{Utilisation de Geogebra en auto-correction}

Lien vers ma séance TICE. Probablement pas aussi développé que le reste.

\paragraph{Fiches de méthodologie}

\paragraph{Retours, productions d'élèves}

Lien vers mon analyse d'évaluation + la toute dernière évaluation faite. Parler de mon élève italien ?

\subsubsection{Angles alternes-internes : comment corriger une potentielle image fausse}

\paragraph{Situation qui a amené les élèves à créer cette image mentale fausse}

Lien vers mini-évaluation + stats rapides sur présence de l'erreur dans les copies.

\paragraph{Utilisation de Geogebra pour corriger une potentielle image mentale fausse}

\paragraph{Une autre approche : la libellule et la coccinelle}

Très rapide, l'évaluation récente montre que cette image mentale n'est pas très populaire au sein de ma classe, alors qu'elle est très utilisée dans les deux autres classes de 5\up{e} (le timing où elle a été présentée est différent).

\paragraph{Initiation à la démonstration : erreurs et correction par les pairs}

{\color{red} Je ne sais pas si celui-ci a sa place dans ce mémoire, dans la mesure où cela se passe intégralement à l'oral}

\paragraph{Cartes mentales - droites parallèles / calcul d'angles}

Les deux sont envisageables et liées à cette séquence. {\color{red}Je ne sais pas si j'aurai le temps de les amorcer avant la date de rendu du mémoire.}

\subsubsection{Calcul algébrique : deux approches différentes pour toucher un maximum d'élèves}

\paragraph{Nombres relatifs - approche vectorielle}

\paragraph{Nécessité de recourir à une deuxième approche}

\paragraph{Nombres relatifs - approche [nom à définir plus tard]}

\subsubsection{Gestion de données : trouver les erreurs à éviter}

\paragraph{Problème initial et productions d'élèves}

\paragraph{Fiche méthodologique résultante}

\subsubsection{Retours élèves}

Cette partie de l'expérimentation n'est pas encore menée. Il s'agit d'un questionnaire à faire remplir par les élèves, en particulier sur l'utilisation des fiches méthodologiques chez eux, sur l'utilisation de GeoGebra/Excel.

% Parle peut-être plutôt d'un tableur, plutôt qu'Excel. Si on peut éviter de mentionner
% des logiciels privateurs quand des alternatives libres sont disponibles, tant mieux !
% Il y a déjà assez de lobbying de la part des GAFAM dans l'EN…
