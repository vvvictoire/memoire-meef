\subsection{Tutorat entre élèves}
\subsubsection{Motivations}
Le tutorat entre élèves peut être une autre source d'apprentissage : pendant que
l'un enseigne, l'autre apprend.

L'enseignement en lui-même est une forme d'apprentissage : la personne qui
enseigne doit assez bien maitriser la notion, les méthodes et le vocabulaire
utilisé afin de pouvoir l'enseigner à quelqu'un d'autre. Savoir enseigner
demande plus que simplement maitriser une technique ; cela demande de savoir
pourquoi on peut utiliser une technique, et quel raisonnement se trouve derrière
la technique.

Permettre et encourager les élèves à enseigner serait donc une autre méthode de
faire apprendre une notion à un élève. C'est aussi une opportunité pour un élève
qui n'aurait pas compris d'avoir un autre regard sur une notion, et un cours
individualisé.

%Est-ce qu'un élève est plus réceptif a un cours individualisé ?
% Demander un retour de Xavière sur la chose
% Voir la réceptivité des élèves a un cours en vidéo

\subsubsection{Risques et contraintes}
Tout comme on n'imaginerait pas un enseignant ne pas maitriser son domaine, il
faudra s'assurer que l'élève maitrise le sien avant de l'enseigner à un autre
élève.

Il y a également un risque de créer ou d'aggraver une scission dans la classe :
les tuteurs pourraient être vus comme des « intellos » ou « fayots ».

% Citer le livre sur la pédagogie coopérative

Il faut également bien maitriser le minutage de sa séance afin de pouvoir controler
les déplacements d'élèves dans la salle de classe

\subsubsection{Expérimentations}

\paragraph{Mise en autonomie sans préparation}

\subsubsection{Pistes de réflexion}

\subsubsection{Aménagement de l'espace classe}

% Classe mutuelle https://profpower.lelivrescolaire.fr/repenser-lamenagement-de-la-classe/
