\subsection{Tutorat entre élèves (Victoire)}
\subsubsection{Motivations}
Le tutorat entre élèves peut être une autre source d'apprentissage : pendant que
l'un enseigne, l'autre apprend.

L'enseignement en lui-même est une forme d'apprentissage : la personne qui
enseigne doit assez bien maitriser la notion, les méthodes et le vocabulaire
utilisé afin de pouvoir l'enseigner à quelqu'un d'autre. Savoir enseigner
demande plus que simplement maitriser une technique ; cela demande de savoir
pourquoi on peut utiliser une technique, et quel raisonnement se trouve derrière
la technique.

Permettre et encourager les élèves à enseigner serait donc une autre méthode de
faire apprendre une notion à un élève. C'est aussi une opportunité pour un élève
qui n'aurait pas compris d'avoir un autre regard sur une notion, et un cours
individualisé. C'est le raisonnement derrière la « méthode Feynman », une méthode
qui applique la méthode d'enseignement de Richard Feynman\footnote{\url{https://www.youtube.com/watch?v=_f-qkGJBPts} vidéo présentant la méthode Feynman.}
.
\subsubsection{Risques et contraintes}

Tout comme on n'imaginerait pas un enseignant ne pas maitriser son domaine, il
faudra s'assurer que l'élève maitrise le sien avant de l'enseigner à un autre
élève.

Il y a également un risque de créer ou d'aggraver une scission dans la classe :
les tuteurs pourraient être vus comme des élèves étant privilégiés, et ainsi créer
une hiérarchie dans la classe.\cite{pedagogie_cooperative_hierarchie}

Il faut également bien maitriser le minutage de sa séance afin de pouvoir controler
les déplacements d'élèves dans la salle de classe : trop de déplacements causent
trop de bruit, et deviennent donc nuisibles à tous.

\subsubsection{Expérimentations}

\paragraph{Mise en autonomie sans préparation}

Une des méthodes de mise en autonomie qui parait le plus simple d'accès est de
simplement les laisser faire et d'autoriser les élèves à s'entraider. J'ai
considéré cette possibilité pour plusieurs raisons :
\begin{itemize}
    \item elle ne demande pas de préparation particulière préalable ;
    \item les règles sont simples pour les élèves ;
    \item elle permet l'apprentissage entre élèves (voir les bénéfices au paragraphe précédent) ;
    \item personne ne m'a dit de ne pas faire ça\footnote{On ne sait jamais, ça aurait pu fonctionner}.
\end{itemize}

Les objectifs de cette expérimentation étaient :
\begin{enumerate}
    \item permettre aux élèves avancés de ne pas « rien faire » ;
    \item permettre aux élèves ayant des difficultés d'avoir une aide plus
    rapidement ;
    \item augmenter l'expertise des élèves avancés, les forçant à adopter d'autres
    façons de voir la notion ;
    \item réduire le nombres d'appels au professeur, restant ainsi disponible
    pour les questions avancées ;
    \item réguler le niveau de bruit\footnote{Ne pas confondre bruit et niveau sonore} de la classe.
\end{enumerate}

Lors d'une séance d'exercices, j'ai donc autorisé les élèves qui voulaient
aider leurs camarades à se lever. Les premiers élèves demandaient à se lever, et
avaient bel et bien fini correctement leurs exercices. Au fur et à mesure, des
élèves avaient pris la liberté de se lever, sans avoir forcément fini leurs exercices,
et sans forcément avoir correctement répondu. Et je ne suis pas persuadée que
tous les élèves s'entraidaient effectivement.

Je considère que cette première expérimentation s'est donc soldée par un échec, car
l'objectif de réguler le niveau de la classe n'était pas atteint : certes certains
élèves sérieux aidaient d'autres élèves qui en avaient besoin, mais d'autres tiraient
profit de la situation pour bavarder, et ainsi perturber d'autres élèves qui
auraient pu réussir les exercices dans des conditions normales.

Pire encore : certains élèves de bonne foi, avaient fini leurs exercices et aidaient
leurs camarades, mais leurs réponses et méthodes étaient fausses, et propagaient
ainsi de mauvaises méthodes au reste de la classe\footnote{Et il est extrêmement difficile de
rattraper le coup après ceci} !

En conclusion, cette méthode ne fonctionne pas (pas de réduction, voire augmentation
du bruit en classe), et est même contre-productive car elle nuit au développement
de certains élèves.

\paragraph{Tétra'aide}

\textit{Voir annexe A}

Le concept du tétra'aide est simple : un tétraèdre permet à l'élève d'indiquer
son besoin d'aide en fonction du sommet orienté vers le haut :
\begin{enumerate}
    \item Tout va bien
    \item J'ai une question non-urgente
    \item À l'aide !
    \item J'aide ou je suis aidé par quelqu'un
\end{enumerate}

Utiliser le tétra'aide permet d'avoir une meilleure visualisation de ceux qui ont
besoin d'aide, et de faire un meilleur triage parmi ceux qui ont besoin d'aide :
ceux qui bloquent vraiment sur l'exercice et ont une question technique, et ceux
qui ont fini ou ont une question qui n'a pas directement rapport au cours.

Objectifs de l'utilisation du tétra'aide :
\begin{enumerate}
    \item pouvoir effectuer un triage des questions urgentes et non-urgentes ;
    \item réduire le bruit dû aux appels au professeur ;
    \item tenter d'intégrer le tutorat entre élèves ;
    \item permettre aux élèves n'osant pas lever la main de pouvoir poser des questions.
\end{enumerate}

J'ai mis le système du tétra'aide en place à la mi-février, en leur fournissant
le patron (qu'ils devaient colorier et assembler à la maison) et en leur expliquant
le fonctionnement. Nous avons mis en œuvre le tétra'aide à la séance suivante,
qui était une séance d'exercices. Les élèves ont assez vite compris le principe,
levant de moins en moins la main au fil de la séance.

Après plusieurs semaines, malgré des explications répétées, certains élèves ont du mal à avoir le réflexe
d'utiliser le tétra'aide et vont plutôt lever la main pour demander mon attention,
parfois pour des questions non-urgentes. Je pensais que ce système allait permettre
aux élèves timides, qui ont du mal à lever la main, de pouvoir profiter de mon aide.
Au final, ceux qui utilisent le tétra'aide sont les même qui osaient lever la main
en début d'année.

Petit à petit, certains élèves ont adopté une attitude de consommateur : si
leur tétra'aide était en position « j'ai besoin d'aide », ils exigeaient une
aide immédiate et sans attendre. Certains élèves jouaient avec leur tétra'aide ou
détruisaient celui d'un autre élève : pour pallier ceci, il faudrait mettre
le patron à disposition sur un espace en ligne auquel les élèves ont l'habitude
d'accéder\footnote{Ce qui est loin d'être le cas : les élèves (en tout cas de mes
classes) ne vont quasiment jamais sur le cartable en ligne, et savent à peine
l'utiliser}.

Une remarque d'ordre pratique : en fonction de la disposition des tables, il
peut être pertinent d'imprimer les patrons dans une taille plus grande : à
l'origine ce système a été fait pour les écoles primaires, beaucoup plus variées
et inventives au niveau des dispositions de tables. Dans une disposition classique
en « rangs d'ognon\footnote{Voir rectifications orthographiques de 1990 en cas de doute
sur l'orthographe} » les tétra'aides sont perdus au milieu des trousses, crayons
et cahiers des élèves. Imprimer en A3 peut être une bonne idée.

\subsubsection{Piste de réflexion : aménagement de l'espace classe}

L'aménagement de l'espace classe peut permettre une meilleure collaboration entre
élèves. La disposition classique (en « autobus » ou « rangs d'ognon ») ne permet
qu'une interaction possible : entre le professeur et les élèves.

D'autres dispositions permettent une meilleure collaboration entre élèves : par
exemple, en ilots bonifiés\cite{ilots_bonifies}, ou en U\cite{amenagement_classe}.

Lors d'un remplacement, j'ai pu tester la méthode en ilots bonifiés qui était mise
en place par la professeur absente. Pour résumer : les tables sont organisés
en ilots permettant d'asseoir 4 élèves (2 tables face à face). Chaque ilot a une
fiche de bonification : à chaque bonne action (les exercices ont été faits, le travail
en séance est fait, participation orale de chaque membre de l'ilot), l'ilot gagne un point.
En cas de mauvaise action (bavardage…) il est possible de retirer un point.

Dès qu'un ilot a accumulé 20 points, les points de tous les ilots sont transformés
en note sur 20 ajoutée à la moyenne\footnote{Avec un coefficient faible en général},
et à la prochaine séance, les configurations des ilots changent.

Je n'ai pas pu tester cette disposition cette année, car je partage ma salle de classe
avec ma tutrice (Mme Sandrine Adam), qui n'était pas partante pour tester ce genre de disposition.
