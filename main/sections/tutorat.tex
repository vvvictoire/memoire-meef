\subsection{Tutorat entre élèves (Victoire)}
\subsubsection{Motivations}
Le tutorat entre élèves peut être une autre source d'apprentissage : pendant que
l'un enseigne, l'autre apprend.

L'enseignement en lui-même est une forme d'apprentissage : la personne qui
enseigne doit assez bien maitriser la notion, les méthodes et le vocabulaire
utilisé afin de pouvoir l'enseigner à quelqu'un d'autre. Savoir enseigner
demande plus que simplement maitriser une technique ; cela demande de savoir
pourquoi on peut utiliser une technique, et quel raisonnement se trouve derrière
la technique.

Permettre et encourager les élèves à enseigner serait donc une autre méthode de
faire apprendre une notion à un élève. C'est aussi une opportunité pour un élève
qui n'aurait pas compris d'avoir un autre regard sur une notion, et un cours
individualisé. C'est le raisonnement derrière la « méthode Feynman », une méthode
qui applique la méthode d'enseignement de Richard Feynman.\footnote{\url{https://www.youtube.com/watch?v=_f-qkGJBPts} vidéo présentant la méthode Feynman.}

%Est-ce qu'un élève est plus réceptif a un cours individualisé ?
% Demander un retour de Xavière sur la chose
% Voir la réceptivité des élèves a un cours en vidéo

\subsubsection{Risques et contraintes}

Tout comme on n'imaginerait pas un enseignant ne pas maitriser son domaine, il
faudra s'assurer que l'élève maitrise le sien avant de l'enseigner à un autre
élève.

Il y a également un risque de créer ou d'aggraver une scission dans la classe :
les tuteurs pourraient être vus comme des élèves étant privilégiés, et ainsi créer
une hiérarchie dans la classe.\cite{pedagogie_cooperative_hierarchie}

Il faut également bien maitriser le minutage de sa séance afin de pouvoir controler
les déplacements d'élèves dans la salle de classe : trop de déplacements causent
trop de bruit, et deviennent donc nuisibles à tous.

\subsubsection{Expérimentations}

\paragraph{Mise en autonomie sans préparation}

Une des méthodes de mise en autonomie qui parait le plus simple d'accès est de
simplement les laisser faire et d'autoriser les élèves à s'entraider. J'ai
considéré cette possibilité pour plusieurs raisons :
\begin{itemize}
    \item elle ne demande pas de préparation particulière préalable ;
    \item les règles sont simples pour les élèves ;
    \item elle permet l'apprentissage entre élèves (voir les bénéfices au paragraphe précédent) ;
    \item personne ne m'a dit de ne pas faire ça.
\end{itemize}

Les objectifs de cette expérimentation étaient :
\begin{enumerate}
    \item permettre aux élèves avancés de ne pas « rien faire » ;
    \item permettre aux élèves ayant des difficultés d'avoir une aide plus
    rapidement ;
    \item augmenter l'expertise des élèves avancés, les forçant à adopter d'autres
    façons de voir la notion ;
    \item réduire le nombres d'appels au professeur, restant ainsi disponible
    pour les questions avancées ;
    \item réguler le niveau de bruit\footnote{Ne pas confondre bruit et niveau sonore} de la classe.
\end{enumerate}

Lors d'une séance d'exercices, j'ai donc autorisé les élèves qui voulaient
aider leurs camarades à se lever. Les premiers élèves demandaient à se lever, et
avaient bel et bien fini correctement leurs exercices. Au fur et à mesure, des
élèves avaient pris la liberté de se lever, sans avoir forcément fini leurs exercices,
et sans forcément avoir correctement répondu. Et je ne suis pas persuadée que
tous les élèves s'entraidaient effectivement.

Je considère que cette première expérimentation s'est donc soldée par un échec, car
l'objectif de réguler le niveau de la classe n'était pas atteint : certes certains
élèves sérieux aidaient d'autres élèves qui en avaient besoin, mais d'autres tiraient
profit de la situation pour bavarder, et ainsi perturber d'autres élèves qui
auraient pu réussir les exercices dans des conditions normales.

Pire encore : certains élèves de bonne foi, avaient fini leurs exercices et aidaient
leurs camarades, mais leurs réponses et méthodes étaient fausses, et propagaient
ainsi de mauvaises méthodes au reste de la classe !

En conclusion, cette méthode ne fonctionne pas (pas de réduction, voire augmentation
du bruit en classe), et est même contre-productive car elle nuit au développement
de certains élèves.

\paragraph{Tétra-aide}

Le concept du tétra-aide est simple : un tétraèdre permet à l'élève d'indiquer
son besoin d'aide en fonction du sommet orienté vers le haut :
\begin{enumerate}
    \item Tout va bien
    \item J'ai une question non-urgente
    \item À l'aide !
    \item J'aide ou je suis aidé par quelqu'un
\end{enumerate}

Utiliser le tétra-aide permet d'avoir une meilleure visualisation de ceux qui ont
besoin d'aide, et de faire un meilleur triage parmi ceux qui ont besoin d'aide :
ceux qui bloquent vraiment sur l'exercice et ont une question technique, et ceux
qui ont fini ou ont une question qui n'a pas directement rapport au cours.

Lent à démarrer

Attention, les élèves détruisent leurs tétra-aides

À faire en plus gros parce qu'ils ont tendence à ne pas se voir

\subsubsection{Pistes de réflexion}

\subsubsection{Aménagement de l'espace classe}

L'aménagement de l'espace classe peut permettre une meilleure collaboration entre
élèves. La disposition classique (en « autobus » ou « rangs d'ognon ») ne permet
qu'une interaction possible : entre le professeur et les élèves.

D'autres dispositions permettent une meilleure collaboration entre élèves : par
exemple, en îlots bonifiés\cite{ilots_bonifies}, ou en U\cite{amenagement_classe}.
