\section{Synthèse de notre réflexion}
Dans cette partie, nous développons la construction de notre problématique à travers les principales questions soulevées par nos lectures. Celles-ci nous ont permis de structurer notre réflexion autour de la notion de différenciation pédagogique :

\begin{itemize}
  \item Que signifie différencier ?
  \item Pourquoi différencier ?
  \item Quelles pratiques pouvons-nous raisonnablement expérimenter lors de l'année de stage ?
  \item Quelles sont les erreurs à ne pas commettre lors de nos expérimentations ?
  \item Comment mesurer l'impact des pratiques expérimentées ?
\end{itemize}

Cette section ne suit pas l'ordre des questions ci-dessus. Il est en effet important pour nous que le lecteur comprenne d'abord pourquoi nous avons choisi de travailler sur des méthodes de différenciation et comment nos lectures nous ont confirmé ce choix, avant de donner une définition précise de ce qu'est la différenciation pédagogique.\\
Après avoir défini pourquoi différencier, nous développons la définition de différenciation pédagogique, ainsi que les pratiques que nous avons envisagées au fil de nos lectures et des entretiens avec nos collègues. Nous abordons ensuite les difficultés que nous avons anticipées pour les pratiques pédagogiques que nous souhaitions mettre en place. Enfin, nous introduisons les premiers questionnements autour de la mesure de l'impact de nos pratiques sur l'autonomie des élèves et l'acquisition des compétences.\\

\subsection{Pourquoi différencier ?}

%\remark{D'après la carte mentale, on devrait retrouver ici la remédiation, rendre les élèves autonomes et faciliter la gestion de classe.}

Lorsque nous avons commencé ce mémoire, nous avions une vision partielle de ce qu'était la différenciation pédagogique. Ce sont nos lectures sur les recherches autour de la notion et des pratiques associées à cette notion qui nous ont confirmé le choix de notre sujet de recherche, et surtout l'objectif de notre travail.
\paragraph{} Une des richesses de notre trinôme est que nous avons chacune un parcours professionnel et des profils de classe différents (dont établissement REP et évaluation par compétences). Cela nous permet d'avoir des échanges intéressants sur ce que nous pouvons rencontrer dans les différents environnements, comment nous abordons les difficultés qui se posent à nous\ldots \\
Malgré nos différences, nous avons toutes eu le même constat à la rentrée : chacune d'entre nous rencontrait des difficultés de gestion de classe lorsqu'elle souhaitait consacrer du temps à des élèves ayant besoin d'une aide dédiée. Nous avons toutes eu le sentiment que les pratiques que nous connaissions (pour les avoir expérimentées en tant qu'élèves) ne suffisaient pas à répondre à notre besoin de fournir un accompagnement personnalisé à nos élèves.\\
Le sujet de différenciation pédagogique proposé par Nadine Grapin dans le cadre du master MEEF de l'UPEC semblait proposer des pistes de réflexion en adéquation avec notre problématique.\\
Sur ses conseils, nous avons lu les documents du Cnesco\footnote{Conseil national d'évalution du système scolaire} produits dans le cadre de la conférence consensus sur la différenciation pédagogique nous ont donné un tout nouvel éclairage sur la notion de différencation pédagogique.
\paragraph{}En particulier, l'article d'Alexia Forget\cite{cnesco_etat_lieux} présente l'état de la recherche autour de la différenciation pédagogique avec une attention particulière à la définition de la notion, mais aussi au « pourquoi » \footnote{Reconnaître la diversité des élèves ? Mettre en valeur cette diversité ? Par devoir ethique ?} et au « pour quoi » \footnote{Quelle finalité ?} différencier.\\
Suite à la lecture de ce document nous avons échangé sur nos situations respectives et sur ce que nous souhaitions pour nos élèves. Il est revenu que nous étions bloquées par l'incapacité de nos élèves à travailler en autonomie, c'est à dire d'effectuer intégralement un travail individuel sans aide et spontanément\footnote{sans forcément réussir le travail dans sa totalité}. Nous avons donc décidé d'orienter notre travail sur l'expérimentation et l'analyse de pratiques pédagogiques de différenciation dans le but de rendre nos élèves plus autonomes, tout en leur proposant un travail adapté à leurs besoins.\\
Nos objectifs finaux se sont dessinés au fil des lectures sur l'autonomie\cite{Meirieu_autonomie}\cite{ilots_bonifies} et de nos échanges. Après un bilan d'étape avec Nadine Grapin, nous nous sommes interrogées sur la place de la différenciation par rapport à l’autonomisation dans notre travail de recherche. Il est apparu que nous n'avons pas toutes les trois le même objectif final :
\begin{itemize}
	\item Victoire a validé ce choix suite à ses propres recherches sur le travail de groupe, qu’elle souhaitait établir dans ses classes de 6\up{e} et 4\up{e}. L’intérêt est notamment pour elle d’explorer la pratique du tutorat par les élèves et les ilots bonifiés.
	\item \remark{A REVOIR} Pour Xavière, qui enseigne en collège REP (5\up{e}, 4\up{e} et 6{e} en AP\footnote{Accompagnement Personnalisé}),  le premier objectif est de rendre les élèves autonomes en classe pour avoir plus de temps à consacrer à l’accompagnement personnalisé des élèves et passer d’une gestion de classe globale systématique à une gestion de classe plus ponctuelle notamment avec les élèves en difficulté. Le second objectif est de donner à tous les élèves les outils d’autoévaluation nécessaires à un travail hors classe autonome. Sa tutrice ESPÉ (Marie-Hélène Le Yaouanq) a validé ce choix, en cohérence avec les observations effectuées lors de visites et avec les contenus de cours déjà proposés.
	\item  Julia cherche à habituer ses élèves à travailler de manière autonome sur des dispositifs de travail différenciés. L'objectif final est de réduire les écarts de niveau dans ses classes de 6\up{e} et 5\up{e}, en limitant son intervention lors des phases d’exercices pour accompagner les élèves nécessitant une assistance personnalisée.
\end{itemize}
  Nous avons cependant un double objectif commun : celui de rendre les élèves autonomes pour pouvoir mieux gérer la classe et pouvoir consacrer plus de temps aux élèves en ayant besoin, et celui de différencier, à savoir fournir aux élèves une manière d’accéder aux compétences qui prend en compte leurs points forts et points faibles.


\subsection{Qu'est-ce que c'est la différenciation ?}
Comme indiqué dans la partie précédente, c'est à la lecture de l'article d'Alexia Forget\cite{cnesco_etat_lieux} que nous avons pris conscience de ce qu'est la différenciation pédagogique. La définition la plus pertinente est, selon nous, celle donnée par Eduscol\cite{Eduscol} : \og La différenciation pédagogique consiste à mettre en \oe{}uvre un ensemble diversifié de moyens
et de procédures d’enseignement et d’apprentissage pour permettre à des élèves d’aptitudes
et de besoins différents d’atteindre par des voies différentes des objectifs communs.\fg{}.\\
Nous avons retrouvé dans ces lectures et dans d'autres ressources\footnote{En particulier, Dominique Lafontaine\cite{cnesco_Lafontaine} traite ainsi de la différenciation dans le système éducatif}, les différents types de différenciation pédagogiques pouvant être mises en place (différenciation successive, différenciation simultanée) ainsi que des exemples de pratiques de différenciation (exercices avec variables didactiques différentes selon les élèves, variation des supports, variation des procédures, organisation spécifique de la classe par groupes, prédagogie inversée\ldots).
\paragraph{}Nous nous sommes aussi rendues compte que nous faisons déjà plus de différenciation que nous le pensions (voir tableau en annexe \ref{tableau_differenciation}) : nous avons tendance à construire le cours avec les élèves, mettre en place des méthodes non expertes avec les élèves, tout en fournissant des méthodes expertes aux élèves qui voudraient les utiliser.\\
Nous n'avons pas axé nos recherches sur un type de différenciation en particulier mais avons recherché des pratiques pédagogiques que nous pourrions expérimenter en classe. De ce point de vue, il existe une documentation riche sur la variété des pratiques ou les modalités de différenciation. Nous citerons en particulier les notes et recommandations des experts ayant contribué à la conférence \og Consensus \fg{} du Cnesco\cite{cnesco_notes_experts}\cite{cnesco_synthese}, les travaux publiés par l'Ifé\footnote{Institut français de l'éducation} \remark{METTRE REF IFE}, les publications de l'inspection académique de Créteil\cite{IPR_math_mouvement}.
\paragraph{} Nous avons ainsi découvert une quantité importante de documents abordant la différenciation pédagogique de manière théorique, soit de manière générale, soit sur un axe spécifique (« Différencier avec les TICE\footnote{Technologies de l'Information et de la Communication pour l'Éducation} », « Différencier sur une évaluation en mathématiques », etc.).\\
Ces lectures nous ont permis de sélectionner une pratique pédagogique adaptée à nos classes et répondant a priori à notre problématique parmi un large choix de dispositifs. Comme nous allons le voir par la suite, l'anticipation de certaines difficultés ou de pièges à éviter a joué sur notre choix final.


\subsection{Quels sont les écueils à éviter ?}
%à caser
En plus de nous donner des pistes de pratiques, nos lectures nous ont également donné des indications sur les erreurs à éviter comme :
\begin{itemize}
	\item se focaliser sur les élèves en difficulté et réduire la différenciation pédagogique à un travail de remédiation\cite{Eduscol}\cite{renc_pedago}
	\item transformer les séances de différenciation en temps d'accompagnements personnalisés\cite{Eduscol}
	\item expérimenter une pratique pédagogique de manière ponctuelle, indépendamment de la séquence travaillée\cite{renc_pedago}\footnote{La pratique de différenciation pédagogique doit au contraire s'inscrire de manière cohérente dans un cycle d'apprentissage des élèves.}
	\item ne porter la différenciation que sur un type de dispositif. Nous avons vu que nous pratiquions déjà plusieurs formes de différenciation dans nos classes, l'objectif est donc de continuer d'enrichir nos pratiques.
	\item différencier pour différencier. Tout dispositif de différenciation doit avoir un objectif clair et atteignable par les élèves\cite{Eduscol}\cite{Meirieu_différenciation}\cite{cnesco_notes_experts}
\end{itemize}
En choisissant les dispositifs de différenciation que nous souhaitions travailler dans le cadre du mémoire, nous avons pris en considération ces avertissements, tout en anticipant d'autres difficultés. Cela nous a fait formuler les questions suivantes :
\begin{itemize}
	\item quelles pratiques peut-on raisonnablement mettre en place pour effectuer une différenciation ? Nous manquons de temps et d’expérience, nous ne pouvons pas différencier chaque exercice ou point de cours ;
	\item sur quelles modalités pouvons nous différencier ? Le plus classique reste les variables didactiques dans un même exercice, un autre que des collègues utilisent déjà est un parcours différencié d’exercices, réactif aux erreurs commises\ldots Il est aussi possible de faire de la différenciation au niveau des supports mis à disposition des élèves, mais là aussi, cela demande beaucoup de temps à mettre en \oe{}uvre.
\end{itemize}

\paragraph{}Suite à nos recherches et aux échanges effectués sur nos pratiques et nos besoins, Victoire a choisi d’axer ses recherches autour des permis de tutorat pour les élèves et des ilots bonifiés. De son côté, Xavière a préféré s’orienter vers une différenciation au service de la création d'outils pour faciliter l'autonomie et l'auto évaluation de ses élèves. Julia a, quant à elle, d’abord opté pour la mise en pratique d’un parcours d’exercices adapté à la réussite des élèves\footnote{Après s'être rapidement rendue compte que le classe inversée était un sujet trop ambitieux pour cette année}, dans le but de les faire travailler en autonomie. Elle a finalement décidé de travailler à une autonomisation des élèves pour un travail différencié plus efficace à travers ce même parcours d’exercice, pour que chaque élève avance à son rythme.
\subsection{Comment évaluer l'impact de la différenciation sur nos classes ?}

Lorsque nous avons voulu mettre en place nos différentes méthodes de différenciation, nous avons eu à faire face à des difficultés de mise en place, dues à notre manque d’expérience en temps que professeurs, et surtout à notre manque d’évaluation de l’efficacité de nos méthodes. Pour le second point, Nadine Grapin nous a recommandé de créer une métrique, par exemple en mettant en place un questionnaire élève afin de voir ce qu’ils en pensent. Bien entendu, il ne faudra pas que ce questionnaire reste notre seule métrique, on pourra aussi prendre en compte les notes, la qualité des exercices faits, la participation en classe\ldots
\paragraph{} La question de « comment évaluer l’impact des méthodes de différenciation mises en place ? » est essentielle car elle nous permet d'ajuster nos pratiques (voir d'abandonner les plus inefficaces d'entre elles). Dans notre cas, nous étudierons l’évolution de l’autonomie des élèves et de l’acquisition des compétences, et nous avons besoin de ces retours pour voir sur quels points améliorer les méthodes de différenciation que nous avons mises en place jusque là, qui ont eu des résultats mitigés \remark{Je ne sais pas qui a mis ça mais il faut le justifier}.

\paragraph{}À ce jour, les exemples trouvés d’évaluations de l’autonomie des élèves sont dédiés au primaire ou à la maternelle\cite{pedagogie_cooperative_hierarchie}, et nous n’avons trouvé aucun retour d’expérience suite à l’utilisation de telles évaluations. Cela nous pose problème dans l’exploitation de l’évaluation de l’autonomie des élèves : « Comment ajuster nos pratiques si les élèves ne sont pas évalués plus autonomes ou si les compétences ne sont pas acquises ? »

\paragraph{}Notre travail a donc à la fois porté sur la construction de pratiques différenciées à expérimenter, mais aussi sur la mise en place d’un mode d’évaluation de l’autonomie et de l’acquisition des compétences par les élèves et l’analyse des évaluations pour ajuster nos pratiques. Nous nous sommes réellement penchées sur la construction d'un questionnaire d'évaluation de l'autonomie des élèves en fin février, une fois que les dispositifs de différenciation pédagogiques avaient déjà été expérimentés.\\
Nous n'avons donc pas trouvé pertinent de proposer un questionnaire des semaines après la fin des expérimentations, d'autant plus qu'il est important pour nous d'avoir une comparaison avant-après expérimentation. Nous avons cependant cherché quels indicateurs nous aurions utilisés pour mesurer l'autonomie des élèves. Ceux-ci nous ont été conseillés par Nadine Grapin ou ont été retrouvés dans des grilles de mesure d'autonomie ds élèves en cycle 2\cite{Grille_autonomie}.\\
Nous avons ainsi noté :
\begin{itemize}
	\item \og Je sais ce que je dois faire après la passation des
	consignes.\fg{}
	\item \og Je sais où trouver les ressources et le matériel dont j'ai besoin pour travailler \fg{}
	\item \og Je suis capable de comprendre si je me suis trompé grâce à la correction \fg{}
	\item \og Je ne pose pas de question sans avoir cherché la réponse avant, ou sans avoir demandé de l'aide à un camarade si c'est autorisé \fg{}
	\item \og On ne me demande jamais de me mettre au travail \fg{}
	\item \og Je saurai travailler à la maison avec ce qui a été fait en classe \fg{}
	\item \og Si je ne comprends pas, je ne baisse pas les bras et je cherche de l'aide (dans les documents d'aide, auprès des autres) \fg{}
	\item \og Je sais expliquer ce qui me bloque dans un exercice \fg{}
\end{itemize}
Comme indiqué dans les parties \og Discussion \fg{} et \og Expérimentations \fg{} qui suivent, ces indicateurs ne sont pas tous pertinents selon les dispositifs expérimentés. Nous avons également trouvé d'autres modes d'évaluation qui seront développés dans la partie \og Discussion \fg{}.
