\section{Synthèse de notre réflexion}
Dans cette partie, nous développons la construction de notre problématique à travers les principales questions soulevées par nos lectures.
Les auteurs y justifient ainsi leurs choix de pratiques en répondant aux questions telles que « \textit{Pourquoi différencier ?} », « \textit{Que signifie différencier ?} », « \textit{Quelles pratiques pouvons-nous raisonnablement expérimenter lors de l'année de stage ?} », « \textit{Comment mesurer l'impact des pratiques expérimentées ?} ».

\subsection{Pourquoi différencier ?}

%\remark{D'après la carte mentale, on devrait retrouver ici la remédiation, rendre les élèves autonomes et faciliter la gestion de classe.}

Suite aux différents entretiens que nous avons pu avoir avec Nadine Grapin et aux retours que nous avons eu suite à notre présentation du 16 janvier, nous avons eu de nouvelles réflexions :
\begin{itemize}
    \item la différenciation peut-elle aider tout le monde ?
    \item quels sont les bénéfices que nous pouvons tirer de la différenciation au niveau gestion de classe ?
    \item quelle est la place de la différenciation par rapport à l’autonomisation ?
\end{itemize}

 Nous avons donc un double objectif : celui de rendre les élèves autonomes pour pouvoir mieux gérer la classe et pouvoir consacrer plus de temps aux élèves en ayant besoin, et celui de différencier, à savoir fournir aux élèves une manière d’accéder aux compétences qui prend en compte leurs points forts et points faibles.

Notre questionnement a évolué avec chaque recherche que nous effectuons. La première question, qui a évolué est « pourquoi différencier ? ». Il y a plusieurs réponses possibles à cette question, chacune dépendante des pratiques et de l’environnement du professeur. Dans notre cas, la réponse a été que nous souhaitons différencier afin de pouvoir rendre les élèves plus autonomes et de faciliter la gestion de classe.

Enfin, la visite de nos tuteurs ESPÉ a été pour nous l’occasion de questionner notre travail autour du mémoire dans une démarche à long terme (certifications, modification de nos pratiques, ambition de carrière). Par exemple, Julia a montré un intérêt pour la question des élèves à besoins particuliers en début d’année. L’entretien a été l’occasion de revenir sur la possibilité de croiser le travail de mémoire avec cet intérêt.

Nos lectures et entretiens nous ont amenées à orienter notre réflexion sur la différenciation pédagogique dans le but de rendre plus autonomes nos élèves. Ce choix est motivé par des besoins différents pour chacune d’entre nous. Pour Xavière, qui enseigne en collège REP\footnote{Réseau d'Éducation Prioritaire} (5\up{e}, 4\up{e} et 6{e} en AP\footnote{Accompagnement Personnalisé}),  le premier objectif est de rendre les élèves autonomes en classe pour avoir plus de temps à consacrer à l’accompagnement personnalisé des élèves et passer d’une gestion de classe globale systématique à une gestion de classe plus ponctuelle notamment avec les élèves en difficulté. Le second objectif est de donner à tous les élèves les outils d’autoévaluation nécessaires à un travail hors classe autonome. Sa tutrice terrain (Marie-Hélène Le Yaouanq) a validé ce choix, en cohérence avec les observations effectuées lors de visites et avec les contenus de cours déjà proposés. Victoire a validé ce choix suite à ses propres recherches sur le travail de groupe, qu’elle souhaitait établir dans ses classes de 6\up{e} et 4\up{e}. L’intérêt est notamment pour elle d’explorer la pratique du tutorat par les élèves et les ilots bonifiés. Julia souhaite de son côté réduire les écarts de niveau dans ses classes de 6\up{e} et 5\up{e}, en limitant son intervention lors des phases d’exercices pour accompagner les élèves nécessitant une assistance personnalisée.

\subsection{Qu'est-ce que c'est la différenciation ?}

\remark{Petits constats en marge de cette sous-problématique, qui pourrait servir à l'introduire.}

Nous nous sommes aussi rendues compte que nous faisons déjà plus de différenciation que nous le pensions (voir tableau en annexe) : nous avons tendance à construire le cours avec les élèves, mettre en place des méthodes non expertes avec les élèves, tout en fournissant des méthodes expertes aux élèves qui voudraient les utiliser.

\remark{D'après la carte mentale, on devrait aborder ici : les deux types de différenciation (avec sources), grande variété de pratiques et de modalités \textit{voir la section Expérimentations} et finalement une définition claire qui nous sert de repère. On peut aussi aborder le problème de la grande richesse des documents proposés sur ce thème.}

%\remark{On découpe en sub-subsection pour la variété des pratiques et des modalités ?}

\subsubsection{Parcours différenciés}

L'entretien du 16 janvier a permis à Julia de mieux formuler sa problématique qui s’avère être de trouver une pratique (le parcours différencié) s’appuyant sur l’autonomie des élèves pour mieux différencier. La présentation orale du parcours différencié a par ailleurs permis de mettre en avant des points de réflexion supplémentaires : quels exercices corriger ? Quels exercices sont obligatoires? Comment évaluer la progression des élèves et pour quoi ? Que faire une fois l’évaluation effectuée ?

\subsubsection{Lectures}

Nous avons tout d’abord ciblé nos lectures sur certains documents généraux (voir bibliographie) faisant l’état des lieux de la différenciation pédagogique. Nous avons en effet découvert une quantité importante de documents abordant le sujet de manière théorique, soit de manière générale, soit sur un axe spécifique (« Différencier avec les TICE\footnote{Technologies de l'Information et de la Communication pour l'Éducation} », « Différencier sur une évaluation en mathématiques », etc.). La quantité de ressources disponibles étant trop importante, nous avons choisi de nous concentrer sur la documentation suggérée par Nadine, la page Eduscol sur la différenciation pédagogique, les documents du Conseil national d’évaluation du système scolaire (Cnesco) et le dossier de veille de l’Institut Français de l’Éducation (IFÉ).

Ces lectures ont amélioré notre compréhension de la différenciation pédagogique : amener tous les élèves à un niveau de connaissance et de compétence commun en tenant compte de leurs différences par des pratiques d’enseignement adaptées à chacun, pour que chaque élève avance à son rythme. Elles nous ont également permis de comprendre que, malgré l'existence d'un consensus sur ce que représente la différenciation pédagogique, elle peut être mise en place par des pratiques très variées (supports différenciés, mise en place de tutorat entre élèves, classe inversée, parcours différenciés, groupes de niveau, différenciation des productions attendues, etc.) pour des objectifs finaux différents (lutter contre le décrochage scolaire, assurer une égalité des chances scolaires, amener chaque élève au maximum de son potentiel, verrouiller l’acquisition d’un niveau de compétence et de connaissance minimal, faire gagner en autonomie les élèves, responsabiliser les élèves face à leur apprentissage, etc.)

Il nous reste des lectures à terminer parmi les documents que nous avions initialement identifiés pour la construction de notre réflexion, en particulier sur les pratiques de différenciation pédagogique que nous souhaitons développer (voir bibliographie).

\subsubsection{Visites des tuteurs ESPÉ}

Les visites des tuteurs ESPÉ et les travaux effectués en UE2\footnote{Cours de didactique} et UE4\footnote{Accompagnement du stage} nous ont également permis de faire un état des lieux de nos pratiques en classe. Nous avons constaté que nous ne faisions quasiment pas de différenciation en classe, hors différenciation successive avec des exercices supplémentaires pour les élèves les plus à l’aise, ou aide individuelle pouvant prendre la forme d’une piste donnée suite à une question posée.
En nous entretenant avec nos collègues, nous avons constaté que peu d’entre eux avaient une pratique de différenciation continue : certains ne cherchent pas spécialement à pratiquer une différenciation pédagogique en classe alors que d’autres adaptent leurs pratiques sur des séquences spécifiques où ils estiment nécessaire de différencier les variables didactiques ou le mode de travail des élèves.

Dès que nous avons choisi d’orienter nos recherches autour de l’autonomisation des élèves par la différenciation pédagogique, nous avons sélectionné dans nos lectures des pratiques que nous souhaiterions expérimenter en classe.

Ainsi Victoire a choisi d’effectuer des recherches autour des permis de tutorat pour les élèves et des ilots bonifiés. De son côté, Xavière a préféré s’orienter vers la différenciation par la mise en place de parcours différenciés (en 4\up{e}) et d’images mentales variées (dans les trois classes), pour faciliter l’auto-évaluation des compétences. Julia a, quant à elle, d’abord opté pour la mise en pratique d’un parcours d’exercices adapté à la réussite des élèves, dans le but de les faire travailler en autonomie. Elle a finalement décidé de travailler à une autonomisation des élèves pour un travail différencié plus efficace à travers ce même parcours d’exercice, pour que chaque élève avance à son rythme.

\subsubsection{Quelles méthodes n'ont pas été explorées ?}

Nous avons toutes les trois choisi de ne pas explorer en profondeur les pratiques comme la classe inversée, la différenciation dans l’enseignement par les TICE, la différenciation dans l’évaluation, etc. Ces pratiques pourront cependant faire l’objet d’une mise en pratique occasionnelle en classe, elles ne sont cependant pas l’objet principal de notre mémoire. Elles pourront faire l’objet d’une analyse annexe.

Afin d’enrichir notre analyse de pratique, Victoire voulait effectuer une visite dans une école primaire afin de pouvoir y analyser les pratiques mises en place pour l’autonomie des élèves, les élèves d’école primaire ayant tendance à être autonomes et à perdre cette autonomie au collège. De même, Julia voulait assister au cours d’une collègue de français qui pratique la classe inversée. L’objectif aurait été d’observer le mode de fonctionnement de la classe en autonomie, le format d’évaluation des compétences, ainsi que les modes de différenciation mis en place. Xavière a assisté et assistera à nouveau à des cours d’EPS, pour étudier le format des consignes, la remédiation par la différenciation et leurs impacts sur l’autonomie des élèves.

\subsection{Quels sont les écueils à éviter ?}

%\remark{D'après la carte mentale, on devrait aborder ici le temps (de préparation), la difficulté à choisir ce qu'on différencie et les problèmes de mise en place (en particulier une discussion sur le travail de groupe. Tout ceci est à garder assez concis, puisque ce sera repris dans la section Discussion.}

La seconde question qui vient est « comment différencier ? ». Cette question a évolué en plusieurs questions qui seront la trame principale de nos réflexions :
\begin{itemize}
    \item quelles pratiques peut-on raisonnablement mettre en place pour effectuer une différenciation ? Nous manquons de temps et d’expérience, nous ne pouvons pas différencier chaque exercice ou point de cours. Nous avons besoin de méthodes pédagogiques ;
    \item sur quelles modalités pouvons nous différencier ? Le plus classique reste les variables didactiques dans un même exercice, un autre que des collègues utilisent déjà est un parcours différencié d’exercices, réactif aux erreurs commises… Il est aussi possible de faire de la différenciation au niveau des supports mis à disposition des élèves, mais là aussi, cela demande beaucoup de temps à mettre en œuvre ;
\end{itemize}

\remark{J'ai dû passer à coté de quelque chose, j'ai du mal à retrouver les problématiques de temps. À voir, me concernant je sais que j'ai eu des remarques pour laisser tomber les parcours différenciés par exemple.}

\subsection{Comment évaluer l'impact de la différenciation sur nos classes ?}

\remark{Une section qui parle de la difficulté de se lancer dans la différenciation pour les nouveaux professeurs. Et finalement, c'est un gros morceau de notre mémoire...}

Lorsque nous avons voulu mettre en place nos différentes méthodes de différenciation, nous avons eu à faire face à des difficultés de mise en place, dues à notre manque d’expérience en temps que professeurs, et surtout à notre manque d’évaluation de l’efficacité de nos méthodes. Pour palier le premier point, nous allons faire appel à différentes personnes avec qui nous avons pris contact : Victoire souhaitait visiter une classe dans le premier degré, Julia voulait visiter un de ses collègues qui pratique une inversion de classe, et Xavière et Victoire peuvent facilement prévoir des visites croisées. Pour le second point, Nadine Grapin nous a recommandé de créer une métrique, par exemple en mettant en place un questionnaire élève afin de voir ce qu’ils en pensent. Bien entendu, il ne faudra pas que ce questionnaire reste notre seule métrique, on pourra aussi prendre en compte les notes, la qualité des exercices faits, la participation en classe…

Enfin, une question qui a beaucoup d’importance dans le cadre de notre réflexion est « comment évaluer l’impact des méthodes de différenciation mises en place ? » Ce type de retour est essentiel à la mise en place de stratégies de différenciation afin de savoir si elles sont efficaces, inefficaces voire nocives. Dans notre cas, nous étudierons l’évolution de l’autonomie des élèves et de l’acquisition des compétences, et nous avons besoin de ces retours pour voir sur quels points améliorer les méthodes de différenciation que nous avons mises en place jusque là, qui ont eu des résultats mitigés.

Pour l’instant, les exemples trouvés d’évaluations de l’autonomie des élèves sont dédiés au primaire ou à la maternelle, et nous n’avons aucun retour d’expérience suite à l’utilisation de telles évaluations. Cela nous pose problème dans l’exploitation de l’évaluation de l’autonomie des élèves : « Comment ajuster nos pratiques si les élèves ne sont pas évalués plus autonomes ou si les compétences ne sont pas acquises ? »

La suite de notre travail va donc à la fois se tourner vers la construction de pratiques différenciées à expérimenter, la mise en place d’un mode d’évaluation de l’autonomie et de l’acquisition des compétences par les élèves et l’analyse des évaluations pour ajuster nos pratiques.

Nous souhaitions effectuer des recherches sur l’évaluation de l’impact de nos pratiques sur l’autonomie et l’acquisition des compétences, mais manquions de ressources à ce sujet. Enrichir notre bibliographie sur ce point a donc été notre principal objectif à court terme dans ce domaine.
