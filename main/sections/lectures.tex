\section{synthèse}
La synthèse doit comprendre une partie de définition et une justification de la problématique, qui s'appuie sur une synthèse argumentée de ce qui vous y a conduit: vos lectures d'une part, et d'autre part certaines de vos observations ou expérimentations.
Elle décrit et justifie également le corpus permettant de répondre à la problématique ou de la faire évoluer: entretiens, analyses de documents (manuels, copies, …), observations, construction de séances / séquences, expérimentations en classe, notes de lecture.\\

\textbf{25 à 30 pages}

\section{Bilan des lectures}


\textit{<- Début du bilan d'étape ->}

tat des lieux des travaux menés depuis Novembre 
\subsection{Recherches documentaires et premières observations}

Nous avons tout d’abord ciblé nos lectures sur certains documents généraux (voir bibliographie en Annexe) faisant l’état des lieux de la différenciation pédagogique. Nous avons en effet découvert une quantité importante de documents abordant le sujet de manière théorique, soit de manière générale, soit sur un axe spécifique («Différencier avec les TIC», «Différencier sur une évaluation en mathématiques», etc.). La quantité de ressources disponibles étant trop importante, nous avons choisi de nous concentrer sur la documentation suggérée par Nadine, la page Eduscol sur la différenciation pédagogique, les documents du Conseil national d’évaluation du système scolaire (Cnesco) et le dossier de veille de l’Institut Français de l’Éducation (IFÉ).

Ces lectures ont amélioré notre compréhension de la différenciation pédagogique : amener tous les élèves à un niveau de connaissance et de compétence commun en tenant compte de leurs différences par des pratiques d’enseignement adaptées à chacun, pour que chaque élève avance à son rythme. Elles nous ont également permis de comprendre que, malgré qu’il existe un consensus sur ce que représente la différenciation pédagogique, elle peut être mise en place par des pratiques très variées (supports différenciés, mise en place de tutorat entre élèves, classe inversée, parcours différenciés, groupes de niveau, différenciation des productions attendues, etc.) pour des objectifs finaux différents (lutter contre le décrochage scolaire, assurer une égalité des chances scolaires, amener chaque élève au maximum de son potentiel, verrouiller l’acquisition d’un niveau de compétence et de connaissance minimal, faire gagner en autonomie les élèves, responsabiliser les élèves face à leur apprentissage, etc.) 

Les entretiens que nous avons effectués avec nos tuteurs terrains, notre directrice de mémoire et nos échanges avec nos collègues (dont les autres élèves stagiaires) nous ont permis de mûrir nos lectures, d’une part, et de spécifier notre axe de recherche, d’autre part.

Les visites des tuteurs ESPÉ et les travaux effectués en UE2 et UE4 nous ont également permis de faire un état des lieux de nos pratiques en classe. Nous avons constaté que nous ne faisions quasiment pas de différenciation en classe, hors différenciation successive avec des exercices supplémentaires pour les élèves les plus à l’aise, ou aide individuelle pouvant prendre la forme d’une piste donnée suite à une question posée1. 
En nous entretenant avec nos collègues, nous avons constaté que peu d’entre eux avaient une pratique de différenciation continue : certains ne cherchent pas spécialement à pratiquer une différenciation pédagogique en classe alors que d’autres adaptent leurs pratiques sur des séquences spécifiques où ils estiment nécessaire de différencier les variables didactiques ou le mode de travail des élèves.

Enfin, la visite de nos tuteurs ESPÉ a été pour nous l’occasion de questionner notre travail autour du mémoire dans une démarche à long terme (certifications, modification de nos pratiques, ambition de carrière). Par exemple, Julia a montré un intérêt pour la question des élèves à besoins particuliers en début d’année. L’entretien a été l’occasion de revenir sur la possibilité de croiser le travail de mémoire avec cet intérêt. 

\subsection{Impact des travaux sur notre réflexion}

Nos lectures et entretiens nous ont amenées à orienter notre réflexion sur la différenciation pédagogique dans le but de rendre plus autonomes nos élèves. Ce choix est motivé par des besoins différents pour chacune d’entre nous. Pour Xavière, qui enseigne en collège REP (5e, 4e et 6e AP),  le premier objectif est de rendre les élèves autonomes en classe pour avoir plus de temps à consacrer à l’accompagnement personnalisé des élèves et passer d’une gestion de classe globale systématique à une gestion de classe plus ponctuelle notamment avec les élèves en difficulté. Le second objectif est de donner à tous les élèves les outils d’autoévaluation nécessaires à un travail hors classe autonome. Sa tutrice terrain (Marie-Hélène Le Yaouanq) a validé ce choix, en cohérence avec les observations effectuées lors de visites et avec les contenus de cours déjà proposés. Victoire a validé ce choix suite à ses propres recherches sur le travail de groupe, qu’elle souhaitait établir dans ses classes de 6e et 4e. L’intérêt est notamment pour elle d’explorer la pratique du tutorat par les élèves et les îlots bonifiés. Julia souhaite de son côté réduire les écarts de niveau dans ses classes de 6e et 5e, en limitant son intervention lors des phases d’exercices pour accompagner les élèves nécessitant une assistance personnalisée.

Dès que nous avons choisi d’orienter nos recherches autour de l’autonomisation des élèves par la différenciation pédagogique, nous avons sélectionné dans nos lectures des pratiques que nous souhaiterions expérimenter en classe. 

Ainsi Victoire a choisi d’effectuer des recherches autour des permis de tutorat pour les élèves et des îlots bonifiés. De son côté, Xavière a préféré s’orienter vers la différenciation par la mise en place de parcours différenciés (en 4e) et d’images mentales variées (dans les trois classes), pour faciliter l’auto-évaluation des compétences. Julia a, quant à elle, d’abord opté pour la mise en pratique d’un parcours d’exercices adapté à la réussite des élèves, dans le but de les faire travailler en autonomie. Elle a finalement décidé de travailler à une autonomisation des élèves pour un travail différencié plus efficace à travers ce même parcours d’exercice, pour que chaque élève avance à son rythme. 
Nous avons toutes les trois choisi de ne pas explorer en profondeur les pratiques comme la classe inversée, la différenciation dans l’enseignement par les TICE, la différenciation dans l’évaluation, etc. Ces pratiques pourront cependant faire l’objet d’une mise en pratique occasionnelle en classe, elles ne sont cependant pas l’objet principal de notre mémoire. Elles pourront faire l’objet d’une analyse annexe.

Lorsque nous avons voulu mettre en place nos différentes méthodes de différenciation, nous avons eu à faire face à des difficultés de mise en place, dues à notre manque d’expérience en temps que professeurs, et surtout à notre manque d’évaluation de l’efficacité de nos méthodes. Pour palier le premier point, nous allons faire appel à différentes personnes avec qui nous avons pris contact : Victoire souhaite visiter une classe dans le premier degré, Julia va visiter un de ses collègues qui pratique une inversion de classe, et Xavière et Victoire peuvent facilement prévoir des visites croisées. Pour le second point, Nadine Grapin nous a recommandé de créer une métrique, par exemple en mettant en place un questionnaire élève afin de voir ce qu’ils en pensent. Bien entendu, il ne faudra pas que ce questionnaire reste notre seule métrique, on pourra aussi prendre en compte les notes, la qualité des exercices faits, la participation en classe…

\subsection{Évolution de notre réflexion}

Grâce aux échanges et aux recherches, nous avons pu découvrir de nouvelles méthodes et pistes de réflexion quant à la différenciation. Ces nouveaux éléments nous permettent de mettre en place des expérimentations, qui nous mèneront, à leur tour, à de nouvelles questions et autres constats.
Notre questionnement a évolué avec chaque recherche que nous effectuons. La première question, qui a évolué est « pourquoi différencier ?». Il y a plusieurs réponses possibles à cette question, chacune dépendante des pratiques et de l’environnement du professeur. Dans notre cas, la réponse a été que nous souhaitons différencier afin de pouvoir rendre les élèves plus autonomes et de faciliter la gestion de classe.

La seconde question qui vient est «comment différencier ?». Cette question a évolué en plusieurs questions qui seront la trame principale de nos réflexions :
\begin{itemize}
    \item quelles pratiques peut-on raisonnablement mettre en place pour effectuer une différenciation ? Nous manquons de temps et d’expérience, nous ne pouvons pas différencier chaque exercice ou point de cours. Nous avons besoin de méthodes pédagogiques ;
    \item sur quelles modalités pouvons nous différencier ? Le plus classique reste les variables didactiques dans un même exercice, un autre que des collègues utilisent déjà est un parcours différencié d’exercices, réactif aux erreurs commises… Il est aussi possible de faire de la différenciation au niveau des supports mis à disposition des élèves, mais là aussi, cela demande beaucoup de temps à mettre en œuvre.
\end{itemize}

Enfin, une question qui a beaucoup d’importance dans le cadre de notre réflexion est «comment évaluer l’impact des méthodes de différenciation mises en place?» Ce type de retour est essentiel à la mise en place de stratégies de différenciation afin de savoir si elles sont efficaces, inefficaces voire nocives. Dans notre cas, nous étudierons l’évolution de l’autonomie des élèves et de l’acquisition des compétences, et nous avons besoin de ces retours pour voir sur quels points améliorer les méthodes de différenciation que nous avons mises en place jusque là, qui ont eu des résultats mitigés. 

Pour l’instant, les exemples trouvés d’évaluations de l’autonomie des élèves sont dédiés au primaire ou à la maternelle, et nous n’avons aucun retour d’expérience suite à l’utilisation de telles évaluations. Cela nous pose problème dans l’exploitation de l’évaluation de l’autonomie des élèves : « Comment ajuster nos pratiques si les élèves ne sont pas évalués plus autonomes ou si les compétences ne sont pas acquises ? »

La suite de notre travail va donc à la fois se tourner vers la construction de pratiques différenciées à expérimenter, la mise en place d’un mode d’évaluation de l’autonomie et de l’acquisition des compétences par les élèves et l’analyse des évaluations pour ajuster nos pratiques.

\subsection{Recherches documentaires et collaborations extérieures supplémentaires}

Il nous reste des lectures à terminer parmi les documents que nous avions initialement identifiés pour la construction de notre réflexion, en particulier sur les pratiques de différenciation pédagogique que nous souhaitons développer (voir bibliographie). 

De plus, nous souhaitons effectuer des recherches sur l’évaluation de l’impact de nos pratiques sur l’autonomie et l’acquisition des compétences, mais manquons de ressources à ce sujet. Enrichir notre bibliographie sur ce point est donc notre principal objectif à court terme dans ce domaine.
Afin d’enrichir notre analyse de pratique, Victoire effectuera une visite dans une école primaire afin de pouvoir y analyser les pratiques mises en place pour l’autonomie des élèves, les élèves d’école primaire ayant tendance à être autonomes et à perdre cette autonomie au collège.  La visite reste à programmer, Victoire est en attente d’un retour de ses collègues du primaire. De même, Julia assistera au cours d’une collègue de français qui pratique la classe inversée. L’objectif sera d’observer le mode de fonctionnement de la classe en autonomie, le format d’évaluation des compétences, ainsi que les modes de différenciation mis en place. Xavière a assisté et assistera à nouveau à des cours d’EPS, pour étudier le format des consignes, la remédiation par la différenciation et leurs impacts sur l’autonomie des élèves.

\textit{<- Fin du bilan d'étape ->}