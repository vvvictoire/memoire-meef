\subsection{Évaluations différenciées (Victoire)}
\subsubsection{Motivations}

L'idée de différencier les évaluations m'est venue au contact de Julia et Xavière,
qui évaluent toutes les deux par compétences. Mon collège est encore resté aux
notes, et je voulais faire en sorte de pouvoir évaluer mes élèves sur des
compétences, et donc leur proposer plusieurs manières de prouver leur maitrise
d'une compétence. Différencier les évaluations est donc un choix naturel.

\subsubsection{Comment je l'ai découvert}
Avec Sandrine, nous étions toutes les deux
intéressées par les évaluations différenciées. Au départ, nous voulions tester
ceci sous sa forme la plus simple : pour un exercice, les élèves pourraient
choisir une version simple, qui rapporterait $n$ points, ou\footnote{Ici, un ou
exclusif} une version plus experte, qui rapporterait $m$ points\footnote{$n < m$}.

Alors que nous allions passer à la rédaction de nos sujets, Julia me fournit
un article traitant des évaluations différenciées\cite{differenciation_devoir_surveille}.
En lisant cet article, je me rends compte que la manière dont Sandrine et moi
allions justement procéder d'une manière qui n'est pas recommandée par les
auteurs\footnote{Cette fois-ci, il y a vraiment quelqu'un pour nous dire de ne
\textbf{pas} faire ce que nous avions prévu.} !

\subsubsection{Expérimentation}

Les auteurs recommandent, plutôt que de proposer aux élèves le choix entre 2
versions d'un exercice, de proposer un « buffet d'exercices » : par exemple, proposer
12 exercices à 2 points chacun\footnote{Les valeurs en points de chaque exercice
peut varier bien sûr}, et laisser aux élèves choisir quels exercices ils veulent
effectuer. Dans cet exemple, il est possible d'obtenir jusqu'à 24 points, et
donc la note de 24/20 !

J'ai donc rédigé un sujet\footnote{Voir annexe \ref{sujet_differencie}}, avec les objectifs suivants
en tête :

\begin{itemize}
    \item permettre aux élèves de prouver leur maitrise d'une notion par différentes manières ;
    \item faire en sorte qu'un élève \textbf{doive} au moins couvrir chaque notion une fois ;
    \item donner du challenge aux élèves en avance ;
    \item redonner confiance aux élèves en difficulté mais qui travaillent.
\end{itemize}

Au final, lors et après l'évaluation, j'ai pu remarquer les choses suivantes :
\begin{itemize}
    \item il y a peu d'effets sur les élèves non-travailleurs et ceux qui sont
    en avance. Par contre, il y a eu une petite amélirotation chez les élèves
    « entre-deux », mais travailleurs ;
    \item l'évaluation était trop difficile, d'où une difficulté d'observer une
    amélioration chez la population ciblée ;
    \item tous les élèves ont travaillé jusqu'au bout.
\end{itemize}

Après l'évaluation, les exercices de base ont été corrigés, et un corrigé complet
a été mis à disposition sur le cartable en ligne.

Le point essenciel à corriger dans mon expérimentation est le sujet de l'évaluation :
j'ai énormément de mal à rédiger des sujets d'évaluation longs\footnote{
Notamment parce que ça prend énormément de temps à préparer, encore plus à corriger,
et que ça demandait une énergie que je n'avais pas cette année.
}.

\subsubsection{Pistes de réflexion}

Pour de prochaines évaluations différenciées, voici ce que je changerai :
\begin{itemize}
    \item rédiger des sujets moins longs, sur moins de notions ;
    \item essayer d'appliquer ce système à des interrogations courtes de début d'heure ;
    \item voir si, dans un prochain établissement, ce système est requis si l'établissement
    utilise une évaluation par compétences.
\end{itemize}

Il faudrait également que je me renseigne sur le principe des ceintures de compétences,
qui donnent un peu plus de liberté dans le rythme d'apprentissage des élèves.
