\section{Discussions}
Dans cette partie, nous faisons le point sur nos pratiques avec un regard critique enrichi par les retours de nos collègues et de nos élèves, et confrontés aux expérimentations d'autres collègues de l'ESPE. Nous aborderons également la question de l'impact des pratiques expérimentées sur l'acquisition des connaissances et des compétences de nos élèves, et sur leur capacité à travailler en autonomie. Nous traiterons enfin des autres dispositifs mis en place lors de notre année de stage et dont l'analyse (brève) peut, selon nous, enrichir ce mémoire.
\subsection{Analyse croisée de nos pratiques}	
Comme indiqué dans la partie \ref{Expérimentations}-Expérimentations, nous avons effectué des choix d'expérimentation différents au regard de nos besoins et de nos recherches littéraires.\\
\remark{Je trouve pas vraiment de titre, je pense que dire ce que nous avons en commun dans nos pratiques ou non et pourquoi nous avons fait ces choix est le plus judicieux}
\subsubsection{Comparaison de nos expérimentations}
\paragraph*{}Nous avons toutes choisies de mettre en place des dispositifs incluant tous élèves de nos classes. Ces dispositifs ont des impacts plus ou moins importants sur les élèves selon leur niveau, mais nous avons choisi de tous les inclure dans nos dispositifs.\\
Dans les faits, le tutorat des élèves, les exercices de remédiation ou les défis des parcours différenciés vont avoir une incidence plus importante sur les élèves en grande difficulté ou au contraire très à l'aise sur les compétences travaillées. Les évaluations différenciées et les supports adaptés à chaque élève sont en revanche supposées avoir un impact sur l'autonomie et l'acquisition des compétences de tous les élèves, sans distinction de niveau.\\
\remark{Je ne sais pas si ce que j'écris vous paraît pertinent. Je pense que c'est une première différence intéressante entre les dispositifs. Pour la suite, @Xavière, je te laisse modifier à ta guise!}
\paragraph*{}Dans la construction des parcours différenciés, Xavière et Julia ont toutes les deux mis en place des parcours de formes différentes. Alors que Julia propose un guide de parcours ainsi qu'une série d'exercices selon la réussite ou non de l'élève, Xavière propose une feuille d'exercice avec une version plus ou moins difficile de l'exercice (variation dans le choix des variables didactiques) et l'élève choisit celui qu'il traite \remark{il me semble, de mémoire}. Dans les deux cas, un élève en difficulté aura la possibilité de traiter un exercice plus facile pour appréhender une notion, et les élèves les plus à l'aise auront accès à des problèmes plus difficiles à aborder.\\
La différence de présentation présente cependant une différence d'approche très intéressante car dans l'expérimentation de Julia, les élèves n'ont pas le choix de traiter la version facile d'un exercice ou le défi de leur parcours. Ils doivent également identifier par eux-même une situation d'échec ou de fragilité dans le traitement d'un exercice, qui demande d'effectuer un exercice supplémentaire. Au contraire, l'approche de Xavière donne le choix aux élèves dans le niveau de l'exercice à traiter. D'un côté on peut s'attendre à ce que les élèves à l'aise essaient toutes les versions ou que les élèves ne choisissent que la facilité, d'un autre côté, un élève en difficulté ne passera pas par une étape "d'échec" avant de traiter un exercice plus abordable.
\paragraph*{}
\remark{Vous avez toutes les deux testé des évaluations différenciées mais pas du tout sous la même forme (je crois que Xavière c'était plus dans le sens de s'auto évaluer), si ça prend du sens, vous pourriez peut-être en parler ici?}\\
\remark{Pour la suite j'ai déplacé la partie "notre retour" après les retours des autres. Comme ça on peut s'appuyer sur cette partie pour alimenter notre propre retour}
\subsubsection{Retours de nos collègues}
\paragraph*{}
\remark{Retours de nos tutrices sur nos expérimentations et analyse de travaux de collègues de l’ESPE. J'ai listé les interlocuteurs mais on peut également les citer chacun leur tour et dire ce qu'ils nous ont apporté à chaque fois par exemple.}
Dans le cadre du mémoire, nous avons pu échanger sur nos expérimentations avec Nadine Grapin, notre responsable de suivi. Elle nous a en particulier aidées à identifier les modalités de différenciation ou de mise en autonomie dans nos pratiques. Elle nous a également indiqué quels collègues de la formation avaient potentiellement des approches communes ou très différentes des notres. Nous avons ainsi eu l'occasion d'échanger avec le groupe de Fanny Mauhé, Morgane Petigat et Léa Serrano\cite{memoire_fanny} en particulier lors des phases de documentation, ainsi qu'avec le groupe formé par Cédric Hamon et Juliette Kirouane\cite{memoire_eval_differenciee}, en particulier lors de la journée de valorisation des mémoires.\\
Nous avons enfin présenté nos pratiques à nos tutrices terrain qui ont pu assister à leurs mises en application en classe et nous faire part de leurs observations.
\paragraph*{}
Les différents échanges avec nos interlocuteurs a mis en avant des paramètres à prendre en compte et que nous n'avions pas toujours anticipé dans la préparation de nos expérimentations. La forme donnée aux supports est, par exemple, ressortie de la part de tous nos interlocuteurs (avec plus ou moins d'importance). Ainsi, la tutrice de Julia a indiqué que certains parcours présentés aux élèves de 6\up{e} avaient une forme inadaptée, voir décourageante pour les élèves, alors qu'elle serait adaptée à une classe de 4\up{e} (après quelques ajustements). Dans leur travail sur la différenciation, le groupe de Fanny, Morgane et Léa a également observé que la forme des supports de travail ou de différenciation avait une grande importance sur la mise au travail des élèves \remark{si j'ai bien compris?}\\
De même l'identification des différents temps de classe est apparue encore plus importante que lors de la préparation d'une séance "normale" pour la plupart des pratiques analysées. En effet, la mise en place d'îlots bonifiés, de tétra-aide ou de parcours de différenciation demande de connaître par avance les temps de travail individuel ou collectif, les période de déplacement des élèves (et leurs buts), les durées approximatives de chaque période, le possibles sources d'agitation ou de sollicitations importantes\ldots Il nous a notamment été demandé lors des analyses de séance d'identifier les différents moments de nos séances, leur objectif et le déroulement anticipé puis observé.\\
Ces temps de classes vont être variables selon le travail effectué mais aussi en fonction des élèves. Par exemple, une classe de 6\up{e} va généralement demander plus d'explications sur ce qui doit être fait qu'une classe de 4\up{e}. \remark{@victoire : Les élèves de 4\up{e} sont ils plus timides dans l'entraide ? Mes collègues m'ont dit que le tutorat après la 4e ça marchait qu'entre copains et encore! Question de puberté}. De plus, la maturité des élèves aura un impact sur sa capacité à effectuer le travail demandé en autonomie. Ainsi, des élèves de 6\up{e} auront plus de difficulté à identifier quelle série d'exercices ils seront capables de faire dans le temps imparti pour optimiser leur note. Pour les mêmes raisons, ces élèves ne seront pas forcément aptes à s'auto-évaluer dans des exercices de raisonnement de parcours différenciés (pour des exercices de géométrie par exemple). Nous devons donc également adapter nos pratiques au niveau d'autonomie de nos élèves sur les tâches demandées et leur apprendre à les effectuer pour la suite de leur scolarité (voir notamment la réflexion de Philippe Meirieu\cite{Meirieu_autonomie} sur le sujet).\\
\remark{Différences/similitudes avec travaux de groupe sur éval différenciée}
Les visites de nos tutrices en séance nous ont permis d'identifier les différentes variables didactiques sur lesquelles nous pouvons différencier dans nos dispositifs. Ainsi, l'analyse d'exercices effectués en classe a mis en avant l'importance de la question des objectifs à atteindre par les élèves dans la résolution de ceux-ci, mais aussi des objectifs leur correction. Cette dernière question est d'autant plus importante dans la construction des parcours différenciés et des évaluations différenciées où les élèves ne traitent pas les mêmes exercices et où la correction n'est pas toujours effectuée en classe entière.\\
Malgré les nombreux échanges entre nous ou avec nos collègues, il a été rapidement clair qu'un dispositif qui fonctionne lors d'une séance ne fonctionnera pas forcément avec une autre classe ou même à un autre horaire. Nous avons tenté d'identifier les paramètres de nos expérimentations qui permettent de rendre nos dispositifs ré-exploitables dans le futur (forme des supports, organisation de la salle, découpage de la séance, choix des exercices\ldots).

\subsubsection{Nos retours}
Dans cette partie, chacune d'entre nous effectue un retour de sa propre expérimentation et commentera les dispositifs mis en place par les deux autres. \remark{si ça vous va, sinon on organise ça par dispositif ou autre :)}\\
\paragraph*{Retour de Julia :} 
Le travail de recherche et d'expérimentation sur les parcours différenciés a été très enrichissant pour moi car il m'a demandé de travailler sur la plupart des points de construction d'une séance. \\
\remark{à compléter Variables didactiques ,ce que je garde comme paramètres :eval diagnostique, forme, temps, aides, gestion de la salle\ldtos}\\
Je le referai mais pas dès la rentrée de la 1ère année ou de manière light. Dès le début de la 2e année. Plutôt sur séquences de calcul ou constructions géométriques. Forme modifiée, intégrer construction "à la Xavère"\\
Je pense utiliser le tétra-aide avec mes élèves mais pas avec un parcours différencié (plutôt tutorat là) + îlots (pas forcément bonifiés) : objectif responsabiliser les élèves et modifier les formes d'apprentissage.\\
Évaluations différenciées intéressant dans une classe découragée + préparation aux examens je trouve (motiver les élèves, éviter le décrochage).
Math's up réutilisé pour les formes d'apprentissage.

\paragraph*{Retour de Xavière :}

\remark{1.	Le (re)ferais-je ?
	2.	De quelle manière (ce que je garderais) ? Avec qui ou sur quelle séquence ?
	3.	Pour quel objectif ?
4. autres expérimentations non abordées dans le mémoire que je referai ou non}
\paragraph*{Retour de Victoire :}
\remark{1.	Le (re)ferais-je ?
	2.	De quelle manière (ce que je garderais) ? Avec qui ou sur quelle séquence ?
	3.	Pour quel objectif ?
	4. autres expérimentations non abordées dans le mémoire que je referai ou non}

\paragraph*{}
Nous avons effectué nos premières expérimentations assez tôt dans l'année (fin octobre-début novembre) et il nous est arrivé de ne pas pouvoir effectuer notre séance normalement à cause de problèmes de gestion de classe. Au fil de l'année nous avons fait évolué nos pratiques, mais la période de "domptage" \remark{je ne trouve pas de bon mot} de la classe a été une période difficile pour construire notre pratique de différenciation. C'est pourquoi nous \remark{ou juste Julia ?} attendrions d'avoir plus d'assurance \remark{je ne trouve pas de bon mot} avec nos futures classes avant de ré-expérimenter certaines de nos pratiques (parcours différenciés, \remark{si vous pensez avoir d'autres pratiques qui demandent de maîtriser sa classe avant d'être mises en place})
\paragraph*{Retour de nos élèves}
\remark{Transition avec la suite. A ECRIRE}
\subsection{Mesurer l’impact de notre travail}
a.	Notre regard sur l’impact
b.	Difficulté à mesurer
c.	Conseils reçus et mis en place ou non
\subsection{Autres expérimentations liées à la différenciation ou l’autonomie}
Nos principales expérimentations n'ont pas été les seuls dispositifs de différenciation ou de travail sur l'autonomie que nous avons menés en classe.
a.	Présentation rapide
Quoi ? Pourquoi ? Modalités ? En complément/indépendante du reste ?
b.	On garde/on jette
\subsection{Retour à la problématique / aux problématiques}
a.	Avons-nous trouvé un dispositif adapté ?
b.	« ré exploitation possible ? » si oui. Modification/autre piste sinon 
