\section{Discussions}
Dans cette partie, nous faisons le point sur nos pratiques avec un regard critique enrichi par les retours de nos collègues et de nos élèves, et confrontés aux expérimentations d'autres collègues de l'ESPÉ. Nous aborderons également la question de l'impact des pratiques expérimentées sur l'acquisition des connaissances et des compétences de nos élèves, et sur leur capacité à travailler en autonomie. Nous traiterons enfin des autres dispositifs mis en place lors de notre année de stage et dont l'analyse (brève) peut, selon nous, enrichir ce mémoire.
\subsection{Discussion autour de nos pratiques}
Comme indiqué dans la partie \ref{Expérimentations}-Expérimentations, nous avons effectué des choix d'expérimentation différents au regard de nos besoins et de nos lectures de début d'année.\\

\subsubsection{Comparaison de nos expérimentations}
\paragraph*{}Nous avons toutes choisi de mettre en place des dispositifs incluant tous élèves de nos classes. Ces dispositifs ont des impacts plus ou moins importants sur les élèves selon leur niveau, mais nous avons choisi de tous les inclure dans nos dispositifs.\\
Dans les faits, le tutorat des élèves, les exercices de remédiation ou les défis des parcours différenciés vont avoir une incidence plus importante sur les élèves en grande difficulté ou au contraire très à l'aise sur les compétences travaillées. Les évaluations différenciées et les supports adaptés à chaque élève sont en revanche supposées avoir un impact sur l'autonomie et l'acquisition des compétences de tous les élèves, sans distinction de niveau.\\
\remark{Je ne sais pas si ce que j'écris vous paraît pertinent. Je pense que c'est une première différence intéressante entre les dispositifs. Pour la suite, @Xavière, je te laisse modifier à ta guise!}
\paragraph*{}Dans la construction des parcours différenciés, Xavière et Julia ont toutes les deux mis en place des parcours de formes différentes. Alors que Julia propose un guide de parcours ainsi qu'une série d'exercices selon la réussite ou non de l'élève, Xavière propose une feuille d'exercices avec une version plus ou moins difficile de l'exercice (variation dans le choix des variables didactiques) et l'élève choisit celui qu'il traite \remark{il me semble, de mémoire}. Dans les deux cas, un élève en difficulté aura la possibilité de traiter un exercice plus facile pour appréhender une notion, et les élèves les plus à l'aise auront accès à des problèmes plus difficiles à aborder.\\
La différence de présentation présente cependant une différence d'approche très intéressante car dans l'expérimentation de Julia, les élèves n'ont pas le choix de traiter la version facile d'un exercice ou le défi de leur parcours. Ils doivent également identifier par eux-même une situation d'échec ou de fragilité dans le traitement d'un exercice, qui demande d'effectuer un exercice supplémentaire. Au contraire, l'approche de Xavière donne le choix aux élèves dans le niveau de l'exercice à traiter. D'un côté on peut s'attendre à ce que les élèves à l'aise essaient toutes les versions ou que les élèves ne choisissent que la facilité, d'un autre côté, un élève en difficulté ne passera pas par une étape « d'échec » avant de traiter un exercice plus abordable.
\paragraph*{}
\remark{Vous avez toutes les deux testé des évaluations différenciées mais pas du tout sous la même forme (je crois que Xavière c'était plus dans le sens de s'auto évaluer), si ça prend du sens, vous pourriez peut-être en parler ici?}\\
\remark{Pour la suite j'ai déplacé la partie "notre retour" après les retours des autres. Comme ça on peut s'appuyer sur cette partie pour alimenter notre propre retour}
\subsubsection{Points d'attention suite aux échanges avec nos collègues}\label{retour_collegues}
\paragraph*{}
Dans le cadre du mémoire, notre responsable de suivi nous a en particulier aidées à identifier les modalités de différenciation ou de mise en autonomie dans nos pratiques. Elle nous a également indiqué quels collègues de la formation avaient potentiellement des approches communes ou très différentes des notres. Nous avons ainsi eu l'occasion d'échanger avec le groupe de Fanny Mauhé, Morgane Petigat et Léa Serrano\cite{memoire_fanny} en particulier lors des phases de documentation, ainsi qu'avec le groupe formé par Cédric Hamon et Juliette Kirouane\cite{memoire_eval_differenciee}, en particulier lors de la journée de valorisation des mémoires.\\
Nous avons enfin présenté nos pratiques à nos tutrices terrain qui ont pu assister à leurs mises en application en classe et nous faire part de leurs observations.
\paragraph*{}
Les différents échanges avec nos interlocuteurs a mis en avant des paramètres à prendre en compte et que nous n'avions pas toujours anticipé dans la préparation de nos expérimentations. La forme donnée aux supports est, par exemple, ressortie de la part de tous nos interlocuteurs (avec plus ou moins d'importance). Ainsi, la tutrice de Julia a indiqué que certains parcours présentés aux élèves de 6\up{e} avaient une forme inadaptée, voir décourageante pour les élèves, alors qu'elle serait adaptée à une classe de 4\up{e} (après quelques ajustements). Dans leur travail sur la différenciation, le groupe de Fanny, Morgane et Léa a également observé que la forme des supports de travail ou de différenciation avait une grande importance sur la mise au travail des élèves. \remark{si j'ai bien compris?}\\
De même l'identification des différents temps de classe est apparue encore plus importante que lors de la préparation d'une séance « normale » pour la plupart des pratiques analysées. En effet, la mise en place d'ilots bonifiés, de tétra'aide ou de parcours de différenciation demande de connaitre par avance les temps de travail individuel ou collectif, les périodes de déplacement des élèves (et leurs buts), les durées approximatives de chaque période, le possibles sources d'agitation ou de sollicitations importantes\ldots Il nous a notamment été demandé lors des analyses de séance d'identifier les différents moments de nos séances, leur objectif et le déroulement anticipé puis observé.\\
Ces temps de classes vont être variables selon le travail effectué mais aussi en fonction des élèves. Par exemple, une classe de 6\up{e} va généralement demander plus d'explications sur ce qui doit être fait qu'une classe de 4\up{e}. Les élèves de 4\up{e} sont plus timides dans l'entraide : le tutorat après la 4\up{e} ne se passe qu'entre copains\footnote{Sans doute pour éviter de paraitre trop brillant auprès des autres. Il doit y avoir tout un mémoire à faire sur
la relation entre l'apprentissage et les élèves en tant que groupe !}.
De plus, la maturité des élèves aura un impact sur sa capacité à effectuer le travail demandé en autonomie. Ainsi, des élèves de 6\up{e} auront plus de difficultés à identifier quelle série d'exercices ils seront capables de faire dans le temps imparti pour optimiser leur note. Pour les mêmes raisons, ces élèves ne seront pas forcément aptes à s'auto-évaluer dans des exercices de raisonnement de parcours différenciés (pour des exercices de géométrie par exemple). Nous devons donc également adapter nos pratiques au niveau d'autonomie de nos élèves sur les tâches demandées et leur apprendre à les effectuer pour la suite de leur scolarité (voir notamment la réflexion de Philippe Meirieu\cite{Meirieu_autonomie} sur le sujet).\\
\remark{Différences/similitudes avec travaux de groupe sur éval différenciée}
Les visites de nos tutrices en séance nous ont permis d'identifier les différentes variables didactiques sur lesquelles nous pouvons différencier dans nos dispositifs. Ainsi, l'analyse d'exercices effectués en classe a mis en avant l'importance de la question des objectifs à atteindre par les élèves dans la résolution de ceux-ci, mais aussi des objectifs leur correction. Cette dernière question est d'autant plus importante dans la construction des parcours différenciés et des évaluations différenciées où les élèves ne traitent pas les mêmes exercices et où la correction n'est pas toujours effectuée en classe entière.\\
Malgré les nombreux échanges entre nous ou avec nos collègues, il a été rapidement clair qu'un dispositif qui fonctionne lors d'une séance ne fonctionnera pas forcément avec une autre classe ou même à un autre horaire. Nous avons tenté d'identifier les paramètres de nos expérimentations qui permettent de rendre nos dispositifs ré-exploitables dans le futur (forme des supports, organisation de la salle, découpage de la séance, choix des exercices\ldots).

\subsubsection{Nos retours}
Dans cette partie, chacune d'entre nous effectue un retour de sa propre expérimentation et commentera les dispositifs mis en place par les deux autres. \remark{si ça vous va, sinon on organise ça par dispositif ou autre :)}\\
\paragraph*{Retour de Julia :}
le travail de recherche et d'expérimentation sur les parcours différenciés a été très enrichissant pour moi car il m'a demandé de travailler sur la plupart des points de construction d'une séance. \\
\remark{à compléter Variables didactiques ,ce que je garde comme paramètres :eval diagnostique, forme, temps, aides, gestion de la salle\ldots}\\
Je le referai mais pas dès la rentrée de la 1\up{ère} année ou de manière allégée. Dès le début de la 2\up{e} année. Plutôt sur séquences de calcul ou constructions géométriques. Forme modifiée, intégrer construction « à la Xavère »\\
Je pense utiliser le tétra'aide avec mes élèves mais pas avec un parcours différencié (plutôt du tutorat dans ce cas), ainsi que l'organisation de la classe en ilots (pas forcément bonifiés). L'objectif étant toujours de responsabiliser les élèves et de modifier les formes d'apprentissage.\\
Évaluations différenciées intéressant dans une classe découragée + préparation aux examens je trouve (motiver les élèves, éviter le décrochage).
Math's up réutilisé pour les formes d'apprentissage.

\paragraph*{Retour de Xavière :}

\remark{1.	Le (re)ferais-je ?
	2.	De quelle manière (ce que je garderais) ? Avec qui ou sur quelle séquence ?
	3.	Pour quel objectif ?
4. autres expérimentations non abordées dans le mémoire que je referai ou non}
\paragraph*{Retour de Victoire :}
\remark{1.	Le (re)ferais-je ?
	2.	De quelle manière (ce que je garderais) ? Avec qui ou sur quelle séquence ?
	3.	Pour quel objectif ?
	4. autres expérimentations non abordées dans le mémoire que je referai ou non}

\paragraph*{}
Nous avons effectué nos premières expérimentations assez tôt dans l'année (fin octobre-début novembre) et il nous est arrivé de ne pas pouvoir effectuer notre séance normalement à cause de problèmes de gestion de classe. Au fil de l'année nous avons fait évolué nos pratiques, mais la période de prise en charge de la classe a été une période difficile pour construire notre pratique de différenciation. C'est pourquoi nous \remark{ou juste Julia ?} attendrions d'avoir plus d'assurance \remark{je ne trouve pas de bon mot} avec nos futures classes avant de ré-expérimenter certaines de nos pratiques (parcours différenciés, \remark{si vous pensez avoir d'autres pratiques qui demandent de maîtriser sa classe avant d'être mises en place})
\paragraph*{Retour de nos élèves}
Les retours de nos élèves sont ont été obtenus de manière informelle, généralement en fin de cours auprès de 2-3 élèves ayant activement participé au dispositif ou au contraire n'ayant pas beaucoup travaillé. \remark{N'hésitez pas à corriger si ça n'est pas votre cas!} \\
Dans tous les cas, les élèves ont apprécié la nouveauté dans le mode de travail ou les supports proposés. Julia a eu un retour négatif sur ses deux classes, un élève n'a pas apprécié devoir travailler. Il l'a exprimé en disant ne pas savoir quoi faire et en annonçant que personne ne voulait l'aider. Après discussion avec cet élève, il est apparu qu'il ne souhaitait pas travailler et qu'il cherchait à se cacher derrière une fausse incompréhension pour ne rien faire. Cette intervention a cependant été l'occasion de redonner aux élèves les modalités de travail et de vérifier leur bonne compréhension.\\
En dehors de ces retours informels, nous avons eu une grande difficulté à mesurer l'impact de nos expérimentations sur nos élèves. Nadine Grapin nous a suggéré de proposer un questionnaire aux élèves mais nous avons choisi de ne pas le faire cette année. En effet, nos expérimentations ont été très variables d'une séance à l'autre et ayant nous-même du mal à anticiper ce que nos modifications apportent aux élèves, nous considérons que les élèves ne sont pas assez mûrs pour estimer l'impact de nos différentes mesures sur leur autonomie ou leur acquisition des compétences. Notre repose également sur d'autres paramètres que nous détaillerons dans la section suivante.
\subsection{Mesurer l’impact de notre travail}
%a.	Notre regard sur l’impact
\paragraph*{} Comme nous l'avons plusieurs fois indiqué dans ce document, il nous a été difficile d'anticiper les impacts de nos pratiques sur les élèves lors de la préparation de nos séances. Nous avons cependant le sentiment que nos élèves ont apprécié travailler de manière différente et ont gagné en autonomie. Julia trouve cependant que ses élèves ne sont pas devenus aussi autonomes que ce quelle prévoyait \remark{si c'est aussi votre cas modifiez "Julia" en "Nous trouvons"}. Après échange avec sa tutrice et des collègues, il lui semble que cela s'explique par le délai important qu'a demandé la construction d'une ambiance de classe studieuse (surtout en 6\up{e}) et par l'aspect "exceptionnel" des séances d'expérimentations. Selon elle, les expérimentations l'ont également été pour les élèves et une répétition des séances de parcours différenciés \remark{ou de tutorat si c'est aussi votre cas?} systématiseraient le travail en autonomie des élèves. \\
\remark{Si votre avis est différent ou si vous voulez compléter, faites vous plaisir!}\\
Malgré des modifications à apporter après analyse a posteriori des séances, les dispositifs expérimentés semblent avoir été bénéfiques sur l'acquisition des compétences des élèves. Les évaluations différenciées nous paraissent ainsi avoir motivé les élèves décrocheurs, rassurés les élèves les plus fragiles et suffisamment exigentes pour les élèves les plus en avance dans l'acquisition des compétences. Pour les parcours différenciés, alors que les élèves les plus fragiles demandent un ajustement de certains parcours sur la forme, les élèves ont traité en moyenne deux fois plus d'exercices lors des séances en autonomie sur les séquences numériques. Les exercices de géométrie ont quant à eux été traités de manière plus difficile selon les compétences testées. De manière générale, les élèves ont traités plus d'exercices qu'à l'accoutumée, mais Julia a dû intervenir de manière plus fréquente, avec beaucoup d'interventions en plénière (voir l'analyse de Julia partie \ref{retour_parcours}). La diversification des supports a, quant à elle, permis de varier les modes d'apprentissage des élèves qui ont ainsi acquis de manière efficace les compétences demandées. En choisissant leur support d'apprentissage, \remark{le terme est-il approprié ?} les élèves ont, selon nous, appris à identifier les canaux d'apprentissages \remark{idem?} qui leur correspondent le mieux.\\
%difficultés de mesure d'impact
\paragraph*{} La plupart des retours que nous avons faits ou que nous avons obtenus sont des ressentis, des observations obtenues sur une séance, à un instant donné pour un élève donné. En nous penchant sur la littérature autour de la différenciation ou sur l'autonomie, nous n'avons pas trouvé de critère clair et mesurable permettant d'évaluer l'impact de nos pratiques. En effet, une évolution des notes d'un élève peut être conséquente d'une motivation personnelle de l'élève obtenue suite à un dispositif mis en place par l'équipe pédagogique, ou suite à une révélation personnelle. L'élève a peut être simplement trouvé plus d'intérêt pour le chapitre étudié ou a au contraire été à l'aise avec la pratique expérimentée et a su travaillé grâce aux outils pédagogiques qui lui ont été proposés.\\
L'évaluation de l'efficacité de nos pratiques nous est d'autant plus difficile que nous sommes novices dans l'enseignement (nous avons donc peu de recul sur ce que les élèves peuvent produire de manière générale). Cela est encore plus vrai pour Xavière qui enseigne en établiessement prioritaire, ou pour Julia dont la classe de 6\up{e} est une classe partiellement sans note (classe non notée en Français, Histoire-Géographie et Technologie et en classe inversée en Français). Leurs élèves sont sujets à des dispositifs pédagogiques nombreux et variés, ce qui augmente la difficulté d'évaluer l'impact des expérimentations à long terme.

%Conseils reçus et mis en place ou non
\paragraph*{} Ne trouvant pas d'exemples pratiques de mesure d'impact de dispositifs de différenciation pédagogique sur l'autonomie ou l'acquisition des compétences, nous avons d'abord demandé conseil à Nadin Grapin. Comme indiqué dans la section précédente, celle-ci nous a conseillé de mettre en place un questionnaire auprès des élèves pour avoir un critère de mesure  de l'autonomie des élèves (subjectif mais du point de vue de l'élève). Nous avons effectué des recherches en ce sens mais n'avons pas trouvé de réponse satisfaisante sur les questions à poser, quand les poser ou comment interpréter les réponses obtenues. Nous manquions également de temps pour proposer un questionnaire suite à une expérimentation (nous avons testé nos pratiques assez tôt dans l'année) et ne n'étions pas convaincue de produire une analyse pertinente à partir des réponses éventuellement collectées.\\
Nous avons également interrogé nos collègues de l'ESPE, nos tutrices et nos formateurs sur leur manière d'évaluer l'impact de leurs pratiques sur l'autonomie des élèves ou leurs évolution. L'observation des notes des élèves est évidemment revenu mais on nous a également proposé des méthodes d'évaluation que nous pouvions tester lors des séances d'expérimentation qui étaient encore en cours.
Ainsi, Marine Doceul, la tutrice terrain de Julia, a indiqué qu'elle créait une évaluation "Travailler en autonomie" pour les élèves sur deux heures de séances. Les élèves peuvent avoir vert plus, vert, jaune ou rouge (établissement avec évaluation par compétences). L'autonomie est notée de vert à rouge dans son cas. Au début de chaque séance, elle annonce aux élèves qu'ils ont tous vert à l'évaluation et qu'elle évalue leur capacité à se mettre immédiatement au travail lors des phases de travail individuel. Si elle doit demander à un élève de travailler après un certain délai, sa note passe à jaune. Lors de la seconde séance, elle propose les même modalités aux élèves mais repart de la grille de notes précédente. Un élève ayant eu jaune à la séance précédente peut passer à rouge, un élève ayant obtenu vert à la séance précédente peut finir à jaune sur l'évaluation. L'évaluation est dégressive car Madame Doceul et nos collègues ont mis ce dispositif en classe suite à des problèmes de mise au travail de la classe dans toutes les matières.\\
Lorsqu'elle a expérimenté ce mode d'évaluation, Julia a proposé le mêmes modalités aux élèves, mais avec la possibilité de rattraper sa note pour les élèves faisant un effort particulier en seconde séance (travail en autonomie et dans le calme). Cela lui a permis d'observer que plus d'élèves qu'elle ne le pensait parvenaient à se mettre rapidement au travail et sans aide. Cependant, ce mode de mesure de l'autonomie est problématique car le fait de savoir qu'ils sont évalués encourage les élèves à travailler en autonomie (plus peut-être que le dispositif expérimenté). Ce mode d'évaluation n'a pas été retenu pour l'évaluation de l'impact des expérimentations mais nous sert actuellement pour motiver nos élèves à se mettre au travail rapidement.\\
Finalement, nous avons préféré observer des élèves-test sur plusieurs séances, en collectant leur travail et en observant le résultat de leur évaluation sommative. Cette idée nous a été donnée par Nadine Grapin lors d'un entretien autour du mémoire. Nos travaux ayant été déjà bien avancés, nous avons peu de vision sur les capacités de nos élèves tests en début d'année et donc sur leur évolution réelle.
\remark{J'arrive pas à conclure HELP!}

\subsection{Autres expérimentations liées à la différenciation ou l’autonomie}
Nos principales expérimentations n'ont pas été les seuls dispositifs de différenciation ou de travail sur l'autonomie que nous avons menés en classe. Nous avons par exemple, suite à des corrections d'exercices, proposés le même sujet avec des valeurs différentes (exercice employant des nombres décimaux proposé avec des valeurs négatives à des élèves toujours en difficulté suite à la correction).\\
En terme de pratiques innovantes, Julia a expérimenté le \textit{Math's up}\cite{maths_up} avec sa classe de 6\up{e} après une présentation de nos collègues de l'ESPÉ. Les élèves ont effectué cette exercice en autonomie et elle a pu proposer des mots différents selon le niveau global du groupe qui jouait. Xavière a quant à elle proposé un projet de groupe de création de jeu vidéo guidé mais en autonomie.\\
Julia va également proposer un projet avec sa classe de 5\up{e} autour des statistiques en partenariat avec les professeurs d'anglais (séquence "Grands nombres"). Les élèves auront un guide de travail en autonomie à faire en groupe durant un créneau dédié et le travail sera revu par groupe puis en classe entière.
\remark{INSERER ICI AUTRES EXPERIMENTATIONS D'AUTONOMIE ou DE DIFF}
L'ensemble de ces dispositifs nous ont permis de proposer de nouveaux modes de travail à nos élèves qui les ont apprécié et ont correctement travaillé les notions demandées. En dehors du projet bilingue où il faut attendre qu'il soit réalisé, nous comptons renouveler ces expériences dans les années à venir.
\subsection{Retour à la problématique / aux problématiques}
Suite à nos échanges et aux retours obtenus sur nos pratiques, nous estimons que les différents dispositifs expérimentés nous ont permis de faire travailler nos élèves de manière différente en autonomie, tout en leur proposant un contenu ou des supports variés et adaptés à leur besoin. Nous avons le sentiment que nos élèves sont plus autonomes mais ne savons pas mesurer dans quelle mesure cela vient de nos pratiques.\\
Nous trouvons également que l'accumulation de pratiques pédagogiques différenciées répétées mais variées et des travaux en autonomie différents et fréquents sont plus enrichissants pour nos classes. Nous avons observé un regain de motivation globale de la classe suite au lancement des expérimentations \remark{Je ne sais pas si ça fait sens? Si oui, je rajouterai les références qui indiquent que varier les pratiques est bénéfique}.\\
Certains de nos dispositifs étaient adaptés à certaines classes mais pas à d'autres, ou pour des séquences différentes (voir l'analyse du retour de nos collègues partie \ref{retour_collegues}), comme nous l'avons plusieurs fois abordé et comme l'ont signalé les différents experts consultés par le Conseil national d'évaluation du système scolaire (Cnesco)\cite{cnesco_synthese}\cite{cnesco_notes_experts}, la ré-exploitation de nos travaux demande une adaptation de nos pratiques au niveau de connaissances des élèves, à leurs habitudes de travail, aux conditions matérielles (salles, présence projecteur\ldots),  à l'organisation des heures dans l'emploi du temps\ldots\\
Nous tenterons cependant de remettre en pratique n os dispositifs dans les années à venir avec nos futures classes.
