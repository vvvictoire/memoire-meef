\section{Discussions}
Dans cette partie, nous faisons le point sur nos pratiques avec un regard critique enrichi par les retours de nos collègues et de nos élèves, et confrontés aux expérimentations d'autres collègues de l'ESPE. Nous aborderons également la question de l'impact des pratiques expérimentées sur l'acquisition des connaissances et des compétences de nos élèves, et sur leur capacité à travailler en autonomie. Nous traiterons enfin des autres dispositifs mis en place lors de notre année de stage et dont l'analyse peut, selon nous, enrichir ce mémoire.
\subsection{Nos choix d'expérimentation}	
Comme indiqué dans la partie \ref{Expérimentations}-Expérimentations, nous avons effectué des choix d'expérimentation différents au regard de nos besoins et de nos recherches littéraires.\\
Nous analyserons dans cette partie les similarités et les différences entre nos dispositifs, ainsi que les retours
\subsubsection{Similarités et différences entre les dispositifs (parcours et évaluation différenciée (?))}
i.	Similarités
ii.	Différences
iii.	Pourquoi ? Quel impact ?
\subsubsection{Retours}
\paragraph*{Regards croisés (dont par celles qui n’ont pas fait l’expérience dans la classe)}
1.	Le (re)ferais-je ?
2.	De quelle manière (ce que je garderais) ? Avec qui ou sur quelle séquence ?
3.	Pour quel objectif ?
\paragraph*{Retours de nos tutrices sur nos expérimentations et analyse de travaux de collègues de l’ESPE}
Exemple de thèmes :
1.	Importance de la forme
2.	Choix du public 
(les 6e sont trop immatures pour s’autoévaluer par exemple pour le parcours)
3.	Différences/similitudes avec travaux de groupe sur éval différenciée
4.	Choix des variables didactiques ?
5.	Transposition des expérimentations
6.	Notre faible expérience dans l’enseignement nous a fait rater des choses dans la préparation de nos expérimentations et nous demande un plus gros effort de gestion de classe qui impactent négativement notre travail ?
7.	 \ldots
\paragraph*{Retour de nos élèves}
Transition avec la suite
\subsection{Mesurer l’impact de notre travail}
a.	Notre regard sur l’impact
b.	Difficulté à mesurer
c.	Conseils reçus et mis en place ou non
\subsection{Autres expérimentations liées à la différenciation ou l’autonomie}
Nos principales expérimentations n'ont pas été les seuls dispositifs de différenciation ou de travail sur l'autonomie que nous avons menés en classe.
a.	Présentation rapide
Quoi ? Pourquoi ? Modalités ? En complément/indépendante du reste ?
b.	On garde/on jette
\subsection{Retour à la problématique / aux problématiques}
a.	Avons-nous trouvé un dispositif adapté ?
b.	« ré exploitation possible ? » si oui. Modification/autre piste sinon 
