\section{Introduction}
{\color{red}De l'ordre de 1 à 2 pages
\begin{itemize}
\item Elle résume de manière succincte l'origine du questionnement initial et explique le titre du mémoire.
\item Elle introduit et justifie le plan du mémoire, en indiquant bien dans quelle partie la problématique est définie et justifiée.
\item Elle explicite par ailleurs comment vous avez collaboré pour rédiger le mémoire à plusieurs (synthèse et analyses).
\end{itemize}

Ce qui suit est issu du bilan d'étape, il nécessite d'être relu et adapté.}

Ayant chacune un parcours différent, nous nous sommes retrouvées autour d’un même constat suite à la prise en main de nos classes : le mode de fonctionnement habituel de nos classes ne nous permet pas de répondre aux besoins spécifiques de nos élèves et de leur donner les moyens de travailler correctement en autonomie. Sur les conseils de nos tuteurs terrain, nous avons décidé de profiter du travail sur le mémoire pour chercher un mode de fonctionnement ou des solutions spécifiques à notre public. Nous nous sommes rapidement tournées vers le thème de la différenciation pédagogique, en ayant une connaissance quasi nulle sur le sujet.

La première étape de notre démarche a donc été de nous renseigner auprès de Nadine Grapin, notre directrice de mémoire, qui nous a orientées vers des lectures et des premiers axes de réflexion sur le sujet : quels objectifs ? Quelle(s) forme(s) de différenciation ?

Suite à différents échanges avec nos encadrants et nos collègues et après une première série de lectures, nous avons choisi d’orienter le sujet de notre mémoire vers la différenciation pédagogique dans le but de rendre les élèves autonomes. Ce choix commun est lié à des motivations différentes pour chacune d’entre nous.
Le cheminement de nos réflexions sera développé dans la première partie de ce document, où nous ferons l’état des lieux des travaux menés sur la période de novembre et décembre et où nous détaillerons les pistes de réflexion qui en sont ressorties.
Dans une seconde partie, suite au bilan que nous viendrons de dresser, nous listerons les pistes de travail que nous envisageons de suivre. Il sera notamment question des lectures envisagées, des mises en pratiques de procédés rencontrés lors de nos recherches et les problématiques de mise en place associées.

Nous développerons en troisième partie notre mode de fonctionnement actuel et l’organisation prévue des pistes identifiées en seconde partie.
Nous terminerons enfin par les retours que nous avons reçus suite à la présentation orale du bilan d’étape du 16 janvier 2019.

\paragraph{Méthodes de travail}


Une description plus approfondie et un manuel de mise en place est dans les projets
afin de pouvoir être utilisé par de futurs étudiants. Note : le template est déjà ré-utilisé par d'autres étudiants pour le portfolio ou le mémoire.

{\color{red}Une autre formulation issue du bilan d'étape, l'ensemble est à reformuler, rendre clair et concis.}

Nous travaillons chacune de notre côté avec des points hebdomadaires de partage
de connaissances et expériences. Nous avons également beaucoup d’échanges informels
en dehors de l’ESPÉ sur un salon de discussion virtuel dédié. Nous discutons aussi
chacune de notre côté avec nos encadrants et collègues, et rapportons les remarques au trinôme.

Concernant la rédaction du mémoire en lui-même, nous avons utilisé \LaTeX\footnote{\url{https://www.latex-project.org/}},
outil avec lequel nous avons l’habitude de travailler, avec une gestion des versions
et des commentaires sur GitHub\footnote{\url{https://github.com}}.
Ceci nous a permis de très facilement gérer et suivre les modifications de chacune
et de passer moins de temps sur la mise en page. Un système d'intégration continue
a été mis en place afin de s'assurer de l'intégrité de notre mémoire, et de
permettre à Nadine Grapin de mieux suivre notre avancée. Les fichiers sources
du mémoire sont disponibles
à l'adresse suivante : \url{https://github.com/vvvictoire/memoire-meef/}
