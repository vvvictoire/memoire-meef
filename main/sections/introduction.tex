\section{Introduction} % Rappel : 2 pages max !

% PARTIE ORIGINE DU QUESTIONNEMENT / JUSTIFICATION DU TITRE, fortement issue du bilan d'étape
Ayant chacune un parcours différent, nous nous sommes retrouvées autour d’un même constat suite à la prise en main de nos classes : le mode de fonctionnement habituel de nos classes ne nous permet pas de répondre aux besoins spécifiques de nos élèves et de leur donner les moyens de travailler correctement en autonomie. Sur les conseils de nos tutrices terrain, nous avons décidé de profiter du travail sur le mémoire pour chercher un mode de fonctionnement ou des solutions spécifiques à notre public. Nous nous sommes rapidement tournées vers le thème de la différenciation pédagogique, en ayant une connaissance quasi nulle sur le sujet.

La première étape de notre démarche a donc été de nous renseigner auprès de Nadine Grapin, notre directrice de mémoire, qui nous a orientées vers des lectures et des premiers axes de réflexion sur le sujet : quels objectifs ? Quelle(s) forme(s) de différenciation ?

Suite à différents échanges avec nos encadrants et nos collègues et après une première série de lectures, nous avons d'abord choisi d’orienter le sujet de notre mémoire sur la découverte de pratiques pédagogiques en autonomie dans le but de rendre les élèves plus autonomes et d'améliorer l'acquisition des compétences.
\remark{Validez-vous? OK, V.}

% PARTIE JUSTIFICATION DU PLAN (doit mentionner dans quelle partie la problématique sera définie et justifiée), repris du bilan d'étape, adaptée à notre nouveau plan
Dans une première partie, à travers les différentes lectures effectuées, nous développerons la problématique et le cheminement de nos réflexions, différent pour chacune d'entre nous. \\
Dans une seconde partie, nous détaillerons chacune les expérimentations menées à titre personnel dans chacune de nos classes. L'objectif est, encore une fois, de montrer l'évolution de notre réflexion et de nos pratiques au cours de l'année suite à ces expérimentations ou de nos lectures.\\
Nous confronterons en troisième partie nos retours d'expérience et dresserons un bilan de nos expérimentations par rapport à la problématique posée en début de document.

% PARTIE COLLABORATION ET REDACTION, la formulation issue du bilan d'étape a reçu l'approbation de Nadine
Nous travaillons chacune de notre côté avec des points hebdomadaires de partage de connaissances et expériences. Nous avons également beaucoup d’échanges informels en dehors de l’ESPÉ sur un salon de discussion virtuel dédié (Discord\footnote{\url{https://discordapp.com}}). Nous discutons aussi chacune de notre côté avec nos encadrants et collègues, et rapportons les remarques au trinôme.

Concernant la rédaction du mémoire en lui-même, nous avons utilisé \LaTeX\footnote{\url{https://www.latex-project.org/}, se prononce comme « la tech »},
outil avec lequel nous avons l’habitude de travailler, avec une gestion des versions et des commentaires sur GitHub\footnote{\url{https://github.com}}. Ceci nous a permis de très facilement gérer et suivre les modifications de chacune et de passer moins de temps sur la mise en page. La mise en page du mémoire et un système d'intégration continue a été mis en place par Victoire afin de s'assurer de l'intégrité de notre mémoire, et de permettre à Nadine Grapin de mieux suivre notre avancée. Victoire a également la charge de référent technique.

Les fichiers sources du mémoire sont disponibles à l'adresse suivante : \url{https://github.com/vvvictoire/memoire-meef/}. Une description plus approfondie des sources et un manuel de mise en place du projet \LaTeX{} sont en cours de rédaction afin qu'il puisse être utilisé\footnote{Le template est déjà ré-utilisé par d'autres étudiants pour le portfolio ou le mémoire} par de futurs étudiants lors des années prochaines\footnote{Le template sera publié à l'adresse
suivante : \url{https://github.com/vvvictoire/memoire-upec}}.

Ce document comporte des parties communes et des parties rédigées de manière indivuduelle. Julia est la rédactrice principale des parties communes et Xavière est la relectrice principale.
Victoire s'occupe du peaufinage, des détails et de la mise en page. Les parties individuelles portent le nom de leur rédactrice dans leur titre et sont relues par le trinôme.
