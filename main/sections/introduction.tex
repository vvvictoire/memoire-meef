\section{Introduction}
%Ceci est une phrase qui fait référence à la bibliographie\cite{bibliotest}.


De l'ordre de 1 à 2 pages 
-	Elle résume de manière succincte l'origine du questionnement initial et explique le titre du mémoire.
-	 Elle introduit et justifie le plan du mémoire, en indiquant bien dans quelle partie la problématique est définie et justifiée. 
-	Elle explicite par ailleurs comment vous avez collaboré pour rédiger le mémoire à plusieurs (synthèse et analyses).


\textit{<- Début du bilan d'étape ->}

Ayant chacune un parcours différent, nous nous sommes retrouvées autour d’un même constat suite à la prise en main de nos classes : le mode de fonctionnement habituel de nos classes ne nous permet pas de répondre aux besoins spécifiques de nos élèves et de leur donner les moyens de travailler correctement en autonomie. Sur les conseils de nos tuteurs terrain, nous avons décidé de profiter du travail sur le mémoire pour chercher un mode de fonctionnement ou des solutions spécifiques à notre public. Nous nous sommes rapidement tournées vers le thème de la différenciation pédagogique, en ayant une connaissance quasi nulle sur le sujet.

La première étape de notre démarche a donc été de nous renseigner auprès de Nadine Grapin, notre directrice de mémoire, qui nous a orientées vers des lectures et des premiers axes de réflexion sur le sujet : quels objectifs ? Quelle(s) forme(s) de différenciation ? 

Suite à différents échanges avec nos encadrants et nos collègues et après une première série de lectures, nous avons choisi d’orienter le sujet de notre mémoire vers la différenciation pédagogique dans le but de rendre les élèves autonomes. Ce choix commun est lié à des motivations différentes pour chacune d’entre nous. 
Le cheminement de nos réflexions sera développé dans la première partie de ce document, où nous ferons l’état des lieux des travaux menés sur la période de novembre et décembre et où nous détaillerons les pistes de réflexion qui en sont ressorties.
Dans une seconde partie, suite au bilan que nous viendrons de dresser, nous listerons les pistes de travail que nous envisageons de suivre. Il sera notamment question des lectures envisagées, des mises en pratiques de procédés rencontrés lors de nos recherches et les problématiques de mise en place associées.

Nous développerons en troisième partie notre mode de fonctionnement actuel et l’organisation prévue des pistes identifiées en seconde partie.
Nous terminerons enfin par les retours que nous avons reçus suite à la présentation orale du bilan d’étape du 16 janvier 2019. 

\textit{<- Complément issu du bilan d'étape ->}

Suite aux différents entretiens que nous avons pu avoir avec Nadine Grapin et aux retours que nous avons eu suite à notre présentation du 16 janvier, nous avons eu de nouvelles réflexions : 
\begin{itemize}
    \item avons-nous une problématique ou plusieurs problématiques ? La réponse à cette question nous permettra de structurer notre mémoire.
    \item quelle est la place de la différenciation par rapport à l’autonomisation ? Nous avons un double objectif : celui de rendre les élèves autonomes pour pouvoir mieux gérer la classe et pouvoir consacrer plus de temps aux élèves en ayant besoin, et celui de différencier, à savoir fournir aux élèves une manière d’accéder aux compétences qui prend en compte leurs points forts et points faibles.
\end{itemize}
    
Nous nous sommes aussi rendues compte que nous faisons déjà plus de différenciation que nous le pensions (voir tableau en annexe) : nous avons tendance à construire le cours avec les élèves, mettre en place des méthodes non expertes avec les élèves, tout en fournissant des méthodes expertes aux élèves qui voudraient les utiliser.

De plus, cet entretien a permis à Julia de mieux formuler sa problématique qui s’avère être de trouver une pratique (le parcours différencié) s’appuyant sur l’autonomie des élèves pour mieux différencier. La présentation orale du parcours différencié a par ailleurs permis de mettre en avant des points de réflexion supplémentaires : quels exercices corriger ? Quels exercices sont obligatoires? Comment évaluer la progression des élèves et pour quoi ? Que faire une fois l’évaluation effectuée ?

\textit{<- Fin du bilan d'étape ->}