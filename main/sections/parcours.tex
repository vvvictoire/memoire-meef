\subsection{Parcours différencié (Julia)}

\textit{<- Début du bilan d'étape ->}
\paragraph{}
Le principe du parcours différencié est de proposer aux élèves de traiter une série d'exercices avec une progression adaptée aux difficultés que ceux-ci peuvent rencontrer. Ce parcours peut être complété par des aides progressives données par l'enseignant ou sous différents supports (capsules vidéos, document avec rappel de cours, questions intermédiaires\ldots ).
\paragraph{}
Avant de mettre en place les parcours différenciés, j'ai eu besoin de voir comment les élèves réagissaient face à une successions de travaux à faire en autonomie. J'ai donc mis en place une séance d'exercices en autonomie avec ma classe de sixième. \\
Les élèves ont commencé avec un exercice sur les fractions à faire seul, la correction et les exercices suivants se trouvant sur un îlot. J'ai ajouté une version "bis" de certains exercices disponibles pour les élèves ayant rencontré des difficultés ou voulant se rassurer. \\
Les élèves ont globalement joué le jeu, ils ont apprécié de pouvoir avancer à leur  rythme et de circuler librement dans la salle même si cela a causé de l'agitation en début de cours. Certains élèves ont cependant choisi de ne faire qu'un ou deux exercices simples, alors que d'autres se sont rapidement jetés sur les corrections.\\
\newline
Lors de cette expérience, je me suis rendue compte des élèves qui devaient a priori me poser des problèmes d'attitude étaient plus épanouis dans ce type de travail (lorsqu'ils faisaient l'effort d'essayer), alors que des élèves très à l'aise en condition \emph{classique} de travail individuel ont été perdus au démarrage de la première séance. De plus, une fois le dispositif en place, des élèves habituellement très discrets m'ont demandé de l'aide. J'ai cependant eu de grandes difficultés à \emph{jongler} entre les élèves. Ceux-ci étaient encore peu autonomes et avaient surtout besoin d'être rassurés. J'ai également eu du mal à trouver le temps à laisser à chacun et à identifier les moments où mettre le travail en commun.\\
Cela m'a confortée dans l'idée que la construction de parcours différenciés, avec des étapes clés et à réaliser en autonomie par les élèves pouvait avoir un apport pédagogique intéressant pour mes classes.

\paragraph{à modifier}
J'ai décidé de lancer ma première expérimentation de parcours différencié avec ma classe de cinquième pour la séquence \emph{Symétrie centrale}. Nous avons commencé la séquence par un travail de rappels sur la symétrie axiale, suivie par une activité de l'IREM de Lille sur la construction par symétrie centrale.\\
J'ai construit un parcours d'exercices applicatifs sur la construction de symétriques. Le parcours  

S’il est à l’aise et a “réussi” (modalités à définir) son exercice, l’élève continue sur la branche “OK”. Des exercices clefs sont corrigés en plénière en début de séance. Les élèves peuvent également demander de l’aide à l’enseignant ou à des camarades (si les modalités de travail et l’enseignant le permettent).

Des exercices supplémentaires avec des variables didactiques plus complexes peuvent également être prévus en cours de parcours pour les élèves les plus avancés (énoncés donnant peu d’informations, question ouverte, demi-droite graduée à tracer par l’élève, etc.).

Afin de suivre leur propre progression, les élèves colorient leur parcours et notent la date sous le dernier exercice effectué en séance. Je le garde avec moi afin de suivre à distance l’évolution des élèves et d’identifier les exercices ayant posé le plus de problèmes.

Les objectifs sont multiples : 
\begin{itemize}
    \item Permettre à tous les élèves d’acquérir les compétences à son rythme.
    \item Faire gagner les élèves en autonomie, ils deviennent ainsi acteurs de leur formation. 
    \item Pour les élèves en difficulté, proposer des exercices complémentaires avec des variables didactiques différentes et adaptées aux difficultés rencontrées.
    \item Donner aux élèves les plus à l’aise des exercices leur permettant de développer des compétences supplémentaires ou de renforcer les compétences attendues.
\end{itemize}

\textit{<- Fin du bilan d'étape ->}