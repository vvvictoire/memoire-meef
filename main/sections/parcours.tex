\subsection{Parcours différencié (Julia)}

\textit{<- Début du bilan d'étape ->}

Le principe du parcours différencié est que les élèves reçoivent un parcours d’exercices progressifs de la forme suivante :

Le premier exercice est corrigé en classe entière afin de vérifier que les instructions ont bien été comprises et que les connaissances anciennes nécessaires sont acquises. Chaque élève progresse ensuite à son rythme, pour chaque exercice terminé il dispose du corrigé au bureau de l’enseignant. S’il fait un trop gros nombre d’erreurs sur l’exercice ou s’il ne se sent pas suffisamment à l’aise, l’élève continue le parcours sur la branche “KO” lorsqu’elle est disponible (ou demander un exercice supplémentaire à l’enseignant pour la prochaine séance). L’exercice de la branche “KO” est généralement un exercice dont l’énoncé va être similaire à l’exercice précédent, avec des variables didactiques différentes (ex : comparaison de nombres entiers positifs, puis décimaux positifs, puis entiers relatifs, puis décimaux relatifs suite à un exercice avec des entiers et décimaux relatifs à comparer)

S’il est à l’aise et a “réussi” (modalités à définir) son exercice, l’élève continue sur la branche “OK”. Des exercices clefs sont corrigés en plénière en début de séance. Les élèves peuvent également demander de l’aide à l’enseignant ou à des camarades (si les modalités de travail et l’enseignant le permettent).

Des exercices supplémentaires avec des variables didactiques plus complexes peuvent également être prévus en cours de parcours pour les élèves les plus avancés (énoncés donnant peu d’informations, question ouverte, demi-droite graduée à tracer par l’élève, etc.).

Afin de suivre leur propre progression, les élèves colorient leur parcours et notent la date sous le dernier exercice effectué en séance. Je le garde avec moi afin de suivre à distance l’évolution des élèves et d’identifier les exercices ayant posé le plus de problèmes.

Les objectifs sont multiples : 
\begin{itemize}
    \item Permettre à tous les élèves d’acquérir les compétences à son rythme.
    \item Faire gagner les élèves en autonomie, ils deviennent ainsi acteurs de leur formation. 
    \item Pour les élèves en difficulté, proposer des exercices complémentaires avec des variables didactiques différentes et adaptées aux difficultés rencontrées.
    \item Donner aux élèves les plus à l’aise des exercices leur permettant de développer des compétences supplémentaires ou de renforcer les compétences attendues.
\end{itemize}

\textit{<- Fin du bilan d'étape ->}