\subsection{Parcours différencié (Julia)}

\paragraph{}
Le principe du parcours différencié est de proposer aux élèves de traiter une série d'exercices avec une progression adaptée aux difficultés que ceux-ci peuvent rencontrer. Ce parcours peut être complété par des aides progressives données par l'enseignant ou sous différents supports (capsules vidéos, document avec rappel de cours, questions intermédiaires\ldots ).
\paragraph{}
Avant de mettre en place les parcours différenciés, j'ai eu besoin de voir comment les élèves réagissaient face à une successions de travaux à faire en autonomie. J'ai donc mis en place une séance d'exercices en autonomie avec ma classe de sixième. \\
Les élèves ont commencé avec un exercice sur les fractions à faire seul, la correction et les exercices suivants se trouvant sur un îlot. J'ai ajouté une version "bis" de certains exercices disponibles pour les élèves ayant rencontré des difficultés ou voulant se rassurer. \\
Les élèves ont globalement joué le jeu, ils ont apprécié de pouvoir avancer à leur  rythme et de circuler librement dans la salle même si cela a causé de l'agitation en début de cours. Certains élèves ont cependant choisi de ne faire qu'un ou deux exercices simples, alors que d'autres se sont rapidement jetés sur les corrections.\\
\newline
Lors de cette expérience, je me suis rendue compte des élèves qui devaient a priori me poser des problèmes d'attitude étaient plus épanouis dans ce type de travail (lorsqu'ils faisaient l'effort d'essayer), alors que des élèves très à l'aise en condition \emph{classique} de travail individuel ont été perdus au démarrage de la première séance. De plus, une fois le dispositif en place, des élèves habituellement très discrets m'ont demandé de l'aide. J'ai cependant eu de grandes difficultés à \emph{jongler} entre les élèves. Ceux-ci étaient encore peu autonomes et avaient surtout besoin d'être rassurés. J'ai également eu du mal à estimer le temps à laisser à chacun pour travailler seul, et à identifier les moments où instaurer le travail en commun.\\
Cela m'a confortée dans l'idée que la construction de parcours différenciés, avec des étapes clés et à réaliser en autonomie par les élèves pouvait avoir un apport pédagogique intéressant pour mes classes, mais qu'il demandait un travail d'organisation en amont plus important.

\paragraph{Premier réel parcours}
J'ai décidé d'effectuer ma première expérimentation de parcours différencié avec ma classe de cinquième pour la séquence \emph{Symétrie centrale}. Nous avons commencé la séquence par un travail de rappels sur la symétrie axiale, suivie par une activité de l'IREM de Lille sur la construction par symétrie centrale.\\
J'ai construit un parcours d'exercices applicatifs sur la construction de symétriques pour les deux séances suivantes. J'ai distribué le parcours sans la feuille d'exercices aux élèves afin de m'assurer que les modalités ont bien été comprises. Le nouveau mode de travail a déstabilisé les élèves dans un premier temps, mais une fois face aux exercices, je n'ai eu que deux élèves \textit{perdus}.\\
Pour s'aider, les élèves disposaient de l'ensemble de leurs ressources (manuels, cours, exercices déjà effectués). Ils devaient effectuer une série de six exercices autour de la construction de symétriques et disposaient d'une feuille de correction consultable à mon bureau\cite{parcours_symetrie_centrale}. Pour trois des exercices, si les élèves faisaient une erreur ou ne se sentaient pas suffisamment à l'aise, ils devaient continuer le parcours sur la branche "KO" et effectuer un exercice similaire (mêmes variables didactiques, sauf l'exercice 1 où j'ai remplacé des dessins à la main par des figures géométriques pour faciliter la compréhension).\\
Une fois les élèves en exercice, j'ai pu assister les élèves les plus en difficulté. J'ai remarqué qu'une majorité des élèves levait la main sans réellement chercher, ce qui parasite le mode de travail. Je m'attendais cependant à ce phénomène, car les élèves n'étaient pas suffisamment autonomes et familiers avec l'exercice.\\
Deux élèves ont terminé la feuille d'exercice dès la première séance. Je les ai autorisé à assister des camarades en demande d'aide dans un premier temps. Je leur ai ensuite donné des exercices de construction de symétriques supplémentaires (constructions de symétriques de figures sur feuille blanche, avec centre de symétrique hors puis dans la figure).\\
Afin de suivre leur propre progression, les élèves ont colorié leur parcours afin de retrouver leur progression à la séance suivante. Les élèves ont collé cette progression avec la feuille d'exercices sur leur cahier.\\
Le parcours a été plutôt bénéfique pour les élèves en difficulté car j'ai pu expliquer à certains leurs erreurs, d'autres ont compris par eux-même ou avec l'aide d'un camarade. Dans tous les cas, l'exercice de réinvestissement supplémentaire a été réussi (hormis pour une élève en grande difficulté et dans le refus d'écoute, à qui j'ai du redonner une feuille d'exercices).\\
Suite à la bonne conduite des séances d'exercices, et pour améliorer le travail en autonomie des élèves, je leur ai fait travailler la suite de la leçon à travers des exercices de découverte des propriétés de la symétrie centrale. Les élèves devaient travailler en autonomie, avec la possibilité d'analyser des constructions sur papier blanc et/ou sur carreaux. Encore une fois, les corrections étaient disponibles sur le bureau (dont sur papier calque pour les constructions)\\
Les élèves ont été à l'aise avec le travail en autonomie mais n'ont généralement pas su faire le bilan de leur travail.

\paragraph{Bilan de l'expérience :}Les élèves qui ont accepté de témoigner m'ont dit apprécier ce format de travail, "mais pas pour tout le temps". Ces élèves m'ont dit qu'ils s'étaient sentis perdus au départ mais qu'ils avaient plaisir à avancer à leur rythme, sans avoir  à attendre ou à se précipiter selon les cas. Les plus rapides m'ont dit ne pas avoir eu l'impression d'avoir travaillé lorsqu'ils réalisaient le parcours.\\
\textbf{RETOUR suite à eval}\\

C'est suite à ce parcours que j'ai réalisé l'importance de poser des \textit{balises} pour la correction d'exercices ou la mise en commun du travail accompli. J'ai également eu la mauvaise surprise de comprendre à quel point les élèves ne savent pas apprendre. Que cela soit de leurs propres découvertes ou des leçons travaillées en classe.\\
J'ai aussi remarqué que les élèves n'avaient pas les mêmes acquis en construction géométrique et en raisonnement, et que cela posait problème dans le déroulé du parcours différencié. En effet, des élèves à l'aise dans l'observation des situations de symétries se sont trouvés incapables de construire le symétrique d'un point car ils ne savaient pas utiliser leur règle ou leur compas correctement.\\
J'aurai pu disposer de ces informations et adapter les exercices proposés en effectuant une évaluation diagnostique en amont (sur la symétrie axiale par exemple).\\
Le parcours différencié suivant ayant été construit en parallèle de l'expérimentation de ce parcours, je n'ai pas pu effectuer de diagnostique en amont. J'ai cependant effectué une évaluation diagnostique pour une troisième expérimentation, qui sera détaillée par la suite.

\paragraph{Seconde expérimentation : un parcours semi-différencié\\}
La seconde différenciation a été effectuée sur ma classe de sixième autour du calcul avec les nombres décimaux.\\
Les élèves ont reçu une première feuille d'exercices composée de calculs aux difficultés progressives, et suivant un parcours différencié. Dans le parcours de l'expérimentation précédente, j'ai été gênée par les termes "OK" et "KO" qui pourraient être incompris ou mal interprétés. Le premier terme peut être vu comme une validation moyenne de l'exercice, "OK j'ai vite-fait compris" alors que le second terme peut être interprété comme "J'ai été mis KO par l'exercice". J'ai donc choisi de passer par des visages souriants ou non pour indiquer l'état d'esprit suite à la résolution d'un exercice. Les élèves étant de plus hyperconnectés, je n'ai pas eu de doute quant à l'interprétation de ces \textit{smileys}.
\paragraph{}Pour cette série d'exercices, j'ai mis d'avantage de possibilités de parcours, ce qui n'a pas été facile à comprendre pour les élèves. Afin de m'assurer que tout le monde a compris comment réaliser le parcours, j'ai fait faire le premier calcul aux élèves et nous avons décidé ensemble du chemin à parcourir pour chacun en fonction du résultat trouvé. À cette étape de la séance, j'ai découvert que les élèves n'étaient pas à l'aise avec les \textit{smileys} et auraient préféré avoir des termes comme "OK"/"KO" ou "Facile"/"Oups". Les élèves se demandaient "pourquoi il y a un bonhomme triste", ce qui allait à l'encontre de l'objectif du \textit{smiley}.\\
Pour éviter les aller-retour incessants des élèves, j'ai disposé en bas de la feuille d'exercice les réponses aux calculs. Je ne les ai pas disposés les uns en dessous des autres mais en colonne afin d'éviter que les élèves ne lisent involontairement les réponses en avance.
\paragraph{}J'ai également ajouté une étape "Aide" au parcours pour les élèves faisant des erreurs à des étapes clefs. Cet ajout a été compris et bien accueilli par les élèves qui savaient quand demander de l'aide et quand essayer de comprendre par eux-même. Les élèves ayant terminé rapidement et sans erreur le premier exercice devaient me montrer leur feuille pour que je vérifie que les calculs ont bien été posés. Une fois cette vérification terminée, les plus rapides et volontaires pouvaient m'accompagner pour aider un camarade (objectif : apprendre à expliquer sans donner la réponse). Une fois qu'un tiers des élèves a terminé le parcours, les élèves pouvaient passer au second parcours.
\paragraph{Bilan de l'expérimentation\\}
Dans l'ensemble, j'ai été agréablement surprise par l'esprit de travail et d'entre-aide qui a régné durant les séances liées à cette séquence. La séance aurait pu se dérouler de manière fluide si je n'avais cependant pas eu de problème avec des élèves très demandeurs d'attention qui refusent de travailler si on n'est pas à côté d'eux. Ces quatre élèves ont gêné l'ensemble de la classe et l'un d'entre eux a même refusé tout travail et a dû être sanctionné.\\
Le reste de la classe a terminé l'exercice et a su, avec mon aide, faire un bilan des techniques de calcul posé. J'ai remarqué moins d'erreurs de calcul par la suite lors de résolution de problèmes mais ne saurait dire s'ils sont dus au travail en autonomie ou au contenu même de la séquence.\\
Avec du recul, je pense que j'aurai dû rendre la prise en note du bilan obligatoire pour tous les élèves. Ceux-ci ont cependant avancé de manière hétérogène, ce qui a posé problème sur où écrire ce bilan. J'en ai retenu qu'on pouvait prévoir un espace bilan sur la feuille d'exercices, afin qu'elle soit au même endroit pour tous, et facile à retrouver. Après les parcours, les élèves devaient résoudre une série de problèmes et j'ai eu d'avantage de participation lors de la correction, avec beaucoup d'élèves demandant à exposer leur méthode résolution.
\paragraph{}Les élèves les plus à l'aise et ne souhaitant pas aider un camarade ont cependant fini par épuiser le travail supplémentaire que j'avais prévu, ils ont donc commencé la suite de la séquence et ont même eu l'autorisation de travailler une autre matière pour l'une d'entre eux. Cela me pose la question du travail à demander à ces élèves à l'issue du parcours. J'envisage de leur demander de réaliser une carte mentale autour de la séquence et de ce qu'ils ont compris et employé lors de la séance. Nous avons d'autant plus expérimenté le jeu "Math's up"\cite{Maths_up} en demi-groupe, les élèves ont apprécié l'aspect jeu et j'ai observé une amélioration du réemploi des notions par la suite en classe. Les élèves les plus avancés pourraient donc travailler sur quelles notions réemployer pour le jeu par exemple (et de partir de ces notions pour construire une carte mentale en bilan de la séance).

\paragraph{Troisième expérimentation\\}
À partir des bilans autour des deux premières séances et suite à des échanges avec Xavière, Victoire, Nadine Grapin et Marine Doceul (tutrice terrain), j'ai rapidement compris qu'il me manquait trois points importants dans mes parcours différenciés :
\begin{itemize}
	\item une évaluation diagnostique sur la notion (cycle 3) ou sur les prérequis de la séquence ;
	\item des balises claires dans le parcours et pour les élèves sur les instants où le travail doit être présenté au professeur, où il faut demander de l'aide ou lorsqu'il faut attendre une mise en commun ;
	\item un ou des moyens d'évaluer l'impact des travaux sur l'acquisition des compétences et des connaissances des élèves, dont le travail en autonomie.
\end{itemize}
\paragraph{} Pour la dernière expérimentation décrite dans le cadre du mémoire, j'ai donc choisi d'évaluer les élèves de sixième sur la séquence \textit{Quadrilatères et triangles particuliers}. L'évaluation diagnostique et son analyse se trouvent en annexe \cite{eval_diag_quadrilatères}. J'ai profité de devoir rendre l'analyse d'une évaluation pour le portfolio pour présenter cette évaluation diagnostique.\\
Suite à cette évaluation, j'ai pu mettre en avant \textbf{Nombre exact} types d'erreurs :
\begin{itemize}
	\item erreur 1 : X élèves
	\item erreur 1 : X élèves
	\item erreur 1 : X élèves
\end{itemize}
Certains élèves appartiennent à plusieurs groupes et un certain nombre d'élèves n'ont pas répondu à l'évaluation.\\
Après analyse des difficultés de chacun, j'ai mis en place une série de parcours par catégories \cite{parcours_diff_julia3} :
\begin{itemize}
\item Rectangles, losanges et carrés
\item Parallélogramme et trapèze
\item Triangle rectangle
\item Les angles
\item Triangle isocèle et triangle équilatéral
\end{itemize}
Les élèves devront traiter certaines catégories d'exercices en priorité.
Les élèves à l'aise partout devront tout de même se soumettre aux exercices de chaque catégorie ne serait-ce que pour réinvestir leurs connaissances, voire aider leurs camarades (compétence \textit{travailler en groupe}).\\
Une fois qu'ils ont terminé une série d'exercices, par exemple sur les triangles rectangles, si je les considère suffisamment en avance, ils pourront mettre leur nom au tableau dans la colonne \textit{Tuteur triangle rectangle}. De même, les élèves dans le besoin pourront mettre leur nom dans la colonne \textit{À aider triangle rectangle} par exemple.\\
J'ai modifié le plan de table de la classe pour rapprocher des élèves au profil "tuteur" et "demandeur d'aide" et en mettant les élèves en grande difficulté ensemble sur un îlot pour mieux les accompagner.
Les élèves seront munis de leur Ordival et auront chargé un dossier depuis un serveur avant la séance pour disposer de toutes les ressources dont ils pourraient avoir besoin (capsules vidéos, exercices geogebra, images \ldots).
Cette expérimentation demande un travail d'organisation important, je compte expérimenter en demi-groupe dans un premier temps, puis en classe entière.
\paragraph{Bilan évaluation des dispositifs}
evaluation travailler en autonomie sur pronote, 2h
questionnaire
retour évaluation
\paragraph{Autres expérimentation autour de la différenciation et l'autonomie des élèves}
TP Excel\\
Projet anglais/math
\paragraph{Annexes à ajouter}
\begin{itemize}
	\item Parcours différencié symétrie centrale
	\item Compositions des élèves témoins
	\item évaluations
	\item questionnaire
	\item Parcours différencié calcul + problèmes
	\item Compositions des élèves témoins
	\item \ldots
\end{itemize}