\begin{abstract}
Ce mémoire rassemble les réflexions et expérimentations de 3 professeurs stagiaires
à propos de différentes méthodes de différenciation. Le but de ces méthodes
étant d'avoir des techniques applicables généralement, en non spécialisées sur
certaines séquences.
\paragraph*{}Le mémoire présente dans un premier temps la découverte de la notion de différenciation pédagogique et le choix d'orienter la problématique vers l'autonomie des élèves, et ce à travers les lectures et expérimentations effectuées au cours du stage.\\
Les auteurs y justifient ainsi leurs choix de pratiques en répondant aux questions telles que « \textit{Pourquoi différencier ?} », « \textit{Que signifie différencier ?} », « \textit{Quelles pratiques pouvons-nous raisonnablement expérimenter lors de l'année de stage ?} », « \textit{Comment mesurer l'impact des pratiques expérimentées ?} ».\\
Les différents choix de pratiques pédagogiques sont ensuite expliqués puis détaillés dans des parties individuelles. Le document comporte enfin une analyse croisée des expérimentations effectuées par chacune, ainsi que l'état des réflexions des auteurs et les difficultés rencontrées par celles-ci autour des pratiques de différenciation pédagogique et de mise en autonomie des élèves.\\
Le but commun entre les réflexions et expérimentations est de rendre les élèves
plus autonomes tout en leur proposant une acquisition des compétences différenciée, sur différents degrés.
\end{abstract}

Mots-clés : \textbf{différenciation}, \textbf{méthodes}, \textbf{autonomie}, \textbf{pédagogie}, \textbf{tutorat}, \textbf{évalutations différenciées}, \textbf{parcours différenciés}, \textbf{supports différenciés}

\vfill
\section*{Remerciements}
Nous tenons à remercier Nadine Grapin d'avoir encadré ce mémoire et pour l'intérêt qu'elle a porté à nos pratiques pédagogiques de manière générale.\\
Merci également à nos tuteurs de l'ESPÉ respectifs, Alain Bernard et Marie-Hélène Le Yaouanq pour leur enseignement et leurs bons conseils tout au long de cette année. De même, nous remercions nos tutrices terrain Sandrine Adam, Marine Doceul et Claire Hizembert pour leur accompagnement dans nos établissement et pour tout ce qu'elles ont apporté à l'amélioration de nos pratiques.\\
De manière générale nous tenons à remercier les collègues formateurs de l'ESPÉ et de nos établissements respectifs pour leur bienveillance et pour avoir pris le temps de nous aider à nous améliorer.\\
Nous remercions enfin nos proches pour nous avoir soutenues lors de cette année de stage et dans notre projet d'enseignement.
\vfill
Nous autorisons l'ESPÉ :
\begin{itemize}
\item à exploiter le texte de notre mémoire dans la future formation des étudiants MEEF : \textbf{\textsc{Oui}} ;
\item à communiquer notre nom et coordonnées à de futurs étudiants MEEF qui souhaiteraient nous contacter au sujet de notre mémoire : \textbf{\textsc{Oui}}.
\end{itemize}
