\section{Conclusion}
Nous avons chacune trouvé que nos dispositifs permettaient de répondre en partie à nos problématiques autour du gain en autonomie et en compétence par un travail différencié \remark{à modifier selon le choix de la problématique!!}. Nous avons cependant trouvé que le temps de préparation de ces dispositifs était long, surtout lors d'une année de stage où nous avons besoin de plus de temps pour préparer nos séquences et séances. De plus, les problèmes de gestion de classe ne nous ont pas permis d'expérimenter dans des conditions optimales. Nos dispositifs pédagogiques nous ont tout de même permis d'en apprendre plus sur nos élèves, sur nos capacités à gérer leurs attentes et leurs difficultés.\\
Nous avons le sentiment que varier les dispositifs de différenciation pédagogiques permettait d'atteindre de différentes manière une plus grande autonomie de nos élèves. Cette autonomie gagnée permet elle-même à nos élèves de travailler plus efficacement sur des dispositifs différenciés à pratiquer en autonomie (et à apprendre à apprendre), et donc à maîtriser plus facilement les compétences attendues, en passant moins de temps à se demander quoi faire, comment, avec quelles informations\ldots \\
Ce travail de mémoire a clairement enrichi notre réflexion d'enseignantes en nous donnant des pistes de recherche sur les pratiques pédagogiques, en nous habituant à échanger sur nos pratiques avec nos collègues et surtout en nous apprenant à remettre continuellement nos pratiques et notre vision de l'enseignement en question. Nous avons apprécié travailler ensemble sur des pratiques différentes car cela nous a permis de découvrir plus de pratiques sur un an. Nous n'avions pas imaginé qu'autant de dispositifs existaient en matière de différenciation et d'autonomie !
\remark{Autre chose à dire?}