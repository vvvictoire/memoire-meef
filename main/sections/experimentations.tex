\section{Expérimentations}

\textit{<- Début du bilan d'étape ->}

Nous travaillons chacune de notre côté avec des points hebdomadaires de partage de connaissances et expériences. Nous avons également beaucoup d’échanges informels en dehors de l’ESPÉ sur un salon de discussion virtuel dédié. Nous discutons aussi chacune de notre côté avec nos encadrants et collègues, et rapportons les remarques au trinôme.
Afin de plus facilement retrouver ce que nous avons acté lors d’un point d’échange, nous mettons nos comptes-rendus sur Mahara, ainsi que nos fiches de lecture qui sont mises en commun avec un autre groupe de mémoire qui travaille aussi sur la différenciation (Fanny, Léa et Morgane).

Concernant la rédaction du mémoire en lui-même, nous allons utiliser LaTeX, outil avec lequel nous avons l’habitude de travailler, avec une gestion des versions et des commentaires sur GitHub. Ceci nous permettra de très facilement gérer et suivre les modifications de chacune et de passer moins de temps sur la mise en page.
Xavière souhaitait rédiger un article (ayant déjà rédigé un article scientifique par le passé pour valider un autre master). C’est à nouveau en réflexion, étant donné que la charge de travail a grandement augmenté depuis le dernier point.

Nous travaillerons chacune de notre côté sur la mise en place des pratiques que nous souhaitons étudier. Nous ferons cependant toutes les semaines une mise en commun de ce que nous avons effectué et des observations que nous avons faites.

\textit{<- Fin du bilan d'étape ->}