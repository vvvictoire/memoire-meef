\begin{frame}
    \frametitle{Multiplication des supports}
    \vfill
    \begin{itemize}
\item différencier en axant sur les images mentales
\item diversifier les images mentales (toutes ne fonctionnent pas sur tous les élèves) grâce aux supports
\begin{itemize}
\item cartes mentales
\item manipulations manuelles/ gestuelles
\item utilisation de Geogebra par les élèves
\item vidéos explicatives
\item fiches de méthodologie cartonnées
\end{itemize}
\end{itemize}
\vfill
\textbf{Les élèves créent eux-mêmes leurs outils} pour apprendre à travailler en autonomie et favoriser l'appropriation d'un concept.
\end{frame}

\begin{frame}
    \frametitle{Exemples de supports utilisés}
    \vfill
    \textbf{En géométrie} (symétrie axiale, symétrie centrale, 5\up{e})
    \begin{enumerate}
        \item Créer une paire de cocottes en papier
        \item Travailler les symétries en bougeant les cocottes\\(aucun tracé attendu)
        \item Contrôler les constructions sur papier à l'aide des cocottes
        \item Cocottes remplacées par des mouvements de main
        \item (alternative) apprendre à s'auto-corriger avec Geogebra
    \end{enumerate}
    
    \vfill
    
    \textbf{En calcul} (nombres relatifs, deux approches, 5\up{e} et 4\up{e})
    \begin{itemize}
        \item Approche vectorielle avec l'image d'un ascenseur/thermomètre...
        \item Approche ensembliste basée sur beaucoup de narration fantasy
    \end{itemize}
\end{frame}
