\begin{frame}
    \frametitle{Principe du parcours différenciés}
	Les élèves \textbf{effectuent à leur rythme} une série d'exercices \textbf{adaptés à leurs difficultés}.\\
	\vspace*{1.5cm}
	\textbf{insertion parcours différencié}\\
	Difficultés :
	\begin{itemize}
		\item Choix des exercices
		\item Fonctionnement en classe, gestion de la progressions
		\item Mise en place de \textit{balises}
	\end{itemize}
\end{frame}

\begin{frame}
	\frametitle{Construire un parcours différencié adapté}
	\vspace*{0.5cm}
	\'{E}tapes clefs de la construction du parcours différencié idéal :\\
	\begin{enumerate}
		\item \'{E}valuation diagnostique
		\item \'{E}laboration de cours ou aides sur divers supports
		\item Choix des exercices du parcours
		\item Choix du rythme des séances et des moments de reprise
		\item Construction des évaluations intermédiaire et sommative
	\end{enumerate}
\end{frame}
