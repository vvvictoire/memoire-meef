\begin{frame}
    \frametitle{Principe du parcours différencié}
    \vspace*{0.8cm}
	Les élèves \textbf{effectuent à leur rythme} une série d'exercices \textbf{adaptés à leurs difficultés}.\\
	\vspace*{1cm}
	Difficultés :
	\begin{itemize}
		\item Choix des exercices
		\item Fonctionnement en classe, gestion de la progressions
		\item Mise en place de \textit{balises}
		\item Évaluation de l'impact
	\end{itemize}
\end{frame}

\begin{frame}
	\frametitle{Construire un parcours différencié adapté}
	\vspace*{0.8cm}
	\'{E}tapes clefs de la construction d'un parcours différencié aussi adapté que possible :\\
	\begin{enumerate}
		\item \'{E}valuation diagnostique
		\item \'{E}laboration de cours ou aides sur différents supports
		\item Choix des exercices du parcours
		\item Choix du rythme des séances et des moments de reprise
		\item Construction des évaluations intermédiaire(s) et sommative
	\end{enumerate}
\end{frame}
